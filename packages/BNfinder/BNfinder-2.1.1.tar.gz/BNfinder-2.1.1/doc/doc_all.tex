\documentclass{howto}
%\documentclass{article}
\usepackage[T1]{fontenc}
\usepackage[utf8]{inputenc}
\usepackage{graphicx}
%\usepackage{html}
\usepackage{amsmath}
\usepackage{amssymb}

\newcommand{\N}{\mathcal{N}}
\newcommand{\D}{\mathcal{D}}
\newcommand{\G}{\mathcal{G}}
\newcommand{\V}{\mathcal{V}}
\newcommand{\Vc}{\{low, high\}}
\newcommand{\g}{\hat{g}}
\newcommand{\OO}{\mathcal{O}}
\newcommand{\X}{\mathbf{X}}
\newcommand{\Y}{\mathbf{Y}}
\newcommand{\x}{\mathbf{x}}
\newcommand{\vp}{\mathbf{v}}
\newcommand{\PP}{\mathbf{P}}
\newcommand{\E}{\mathbf{E}}
\newcommand{\Pa}{\mathbf{Pa}}


\newtheorem{theorem}{Theorem}
\newtheorem{proposition}{Proposition}
\newtheorem{alg}{Algorithm}{\bfseries}{}
\newtheorem{ass}{Assumption}{\bfseries}{\itshape}

%\newcommand{\url}{}
\release{2.1}
\title{ BNfinder documentation}
\author{Norbert Dojer, Paweł Bednarz, Agnieszka Podsiadło i Bartek Wilczyński}
\begin{document}
\maketitle
\tableofcontents
\newpage

\section{Users manual}
\label{sec:man}

\documentclass[dvips]{scrreprt}
\usepackage{pyx,tabularx,graphicx}
\usepackage[latin1]{inputenc}
\usepackage{url}
\usepackage{hyperref}
\usepackage{units}
\hypersetup{
   pdftitle={PyX User Manual},
   pdfauthor={J�rg Lehmann, Andr� Wobst},
   colorlinks={true}
}
% \includeonly{graph} % for temporarily skipping of chapters
\setlength{\parindent}{0pt}
\begin{document}
\subject{\href{http://pyx.sourceforge.net/}{\url{http://pyx.sourceforge.net/}}}
\title{\PyX{} 
\input pyxversion.tex
\\User Manual}
\author{J\"org Lehmann \href{mailto:joergl@users.sourceforge.net}{\url{<joergl@users.sourceforge.net>}}\and
Andr\'e Wobst \href{mailto:wobsta@users.sourceforge.net}{\url{<wobsta@users.sourceforge.net>}}}
\maketitle
\quad\vfill\centerline{PostScript is a trademark of Adobe Systems Incorporated.}
\tableofcontents
\chapter{Introduction}
\label{intro}

\PyX{} is a Python package for the creation of vector drawings. As
such it allows one to readily generate encapsulated PostScript files
by providing an abstraction of the PostScript graphics model.  Based
on this layer and in combination with the full power of the Python
language itself, the user can just code any complexity of the figure
wanted. \PyX{} distinguishes itself from other similar solution by its
\TeX{}/\LaTeX{} interface that enables one to make directly use of the
famous high quality typesetting of these programs.

A major part of \PyX{} on top of the already described basis is the
provision of high level functionality for complex tasks like 2d plots
in publication-ready quality.

\section{Organisation of the \PyX{} package}

The \PyX{} package is split in several modules, which can be
categorised in the following groups

\begin{tableii}{l|l}{textrm}{Functionality}{Modules}
  basic graphics functionality &   \module{canvas}, \module{path}, \module{deco}, \module{style}, \module{color},
  and \module{connector}
  \\
  text output via \TeX{}/\LaTeX{} &   \module{text} and \module{box}
  \\
  linear transformations and units &   \module{trafo} and \module{unit}
  \\
  graph plotting functionality &  \module{graph} (including sub modules)
  and \module{graph.axis} (including sub modules)
  \\
  EPS file inclusion & \module{epsfile}
\end{tableii}

These modules (and some other less import ones) are imported into the
module namespace by using 
\begin{verbatim}
from pyx import *
\end{verbatim}
at the beginning of your Python program.  However, in order to prevent
namespace pollution, you may also simply use \samp{import pyx}.
Throughout this manual, we shall always assume that former import line
form has been used.



%%% Local Variables:
%%% mode: latex
%%% TeX-master: "manual.tex"
%%% ispell-dictionary: "british"
%%% End:

\chapter{Module unit}
\label{unit}

\sectionauthor{J\"org Lehmann}{joergl@users.sourceforge.net}

\declaremodule{}{unit}

With the \verb|unit| module \PyX{} makes available classes and
functions for the specification and manipulation of lengths. As usual,
lengths consist of a number together with a measurement unit, e.g.,
\unit[1]{cm}, \unit[50]{points}, \unit[0.42]{inch}.  In addition,
lengths in \PyX{} are composed of the five types ``true'', ``user'',
``visual'', ``width'', and ``\TeX'', e.g., \unit[1]{user cm},
\unit[50]{true points}, $(0.42\ \mathrm{visual} + 0.2\ 
\mathrm{width})$ inch.  As their names indicate, they serve different
purposes. True lengths are not scalable and are mainly used for return
values of \PyX{} functions.  The other length types can be rescaled by
the user and differ with respect to the type of object they are
applied to:

\begin{description}
\item[user length:] used for lengths of graphical objects like
  positions etc.
\item[visual length:] used for sizes of visual elements, like arrows,
  graph symbols, axis ticks, etc.
\item[width length:] used for line widths
\item[\TeX{} length:] used for all \TeX{} and \LaTeX{} output
\end{description}

    When not specified otherwise, all types of lengths are interpreted
in terms of a default unit, which, by default, is \unit[1]{cm}.
You may change this default unit by using the module level function
\begin{funcdesc}{set}{uscale=None, vscale=None, wscale=None,
xscale=None, defaultunit=None}
When \var{uscale}, \var{vscale}, \var{wscale}, or \var{xscale} is not
\keyword{None}, the corresponding scaling factor(s) is redefined to
the given number. When \var{defaultunit} is not \keyword{None}, 
the default unit is set to the given value, which has to be
one of \code{"cm"}, \code{"mm"}, \code{"inch"}, or \code{"pt"}.
\end{funcdesc}

For instance, if you only want thicker lines for a publication
version of your figure, you can just rescale all width lengths using
\begin{verbatim}
unit.set(wscale=2)
\end{verbatim}
Or suppose, you are used to specify length in imperial units. In this,
admittedly rather unfortunate case, just use
\begin{verbatim}
unit.set(defaultunit="inch")
\end{verbatim}
at the beginning of your program.

\section{Class length}

\begin{classdesc}{length}{f, type="u", unit=None}
The constructor of the \class{length} class expects as its first
argument a number \var{f}, which represents the prefactor of the given length.
By default this length is interpreted as a user length (\code{type="u"}) in units
of the current default unit (see \function{set()} function of the \module{unit}
module). Optionally, a different \var{type} may be specified, namely
\code{"u"} for user lengths, \code{"v"} for visual lengths, \code{"w"}
for width lengths, \code{"x"} for \TeX{} length, and \code{"t"} for true
lengths. Furthermore, a different unit may be specified using the \var{unit}
argument. Allowed values are \code{"cm"}, \code{"mm"}, \code{"inch"},
and \code{"pt"}.
\end{classdesc}

Instances of the \class{length} class support addition and substraction either by another \class{length}
or by a number which is then interpeted as being a user length in 
default units, multiplication by a number and division either by another
\class{length} in which case a float is returned or by a number in which
case a \class{length} instance is returned. When two lengths are
compared, they are first converted to meters (using the currently set
scaling), and then the resulting values are compared.

\section{Predefined length instances}

A number of \verb|length| instances are already predefined, which
only differ in there values for \verb|type| and \verb|unit|. They are
summarized in the following table

\medskip
\begin{center}
\begin{tabular}{lll|lll}
name & type & unit & name & type & unit\\
\hline
\constant{m} & user & m & \constant{v\_m} & visual & m\\
\constant{cm} & user & cm & \constant{v\_cm} & visual & cm\\
\constant{mm} & user & mm & \constant{v\_mm} & visual & mm\\
\constant{inch} & user & inch & \constant{v\_inch} & visual & inch\\
\constant{pt} & user & points & \constant{v\_pt} & visual & points\\
\constant{t\_m} & true & m & \constant{w\_m} & width & m\\
\constant{t\_cm} & true & cm & \constant{w\_cm} & width & cm\\
\constant{t\_mm} & true & mm & \constant{w\_mm} & width & mm\\
\constant{t\_inch} & true & inch & \constant{w\_inch} & width & inch\\
\constant{t\_pt} & true & points & \constant{w\_pt} & width & points\\
\constant{u\_m} & user & m & \constant{x\_m} & \TeX & m \\
\constant{u\_cm} & user & cm & \constant{x\_cm} & \TeX & cm \\
\constant{u\_mm} & user & mm & \constant{x\_mm} & \TeX & mm \\
\constant{u\_inch} & user & inch & \constant{x\_inch} & \TeX & inch \\
\constant{u\_pt} & user & points & \constant{x\_pt} & \TeX & points\\

\end{tabular}
\end{center}
\medskip

Thus, in order to specify, e.g., a length of 5 width points, just use
\code{5*unit.w_pt}.

\section{Conversion functions}
If you want to know the value of a \PyX{} length in certain units, you
may use the predefined conversion functions which are given in the
following table
\begin{center}
\begin{tabular}{ll}
function & result \\
\hline
\texttt{tom(l)} & \texttt{l} in units of m\\
\texttt{tocm(l)} & \texttt{l} in units of cm\\
\texttt{tomm(l)} & \texttt{l} in units of mm\\
\texttt{toinch(l)} & \texttt{l} in units of inch\\
\texttt{topt(l)} & \texttt{l} in units of points\\
\end{tabular}
\end{center}
If \verb|l| is not yet a \verb|length| instance but a number, it first
is interpreted as a user length in the default units. 



%%% Local Variables:
%%% mode: latex
%%% TeX-master: "manual.tex"
%%% End:

\section{Module \module{path}}

\sectionauthor{J\"org Lehmann}{joergl@users.sourceforge.net} 

\label{path}

The \module{path} module defines several important classes which are
documented in the present section.

\subsection{Class \class{path} --- PostScript-like paths}

\label{path:path}

\declaremodule{}{path}

\begin{classdesc}{path}{*pathels}
This class represents a PostScript like path consisting of the
path elements \var{pathels}.

All possible path elements are described in Sect.~\ref{path:pathel}.
Note that there are restrictions on the first path element and likewise
on each path element after a \class{closepath} directive. In both cases,
no current point is defined and the path element has to be an instance
of one of the following classes: \class{moveto}, \class{arc}, and
\class{arcn}.
\end{classdesc}

Instances of the class \class{path} provide the following
methods (in alphabetic order):

\begin{methoddesc}{append}{pathel}
Appends a \var{pathel} to the end of the path.
\end{methoddesc}

\begin{methoddesc}{arclen}{}
Returns the total arc length of the path.$^\dagger$
\end{methoddesc}

\begin{methoddesc}{arclentoparam}{lengths}
  Returns the parameter values corresponding to the arc lengths
  \var{lengths}.$^\dagger$
\end{methoddesc}

\begin{methoddesc}{at}{param=None, arclen=None}
  Returns the coordinates (as 2-tuple) of the path point corresponding to the
  parameter value \var{param} or, alternatively, the arc length
  \var{arclen}. The parameter value \var{param} (\var{arclen}) has to be smaller
  or equal to \method{self.range()} (\method{self.arclen()}),
  otherwise an exception is raised.  At discontinuities in the path,
  the limit from below is returned.$^\dagger$
\end{methoddesc}

\begin{methoddesc}{bbox}{}
  Returns the bounding box of the path. Note that this returned
  bounding box may be too large, if the path contains any
  \class{curveto} elements, since for these the control box, i.e., the
  bounding box enclosing the control points of the B\'ezier curve is
  returned.
\end{methoddesc}

\begin{methoddesc}{begin}{}
  Returns the coordinates (as 2-tuple) of the first point of the path.$^\dagger$
\end{methoddesc}

\begin{methoddesc}{curvradius}{param=None, arclen=None}
  Returns the curvature radius (or None if infinite) at parameter
  param or, alternatively, arc length \var{arclen}.  This is the
  inverse of the curvature at this parameter Please note that this
  radius can be negative or positive, depending on the sign of the
  curvature.$^\dagger$
\end{methoddesc}

\begin{methoddesc}{end}{}
  Returns the coordinates (as 2-tuple) of the end point of the path.$^\dagger$
\end{methoddesc}

\begin{methoddesc}{intersect}{opath}
  Returns a tuple consisting of two lists of parameter values
  corresponding to the intersection points of the path with the other
  path \var{opath}, respectively.$^\dagger$
\end{methoddesc}

\begin{methoddesc}{joined}{opath}
  Appends \var{opath} to the end of the path, thereby merging the last
  sub path (which must not be closed) of the path with the first sub
  path of \var{opath} and returns the resulting new path.$^\dagger$
\end{methoddesc}

\begin{methoddesc}{range}{}
  Returns the maximal parameter value \var{param} that is allowed in the
  path methods.
\end{methoddesc}

\begin{methoddesc}{reversed}{}
  Returns the reversed path.$^\dagger$
\end{methoddesc}

\begin{methoddesc}{split}{params}
  Splits the path at the parameters \var{params}, which have to be
  sorted in ascending order, and returns a corresponding list of
  \class{normpath} instances.$^\dagger$
\end{methoddesc}

\begin{methoddesc}{tangent}{param=None, arclen=None, length=None}
  Return a \class{line} instance corresponding to the tangent vector
  to the path at the parameter value \var{param} or, alternatively, the arc length
  \var{arclen}. The parameter value \var{param} (\var{arclen}) has to be smaller
  or equal to \method{self.range()} (\method{self.arclen()}),
  otherwise an exception is raised.  At discontinuities in the path,
  the limit from below is returned. If \var{length} is not
  \texttt{None}, the tangent vector will be scaled correspondingly.$^\dagger$
\end{methoddesc}


\begin{methoddesc}{trafo}{param=None, arclen=None}
  Returns a trafo which maps a point $(0, 1)$ to the tangent vector to
  the path at the parameter value \var{param} or, alternatively, the
  arc length \var{arclen}.  The parameter value \var{param} (\var{arclen}) has to
  be smaller or equal to \method{self.range()}
  (\method{self.arclen()}), otherwise an exception is raised.  At
  discontinuities in the path, the limit from below is returned.$^\dagger$
\end{methoddesc}

\begin{methoddesc}{transformed}{trafo}
  Returns the path transformed according to the linear transformation
  \var{trafo}. Here, \texttt{trafo} must be an instance of the
  \class{trafo.trafo} class.$^\dagger$
\end{methoddesc}

Some notes on the above:
\begin{itemize}
\item The $\dagger$ denotes methods which require a prior
  conversion of the path into a \class{normpath} instance. This is
  done automatically, but if you need to call such methods often or
  if you need to change the precision used for this conversion, 
  it is a good idea to manually perform the conversion.
\item Instead of using the \method{joined()} method, you can also join two
paths together with help of the \verb|<<| operator, for instance
\samp{p = p1 << p2}.
\item In the methods accepting both a parameter value \var{param} and
  an arc length \var{arclen}, exactly one of these arguments has to
  provided.
\end{itemize}

\subsection{Path elements}

\label{path:pathel}

The class \class{pathel} is the superclass of all PostScript path
construction primitives. It is never used directly, but only by
instantiating its subclasses, which correspond one by one to the
PostScript primitives.

Except for the path elements ending in \code{_pt}, all coordinates
passed to the path elements can be given as number (in which case they
are interpreted as user units with the currently set default type) or in
\PyX\ lengths.

The following operation move the current point and open a new sub path:

\begin{classdesc}{moveto}{x, y}
Path element which sets the current point to the absolute coordinates (\var{x},
\var{y}). This operation opens a new subpath.
\end{classdesc}

\begin{classdesc}{rmoveto}{dx, dy}
Path element which moves the current point by (\var{dx}, \var{dy}).
This operation opens a new subpath.
\end{classdesc}

Drawing a straight line can be accomplished using:

\begin{classdesc}{lineto}{x, y}
Path element which appends a straight line from the current point to the
point with absolute coordinates (\var{x}, \var{y}), which becomes
the new current point.
\end{classdesc}

\begin{classdesc}{rlineto}{dx, dy}
Path element which appends a straight line from the current point to the
a point with relative coordinates (\var{dx}, \var{dy}), which becomes
the new current point.
\end{classdesc}

For the construction of arc segments, the following three operations
are available:

\begin{classdesc}{arc}{x, y, r, angle1, angle2}
Path element which appends an arc segment in counterclockwise direction
with absolute coordinates (\var{x}, \var{y}) of the center and 
radius \var{r} from \var{angle1} to \var{angle2} (in degrees).
If before the operation, the current point is defined, a straight line
is from the current point to the beginning of the arc segment is
prepended. Otherwise, a subpath, which thus is the first one in the
path, is opened. After the operation, the current point is at the end
of the arc segment.
\end{classdesc}

\begin{classdesc}{arcn}{x, y, r, angle1, angle2}
Path element which appends an arc segment in clockwise direction
with absolute coordinates (\var{x}, \var{y}) of the center and 
radius \var{r} from \var{angle1} to \var{angle2} (in degrees).
If before the operation, the current point is defined, a straight line
is from the current point to the beginning of the arc segment is
prepended. Otherwise, a subpath, which thus is the first one in the
path, is opened. After the operation, the current point is at the end
of the arc segment.
\end{classdesc}

\begin{classdesc}{arct}{x1, y1, x2, y2, r}
Path element which appends an arc segment of radius \var{r}
connecting between (\var{x1}, \var{y1}) and (\var{x2}, \var{y2}).\\
\end{classdesc}

B\'ezier curves can be constructed using: \

\begin{classdesc}{curveto}{x1, y1, x2, y2, x3, y3}
Path element which appends a B\'ezier curve with
the current point as first control point and the other control points
(\var{x1}, \var{y1}), (\var{x2}, \var{y2}), and (\var{x3}, \var{y3}).
\end{classdesc}

\begin{classdesc}{rcurveto}{dx1, dy1, dx2, dy2, dx3, dy3}
Path element which appends a B\'ezier curve with
the current point as first control point and the other control points
defined relative to the current point by the coordinates
(\var{dx1}, \var{dy1}), (\var{dx2}, \var{dy2}), and (\var{dx3}, \var{dy3}).
\end{classdesc}

Note that when calculating the bounding box (see Sect.~\ref{bbox}) of
B\'ezier curves, \PyX{} uses for performance reasons the so-called
control box, i.e., the smallest rectangle enclosing the four control
points of the B\'ezier curve. In general, this is not the smallest
rectangle enclosing the B\'ezier curve.

Finally, an open sub path can be closed using:

\begin{classdesc}{closepath}{}
Path element which closes the current subpath.
\end{classdesc}

For performance reasons, two non-PostScript path elements are defined, 
which perform multiple identical operations:

\begin{classdesc}{multilineto_pt}{points}
Path element which appends straight line segments starting from
the current point and going through the list of points given 
in the \var{points} argument. All coordinates have to 
be given in PostScript points.
\end{classdesc}

\begin{classdesc}{multicurveto_pt}{points}
Path element which appends B\'ezier curve segments starting from
the current point and going through the list of each three control
points given in the \var{points} argument.
\end{classdesc}


\subsection{Class \class{normpath}}

\label{path:normpath}

The \class{normpath} class represents a specialized form of a
\class{path} containing only the elements \class{moveto},
\class{lineto}, \class{curveto} and \class{closepath}. Such normalized
paths are used for all of the more sophisticated path operations
which are denoted by a $\dagger$ in the description of the \class{path}
class above.

Any path can easily be converted to its normalized form by passing it
as parameter to the \class{normpath} constructor,
\begin{verbatim}
np = normpath(p)
\end{verbatim}
Additionally, you can specify the accuracy (in points) which is used
in all \class{normpath} calculations by means of the keyword argument
\var{epsilon}, which defaults to $10^{-5}$.  Note that the sum of a
\class{normpath} and a \class{path} always yields a \class{normpath}.

\begin{classdesc}{normpath}{arg=[], epsilon=1e-5}
  Construct a normpath from \var{arg}. All numerical calculations will
  be performed using an accuracy of the order of \var{epsilon} points.
  The first argument \var{arg} can be a \class{path} or a another
  \class{normpath} instance. Alternatively, a list of
  \class{normsubpath} instances can be supplied as argument.
\end{classdesc}

In addition to the \class{path} methods, a \class{normpath} instance
also offers the following methods, which operate on the instance itself:

\begin{methoddesc}{join}{other}
  Join \var{other}, which has to be a \class{path} instance, to
  the \class{normpath} instance.
\end{methoddesc}

\begin{methoddesc}{reverse}{}
  Reverses the \class{normpath} instance.
\end{methoddesc}

\begin{methoddesc}{transform}{trafo}
  Transforms the \class{normpath} instance according to the linear
  transformation \var{trafo}.
\end{methoddesc}

\subsection{Predefined paths}

\label{path:predefined}

For your convenience, some oft-used paths are already pre-defined. All
of them are sub classes of the \class{path} class.

\begin{classdesc}{line}{x1, y1, x2, y2, x3, y3}
A straight line from the point (\var{x1}, \var{y1}) to the point (\var{x2}, \var{y2}).
\end{classdesc}

\begin{classdesc}{curve}{x1, y1, x2, y2, x3, y3, x4, y4}
A B\'ezier curve with 
control points  (\var{x0}, \var{y0}), $\dots$, (\var{x3}, \var{y3}).\
\end{classdesc}

\begin{classdesc}{rect}{x, y, w, h}
A closed rectangle with lower left point (\var{x}, \var{y}), width \var{w}, and
  height \var{h}.
\end{classdesc}

\begin{classdesc}{circle}{x, y, r}
A closed circle with center (\var{x}, \var{y}) and radius \var{r}.
\end{classdesc}

%%% Local Variables:
%%% mode: latex
%%% TeX-master: "manual.tex"
%%% ispell-dictionary: "british"
%%% End:

\chapter{Module trafo: linear transformations}

\label{trafo}

With the  \verb|trafo| module \PyX\ supports linear transformations, which can 
then be applied to canvases,  B\'ezier paths and other objects. It consists
of the main class \verb|trafo| representing a general linear
transformation and subclasses thereof, which provide special operations
like translation, rotation, scaling, and mirroring.

\section{Class trafo}

The \verb|trafo| class represents a general linear
transformation, which is defined for a vector $\vec{x}$ as
\[
  \vec{x}' = \mathsf{A}\, \vec{x} + \vec{b}\ ,
\]
where $\mathsf{A}$ is the transformation matrix and $\vec{b}$ the
translation vector. The transformation matrix must not be singular,
\textit{i.e.} we require $\det \mathsf{A} \ne 0$.



Multiple \verb|trafo| instances can be multiplied, corresponding to a
consecutive application of the respective transformation. Note that
\verb|trafo1*trafo2| means that \verb|trafo1| is applied after
\verb|trafo2|, \textit{i.e.} the new transformation is given 
by $\mathsf{A} = \mathsf{A}_1 \mathsf{A}_2$ and
$\vec{b} = \mathsf{A}_1 \vec{b}_2 + \vec{b}_1$.  Use the \verb|trafo|
methods described below, if you prefer thinking the other way round.
The inverse of a transformation can be obtained via the \verb|trafo|
method \verb|inverse()|, defined by the inverse $\mathsf{A}^{-1}$ of
the transformation matrix and the translation vector
$-\mathsf{A}^{-1}\vec{b}$.

The methods of the \verb|trafo| class are summarized in the following
table.

\medskip
\begin{tabularx}{\linewidth}{>{\hsize=.8\hsize}X>{\raggedright\arraybackslash\hsize=1.2\hsize}X}
\texttt{trafo} method & function \\
\hline
\texttt{\_\_init\_\_(matrix=((1,0),(0,1)),\newline
\phantom{\_\_init\_\_(}vector=(0,0)):} & create new \texttt{trafo}
instance with transformation \texttt{matrix} and \texttt{vector}.
\\
\texttt{apply(x, y)} & apply \texttt{trafo} to point vector
$(\mathtt{x}, \mathtt{y})$.\\
\texttt{inverse()} & returns inverse transformation of
\texttt{trafo}.\\
\texttt{mirrored(angle)} & returns \texttt{trafo} followed by mirroring
at line through $(0,0)$ with  direction \texttt{angle} in degrees.\\
\texttt{rotated(angle, \newline\phantom{rotate(}x=None, y=None)} &
returns \texttt{trafo} followed by rotation by \texttt{angle} degrees
around point $(\mathtt{x},
\mathtt{y})$, or $(0,0)$, if not given.\\
\texttt{scaled(sx, sy=None,\newline\phantom{scale(}x=None, y=None)} &
returns \texttt{trafo} followed by
scaling with scaling factor \texttt{sx} in $x$-direction, \texttt{sy} in
$y$-direction ($\mathtt{sy}=\mathtt{sx}$, if not given) with scaling
center $(\mathtt{x}, \mathtt{y})$, or $(0,0)$, if not given.\\
\texttt{translated(x, y)} & returns \texttt{trafo} followed by
translation by vector $(\mathtt{x}, \mathtt{y})$.\\
\texttt{slanted(a, angle=0, x=None, y=None)} & returns \texttt{trafo}
followed by XXX\\
\end{tabularx}
\medskip



\section{Subclasses of trafo}

The \verb|trafo| module provides a number of subclasses of
the \verb|trafo| class, each of which corresponds to one \verb|trafo|
method. They are listed in the following table:

\medskip
\begin{tabularx}{\linewidth}{>{\hsize=.7\hsize}X>{\raggedright\arraybackslash\hsize=1.3\hsize}X}
  \texttt{trafo} subclass & function \\
  \hline 
  \texttt{mirror(angle)} & mirroring at line through $(0,0)$
  with direction  \texttt{angle} in degrees.\\
  \texttt{rotate(angle, \newline\phantom{rotation(}x=None, y=None)} &
  rotation by \texttt{angle} degrees around point $(\mathtt{x},
  \mathtt{y})$, or $(0,0)$, if not given.\\
  \texttt{scale(sx, sy=None,\newline\phantom{scaling(}x=None, y=None)} &
  scaling with scaling factor \texttt{sx} in $x$-direction,
  \texttt{sy} in $y$-direction ($\mathtt{sy}=\mathtt{sx}$, if not
  given) with scaling
  center $(\mathtt{x}, \mathtt{y})$, or $(0,0)$, if not given.\\
  \texttt{translate(x, y)} &
  translation by vector $(\mathtt{x}, \mathtt{y})$.\\
  \texttt{slant(a, angle=0, x=None, y=None)} & XXX \\
\end{tabularx}
\medskip


% \section{Examples}



%%% Local Variables:
%%% mode: latex
%%% TeX-master: "manual.tex"
%%% End:

\section{Module \module{canvas}}
\label{canvas}

\sectionauthor{J\"org Lehmann}{joergl@users.sourceforge.net} 

One of the central modules for the PostScript access in \PyX{} is
named \verb|canvas|. Besides providing the class \verb|canvas|, which
presents a collection of visual elements like paths, other canvases,
\TeX{} or \LaTeX{} elements, it contains the class
\texttt{canvas.clip} which allows clipping of the output.

A canvas may also be embedded in another one using its \texttt{insert}
method. This may be useful when you want to apply a transformation on
a whole set of operations..

\declaremodule{}{canvas}

\subsection{Class \class{canvas}}

This is the basic class of the canvas module, which serves to collect
various graphical and text elements you want to write eventually to an
(E)PS file.

\begin{classdesc}{canvas}{attrs=[], texrunner=None}
  Construct a new canvas, applying the given \var{attrs}, which can be
  instances of \class{trafo.trafo}, \class{canvas.clip},
  \class{style.strokestyle} or \class{stlye.fillstyle}.  The
  \var{texrunner} argument can be used to specify the texrunner
  instance used for the \method{text()} method of the canvas.  If not
  specified, it defaults to \var{text.defaulttexrunner}.
\end{classdesc}


Paths can be drawn on the canvas using one of the following methods:

\begin{methoddesc}{draw}{path, attrs}
  Draws \var{path} on the canvas applying the given \var{attrs}.
\end{methoddesc}

\begin{methoddesc}{fill}{path, attrs=[]}
  Fills the given \var{path} on the canvas applying the given
  \var{attrs}. 
\end{methoddesc}

\begin{methoddesc}{stroke}{path, attrs=[]}
  Strokes the given \var{path} on the canvas applying the given
  \var{attrs}.
\end{methoddesc}

Arbitrary allowed elements like other \class{canvas} instances can
be inserted in the canvas using

\begin{methoddesc}{insert}{item, attrs=[]}
  Inserts an instance of \class{base.canvasitem} into the canvas.  If
  \var{attrs} are present, \var{item} is inserted into a new
  \class{canvas}instance with \var{attrs} as arguments passed to its
  constructor is created. Then this \class{canvas} instance is
  inserted itself into the canvas. Returns \var{item}.
\end{methoddesc}

Text output on the canvas is possible using

\begin{methoddesc}{text}{x, y, text, attrs=[]}
  Inserts \var{text} at position (\var{x}, \var{y}) into the
  canvas applying \var{attrs}. This is a shortcut for
  \texttt{insert(texrunner.text(x, y, text, attrs))}).
\end{methoddesc}

The \class{canvas} class provides access to the total geometrical size
of its element:

\begin{methoddesc}{bbox}{}
  Returns the bounding box enclosing all elements of the canvas.
\end{methoddesc}

A canvas also allows one to set global options:

\begin{methoddesc}{set}{styles}
  Sets the given \var{styles} (instances of \class{style.fillstyle} or
  \class{style.strokestyle} or subclasses thereof).  for the rest of
  the canvas.
\end{methoddesc}

\begin{methoddesc}{settexrunner}{texrunner}
  Sets a new \var{texrunner} for the canvas.
\end{methoddesc}

The contents of the canvas can be written using:

\begin{methoddesc}{writeEPSfile}{filename, paperformat=None,
    rotated=0, fittosize=0, margin=1*unit.t_cm, bbox=None,
    bboxenlarge=1*unit.t_pt}
  Writes the canvas to \var{filename} (the extension \texttt{.eps} is
  appended automatically). Optionally, a \var{paperformat} can be
  specified, in which case the output will be centered with respect to
  the corresponding size using the given \var{margin}. See
  \var{canvas.\_paperformats} for a list of known paper formats .  Use
  \var{rotated}, if you want to center on a $90^\circ$ rotated version
  of the respective paper format. If \var{fittosize} is set, the
  output is additionally scaled to the maximal possible size.
  Normally, the bounding box of the canvas is calculated automatically
  from the bounding box of its elements.  Alternatively, you may
  specify the \var{bbox} manually. In any case, the bounding box
  becomes enlarged on all side by \var{bboxenlarge}. This may be used
  to compensate for the inability of \PyX{} to take the linewidths
  into account for the calculation of the bounding box.
\end{methoddesc}

\subsection{Patterns}

The \class{pattern} class allows the definition of PostScript Tiling
patterns (cf.\ Sect.~4.9 of the PostScript Language Reference Manual)
which may then be used to fill paths. The classes \class{pattern} and
\class{canvas} differ only in their constructor and in the absence of
a \method{writeEPSfile()} method in the former. The \class{pattern}
constructor accepts the following keyword arguments:

\medskip
\begin{tabularx}{\linewidth}{l>{\raggedright\arraybackslash}X}
keyword&description\\
\hline
\texttt{painttype}&\texttt{1} (default) for coloured patterns or
\texttt{2} for uncoloured patterns\\
\texttt{tilingtype}&\texttt{1} (default) for constant spacing tilings
(patterns are spaced constantly by a multiple of a device pixel),
\texttt{2} for undistored pattern cell, whereby the spacing may vary
by as much as one device pixel, or \texttt{3} for constant spacing and
faster tiling which behaves as tiling type \texttt{1} but with
additional distortion allowed to permit a more efficient
implementation.\\
\texttt{xstep}&desired horizontal spacing between pattern cells, use
\texttt{None} (default) for automatic calculation from pattern
bounding box.\\
\texttt{ystep}&desired vertical spacing between pattern cells, use
\texttt{None} (default) for automatic calculation from pattern
bounding box.\\
\texttt{bbox}&bounding box of pattern. Use \texttt{None} for an
automatical determination of the bounding box (including an
enlargement by $5$ pts on each side.)\\
\texttt{trafo}&additional transformation applied to pattern or
\texttt{None} (default). This may be used to rotate the pattern or to
shift its phase (by a translation).
\end{tabularx}
\medskip

After you have created a pattern instance, you define the pattern
shape by drawing in it like in an ordinary canvas. To use the pattern,
you simply pass the pattern instance to a \method{stroke()},
\method{fill()}, \method{draw()} or \method{set()} method of the
canvas, just like you would do with a colour, etc.



%%% Local Variables:
%%% mode: latex
%%% TeX-master: "manual.tex"
%%% End:

\chapter[Module \module{text}: TeX/LaTeX interface]{Module \module{text}: \TeX/\LaTeX{} interface}
\label{module:text}

\section{Basic functionality}

The \module{text} module seamlessly integrates Donald E. Knuths famous
\TeX{} typesetting engine into \PyX. The basic procedure is:
\begin{itemize}
\item start a \TeX/\LaTeX{} instance as soon as a \TeX/\LaTeX{}
preamble setting or a text creation is requested
\item create boxes containing the requested text and shipout those
boxes to the dvi file
\item immediately analyse the \TeX/\LaTeX{} output for errors; the box
extents are also contained in the \TeX/\LaTeX{} output and thus become
available immediately
\item when your TeX installation supports the \texttt{ipc} mode and
\PyX{} is configured to use it, the dvi output is also analysed
immediately; alternatively \PyX{} quits the \TeX/\LaTeX{} instance to
read the dvi file once the output needs to be generated or marker
positions are accessed
\item Type1 fonts are used for the PostScript generation
\end{itemize}

Note that for using Type1 fonts an appropriate font mapping file has
to be provided. When your \TeX{} installation is configured to use
Type1 fonts by default, the \texttt{psfonts.map} will contain entries
for the standard \TeX{} fonts already. Alternatively, you may either
look for \texttt{updmap} used by many \TeX{} distributions to create
an appropriate font mapping file. You may also specify one or several
alternative font mapping files like \texttt{psfonts.cmz} in the global
\texttt{pyxrc} or your local \texttt{.pyxrc}. Finally you can also use
the \var{fontmap} keyword argument to a texrunners \method{text}
method to use different mappings within a single outout file.

\section[TeX/LaTeX instances: the \class{texrunner} class]%
{\TeX/\LaTeX{} instances: the \class{texrunner} class}
\declaremodule{}{text}
\modulesynopsis{\TeX/\LaTeX interface}

Instances of the class \class{texrunner} are responsible for executing
and controling a \TeX/\LaTeX{} instance.

\begin{classdesc}{texrunner}{mode="tex",
                             lfs="10pt",
                             docclass="article",
                             docopt=None,
                             usefiles=[],
                             fontmaps=config.get("text", "fontmaps", "psfonts.map"),
                             waitfortex=config.getint("text", "waitfortex", 60),
                             showwaitfortex=config.getint("text", "showwaitfortex", 5),
                             texipc=config.getboolean("text", "texipc", 0),
                             texdebug=None,
                             dvidebug=0,
                             errordebug=1,
                             pyxgraphics=1,
                             texmessagesstart=[],
                             texmessagesdocclass=[],
                             texmessagesbegindoc=[],
                             texmessagesend=[],
                             texmessagesdefaultpreamble=[],
                             texmessagesdefaultrun=[]}
  \var{mode} should the string \samp{tex} or \samp{latex} and defines
  whether \TeX{} or \LaTeX{} will be used. \var{lfs} specifies an
  \texttt{lfs} file to simulate \LaTeX{} font size selection macros in
  plain \TeX. \PyX{} comes with a set of \texttt{lfs} files and a
  \LaTeX{} script to generate those files. For \var{lfs} being
  \code{None} and \var{mode} equals \samp{tex} a list of installed
  \texttt{lfs} files is shown.
  
  \var{docclass} is the document class to be used in \LaTeX{} mode and
  \var{docopt} are the options to be passed to the document class.

  \var{usefiles} is a list of \TeX/\LaTeX{} jobname files. \PyX{} will
  take care of the creation and storing of the corresponding temporary
  files. A typical use-case would be \var{usefiles=["spam.aux"]}, but
  you can also use it to access \TeX{}s log and dvi file.

  \var{fontmaps} is a string containing whitespace separated names of
  font mapping files. \var{waitfortex} is a number of seconds \PyX{}
  should wait for \TeX/\LaTeX{} to process a request. While waiting
  for \TeX/\LaTeX{} a \PyX{} process might seem to do not perform any
  work anymore. To give some feedback to the user, a messages is
  issued each \var{waitfortex} seconds. The \texttt{texipc} flag
  indicates whether \PyX{} should use the \texttt{--ipc} option of
  \TeX/\LaTeX{} for immediate dvi file access to increase the
  execution speed of certain operations. See the output of
  \texttt{tex~--help} whether the option is available at your \TeX{}
  installation.

  \var{texdebug} can be set to a filename to store the commands passed
  to \TeX/\LaTeX{} for debugging. The flag \var{dvidebug} enables
  debugging output in the dvi parser similar to \texttt{dvitype}.
  \var{errordebug} controls the amount of information returned, when
  an texmessage parser raises an error. Valid values are \code{0},
  \code{1}, and \code{2}.

  \var{pyxgraphics} allows use \LaTeX{}s graphics package without
  further configuration of \texttt{pyx.def}.

  The \TeX{} message parsers verify whether \TeX/\LaTeX{} could
  properly process its input. By the parameters
  \var{texmessagesstart}, \var{texmessagesdocclass},
  \var{texmessagesbegindoc}, and \var{texmessagesend} you can set
  \TeX{} message parsers to be used then \TeX/\LaTeX{} is started,
  when the \texttt{documentclass} command is issued (\LaTeX{} only),
  when the \texttt{\textbackslash{}begin\{document\}} is sent, and
  when the \TeX/\LaTeX{} is stopped, respectively. The lists of
  \TeX{} message parsers are merged with the following defaults:
  \code{[texmessage.start]} for \var{texmessagesstart},
  \code{[texmessage.load]} for \var{texmessagesdocclass},
  \code{[texmessage.load, texmessage.noaux]} for
  \var{texmessagesbegindoc}, and \code{[texmessage.texend,
  texmessage.fontwarning]} for \var{texmessagesend}.

  Similarily \var{texmessagesdefaultpreamble} and
  \var{texmessagesdefaultrun} take \TeX{} message parser to be merged
  to the \TeX{} message parsers given in the \method{preamble()} and
  \method{text()} methods. The \var{texmessagesdefaultpreamble} and
  \var{texmessagesdefaultrun} are merged with \code{[texmessage.load]}
  and \code{[texmessage.loaddef, texmessage.graphicsload,
  texmessage.fontwarning, texmessage.boxwarning]}, respectively.
\end{classdesc}

\class{texrunner} instances provides several methods to be called by
the user:

\begin{methoddesc}{set}{**kwargs}
  This method takes the same keyword arguments as the
  \class{texrunner} constructor. Its purpose is to reconfigure an
  already constructed \class{texrunner} instance. The most prominent
  use-case is to alter the configuration of the default
  \class{texrunner} instance \code{defaulttexrunner} which is created
  at the time of loading of the \module{text} module.

  The \verb|set| method fails, when a modification cannot be applied
  anymore (e.g. \TeX/\LaTeX{} has already been started).
\end{methoddesc}

\begin{methoddesc}{preamble}{expr, texmessages=[]}
  The \method{preamble()} can be called prior to the \method{text()}
  method only or after reseting a texrunner instance by
  \method{reset()}. The \var{expr} is passed to the \TeX/\LaTeX{}
  instance not encapsulated in a group. It should not generate any
  output to the dvi file. In \LaTeX{} preamble expressions are
  inserted prior to the \texttt{\textbackslash{}begin\{document\}} and
  a typical use-case is to load packages by
  \texttt{\textbackslash{}usepackage}. Note, that you may use
  \texttt{\textbackslash{}AtBeginDocument} to postpone the
  immediate evaluation.

  \var{texmessages} are \TeX{} message parsers to handle the output of
  \TeX/\LaTeX. They are merged with the default \TeX{} message
  parsers for the \method{preamble()} method. See the constructur
  description for details on the default \TeX{} message parsers.
\end{methoddesc}

\begin{methoddesc}{text}{x, y, expr, textattrs=[], texmessages=[]}
  \var{x} and \var{y} are the position where a text should be typeset
  and \var{expr} is the \TeX/\LaTeX{} expression to be passed to
  \TeX/\LaTeX{}.

  \var{textattrs} is a list of \TeX/\LaTeX{} settings as described
  below, \PyX{} transformations, and \PyX{} fill styles (like colors).

  \var{texmessages} are \TeX{} message parsers to handle the output of
  \TeX/\LaTeX. They are merged with the default \TeX{} message
  parsers for the \method{text()} method. See the constructur
  description for details on the default \TeX{} message parsers.

  The \method{text()} method returns a \class{textbox} instance, which
  is a special \class{canvas} instance. It has the methods
  \method{width()}, \method{height()}, and \method{depth()} to access
  the size of the text. Additionally the \method{marker()} method,
  which takes a string \emph{s}, returns a position in the text, where
  the expression \texttt{\textbackslash{}PyXMarker\{\emph{s}\}} is
  contained in \var{expr}. You should not use \texttt{@} within your
  strings \emph{s} to prevent name clashes with \PyX{}
  internal macros (although we don't the marker feature internally
  right now).
\end{methoddesc}

Note that for the outout generation and the marker access the
\TeX/\LaTeX{} instance must be terminated except when \texttt{texipc} is
turned on. However, after such a termination a new \TeX/\LaTeX{}
instance is started when the \method{text()} method is called again.

\begin{methoddesc}{reset}{reinit=0}
  This method can be used to manually force a restart of
  \TeX/\LaTeX{}. The flag \var{reinit} will initialize the
  \TeX/\LaTeX{} by repeating the \method{preamble()} calls. New
  \method{set()} and \method{preamble()} calls are allowed when
  \var{reinit} was not set only.
\end{methoddesc}


\section[TeX/LaTeX attributes]
{\TeX/\LaTeX{} attributes}
\declaremodule{}{text}
\modulesynopsis{\TeX/\LaTeX interface}

\TeX/\LaTeX{} attributes are instances to be passed to a
\class{texrunner}s \method{text()} method. They stand for
\TeX/\LaTeX{} expression fragments and handle dependencies by proper
ordering.

\begin{classdesc}{halign}{boxhalign, flushhalign}
  Instances of this class set the horizontal alignment of a text box
  and the contents of a text box to be left, center and right for
  \var{boxhalign} and \var{flushhalign} being \code{0}, \code{0.5},
  and \code{1}. Other values are allowed as well, although such an
  alignment seems quite unusual.
\end{classdesc}

Note that there are two separate classes \class{boxhalign} and
\class{flushhalign} to set the alignment of the box and its contents
independently, but those helper classes can't be cleared independently
from each other. Some handy instances available as class members:

\begin{memberdesc}{boxleft}
  Left alignment of the text box, \emph{i.e.} sets \var{boxhalign} to
  \code{0} and doesn't set \var{flushhalign}.
\end{memberdesc}

\begin{memberdesc}{boxcenter}
  Center alignment of the text box, \emph{i.e.} sets \var{boxhalign} to
  \code{0.5} and doesn't set \var{flushhalign}.
\end{memberdesc}

\begin{memberdesc}{boxright}
  Right alignment of the text box, \emph{i.e.} sets \var{boxhalign} to
  \code{1} and doesn't set \var{flushhalign}.
\end{memberdesc}

\begin{memberdesc}{flushleft}
  Left alignment of the content of the text box in a multiline box,
  \emph{i.e.} sets \var{flushhalign} to \code{0} and doesn't set
  \var{boxhalign}.
\end{memberdesc}

\begin{memberdesc}{raggedright}
  Identical to \member{flushleft}.
\end{memberdesc}

\begin{memberdesc}{flushcenter}
  Center alignment of the content of the text box in a multiline box,
  \emph{i.e.} sets \var{flushhalign} to \code{0.5} and doesn't set
  \var{boxhalign}.
\end{memberdesc}

\begin{memberdesc}{raggedcenter}
  Identical to \member{flushcenter}.
\end{memberdesc}

\begin{memberdesc}{flushright}
  Right alignment of the content of the text box in a multiline box,
  \emph{i.e.} sets \var{flushhalign} to \code{1} and doesn't set
  \var{boxhalign}.
\end{memberdesc}

\begin{memberdesc}{raggedleft}
  Identical to \member{flushright}.
\end{memberdesc}

\begin{memberdesc}{left}
  Combines \member{boxleft} and \member{flushleft}, \emph{i.e.}
  \code{halign(0, 0)}.
\end{memberdesc}

\begin{memberdesc}{center}
  Combines \member{boxcenter} and \member{flushcenter}, \emph{i.e.}
  \code{halign(0.5, 0.5)}.
\end{memberdesc}

\begin{memberdesc}{right}
  Combines \member{boxright} and \member{flushright}, \emph{i.e.}
  \code{halign(1, 1)}.
\end{memberdesc}

\begin{figure}
\centerline{\includegraphics{textvalign}}
\caption{valign example}
\label{fig:textvalign}
\end{figure}

\begin{classdesc}{valign}{valign}
  Instances of this class set the vertical alignment of a text box to
  be top, center and bottom for \var{valign} being \code{0},
  \code{0.5}, and \code{1}. Other values are allowed as well, although
  such an alignment seems quite unusual. See the left side of
  figure~\ref{fig:textvalign} for an example.
\end{classdesc}

Some handy instances available as class members:

\begin{memberdesc}{top}
  \code{valign(0)}
\end{memberdesc}

\begin{memberdesc}{middle}
  \code{valign(0.5)}
\end{memberdesc}

\begin{memberdesc}{bottom}
  \code{valign(1)}
\end{memberdesc}

\begin{memberdesc}{baseline}
  Identical to clearing the vertical alignment by \member{clear} to
  emphasise that a baseline alignment is not a box-related alignment.
  Baseline alignment is the default, \emph{i.e.} no valign is set by
  default.
\end{memberdesc}

\begin{classdesc}{parbox}{width, baseline=top}
  Instances of this class create a box with a finite width, where the
  typesetter creates multiple lines in. Note, that you can't create
  multiple lines in \TeX/\LaTeX{} without specifying a box width.
  Since \PyX{} doesn't know a box width, it uses \TeX{}s LR-mode by
  default, which will always put everything into a single line. Since
  in a vertical box there are several baselines, you can specify the
  baseline to be used by the optional \var{baseline} argument. You can
  set it to the symbolic names \member{top}, \member{parbox.middle},
  and \member{parbox.bottom} only, which are members of
  \class{valign}. See the right side of figure~\ref{fig:textvalign}
  for an example.
\end{classdesc}

Since you need to specify a box width no predefined instances are
available as class members.

\begin{classdesc}{vshift}{lowerratio, heightstr="0"}
  Instances of this class lower the output by \var{lowerratio} of the
  height of the string \var{heigthstring}. Note, that you can apply
  several shifts to sum up the shift result. However, there is still a
  \member{clear} class member to remove all vertical shifts.
\end{classdesc}

Some handy instances available as class members:

\begin{memberdesc}{bottomzero}
  \code{vshift(0)} (this doesn't shift at all)
\end{memberdesc}

\begin{memberdesc}{middlezero}
  \code{vshift(0.5)}
\end{memberdesc}

\begin{memberdesc}{topzero}
  \code{vshift(1)}
\end{memberdesc}

\begin{memberdesc}{mathaxis}
  This is a special vertical shift to lower the output by the height
  of the mathematical axis. The mathematical axis is used by \TeX{}
  for the vertical alignment in mathematical expressions and is often
  usefull for vertical alignment. The corresponding vertical shift is
  less than \member{middlezero} and usually fits the height of the
  minus sign. (It is the height of the minus sign in mathematical
  mode, since that's that the mathematical axis is all about.)
\end{memberdesc}

There is a \TeX/\LaTeX{} attribute to switch to \TeX{}s math mode. The
appropriate instances \code{mathmode} and \code{clearmathmode} (to
clear the math mode attribute) are available at module level.

\begin{datadesc}{mathmode}
  Enables \TeX{}s mathematical mode in display style.
\end{datadesc}

The \class{size} class creates \TeX/\LaTeX{} attributes for changing
the font size.

\begin{classdesc}{size}{sizeindex=None, sizename=None,
                        sizelist=defaultsizelist}
  \LaTeX{} knows several commands to change the font size. The command
  names are stored in the \var{sizelist}, which defaults to
  \code{[\textquotedbl{}normalsize\textquotedbl{},
  \textquotedbl{}large\textquotedbl{},
  \textquotedbl{}Large\textquotedbl{},
  \textquotedbl{}LARGE\textquotedbl{},
  \textquotedbl{}huge\textquotedbl{},
  \textquotedbl{}Huge\textquotedbl{},
  None, \textquotedbl{}tiny\textquotedbl{},
  \textquotedbl{}scriptsize\textquotedbl{},
  \textquotedbl{}footnotesize\textquotedbl{},
  \textquotedbl{}small\textquotedbl{}]}.

  You can either provide an index \var{sizeindex} to access an item in
  \var{sizelist} or set the command name by \var{sizename}.
\end{classdesc}

Instances for the \LaTeX{}s default size change commands are available
as class members:

\begin{memberdesc}{tiny}
  \code{size(-4)}
\end{memberdesc}

\begin{memberdesc}{scriptsize}
  \code{size(-3)}
\end{memberdesc}

\begin{memberdesc}{footnotesize}
  \code{size(-2)}
\end{memberdesc}

\begin{memberdesc}{small}
  \code{size(-1)}
\end{memberdesc}

\begin{memberdesc}{normalsize}
  \code{size(0)}
\end{memberdesc}

\begin{memberdesc}{large}
  \code{size(1)}
\end{memberdesc}

\begin{memberdesc}{Large}
  \code{size(2)}
\end{memberdesc}

\begin{memberdesc}{LARGE}
  \code{size(3)}
\end{memberdesc}

\begin{memberdesc}{huge}
  \code{size(4)}
\end{memberdesc}

\begin{memberdesc}{Huge}
  \code{size(5)}
\end{memberdesc}

There is a \TeX/\LaTeX{} attribute to create empty text boxes with the
size of the material passed in. The appropriate instances
\code{phantom} and \code{clearphantom} (to clear the phantom
attribute) are available at module level.

\begin{datadesc}{phantom}
  Skip the text in the box, but keep its size.
\end{datadesc}

\section[Using the graphics-bundle with LaTeX]%
{Using the graphics-bundle with \LaTeX}

The packages in the \LaTeX{} graphics bundle (\texttt{color.sty},
\texttt{graphics.sty}, \texttt{graphicx.sty}, \ldots) make extensive use of
\texttt{\textbackslash{}special} commands. \PyX{} defines a clean set of such
commands to fit the needs of the \LaTeX{} graphics bundle. This is done via the
\texttt{pyx.def} driver file, which tells the graphics bundle about the syntax
of the \texttt{\textbackslash{}special} commands as expected by \PyX{}. You can
install the driver file \texttt{pyx.def} into your \LaTeX{} search path and add
the content of both files \texttt{color.cfg} and \texttt{graphics.cfg} to your
personal configuration files.\footnote{If you do not know what this is all
about, you can just ignore this paragraph. But be sure that the
\var{pyxgraphics} keyword argument is always set!} After you have installed the
\texttt{cfg} files, please use the \module{text} module with unset
\code{pyxgraphics} keyword argument which will switch off a convenience hack
for less experienced \LaTeX{} users. You can then import the \LaTeX{} graphics
bundle packages and related packages (e.g.~\texttt{rotating}, \ldots) with the
option~\texttt{pyx},
e.g.~\texttt{\textbackslash{}usepackage[pyx]\{color,graphicx\}}. Note that the
option~\texttt{pyx} is only available with unset \var{pyxgraphics} keyword
argument and a properly installed driver file. Otherwise, omit the
specification of a driver when loading the packages.

When you define colors in \LaTeX{} via one of the color models \texttt{gray},
\texttt{cmyk}, \texttt{rgb}, \texttt{RGB}, \texttt{hsb}, then \PyX{} will use
the corresponding values (one to four real numbers). In case you use any of the
\texttt{named} colors in \LaTeX{}, \PyX{} will use the corresponding predefined
color (see module \texttt{color} and the color table at the end of the manual).
The additional \LaTeX{} color model \texttt{pyx} allows to use a PyX color
expression, such as \texttt{color.cmyk(0,0,0,0)} directly in LaTeX. It is
passed to PyX.

When importing Encapsulated PostScript files (\texttt{eps} files) \PyX{} will
rotate, scale and clip your file like you expect it. Other graphic formats can
not be imported via the graphics package at the moment.

For reference purpose, the following specials can be handled by \PyX{} at the
moment:

\begin{description}
\item[\texttt{PyX:color\_begin (model) (spec)}]
  starts a color. \texttt{(model)}~is one of
  \texttt{gray}, \texttt{cmyk}, \texttt{rgb}, \texttt{hsb}, \texttt{texnamed}, or
  \texttt{pyxcolor}. \texttt{(spec)}~depends on the model: a name or
  some numbers
\item[\texttt{PyX:color\_end}]
  ends a color.
\item[\texttt{PyX:epsinclude file= llx= lly= urx= ury= width= height= clip=0/1}]
  includes an Encapsulated PostScript file (\texttt{eps}
  files). The values of \texttt{llx} to \texttt{ury} are in the files'
  coordinate system and specify the part of the graphics that should
  become the specified \texttt{width} and \texttt{height} in the
  outcome. The graphics may be clipped. The last three parameters are
  optional.
\item[\texttt{PyX:scale\_begin (x) (y)}]
  begins scaling from the current point.
\item[\texttt{PyX:scale\_end}]
  ends scaling.
\item[\texttt{PyX:rotate\_begin (angle)}]
  begins rotation around the current point.
\item[\texttt{PyX:rotate\_end}]
  ends rotation.
\end{description}

\section[TeX message parsers]%
{\TeX{} message parsers}
\declaremodule{}{text}
\modulesynopsis{\TeX/\LaTeX interface}

Message parsers are used to scan the output of \TeX/\LaTeX. The output
is analysed by a sequence of \TeX{} message parsers. Each message
parser analyses the output and removes those parts of the output, it
feels responsible for. If there is nothing left in the end, the
message got validated, otherwise an exception is raised reporting the
problem. A message parser might issue a warning when removing some
output to give some feedback to the user.

\begin{classdesc}{texmessage}{}
  This class acts as a container for \TeX{} message parsers instances,
  which are all instances of classes derived from \class{texmessage}.
\end{classdesc}

The following \TeX{} message parser instances are available:

\begin{memberdesc}{start}
  Check for \TeX/\LaTeX{} startup message including scrollmode test.
\end{memberdesc}
\begin{memberdesc}{noaux}
  Ignore \LaTeX{}s no-aux-file warning.
\end{memberdesc}
\begin{memberdesc}{end}
  Check for proper \TeX/\LaTeX{} tear down message.
\end{memberdesc}
\begin{memberdesc}{load}
  Accepts arbitrary loading of files without checking for details,
  \emph{i.e.} accept \texttt{(\emph{file} ...)} where
  \texttt{\emph{file}} is an readable file.
\end{memberdesc}
\begin{memberdesc}{loaddef}
  Accepts arbitrary loading of \texttt{fd} files, \emph{i.e.} accept
  \texttt{(\emph{file}.def)} and \texttt{(\emph{file}.fd)} where
  \texttt{\emph{file}.def} or \texttt{\emph{file}.fd} is an readable
  file, respectively.
\end{memberdesc}
\begin{memberdesc}{graphicsload}
  Accepts arbitrary loading of \texttt{eps} files,
  \emph{i.e.} accept \texttt{(\emph{file}.eps)} where
  \texttt{\emph{file}.eps} is an readable file.
\end{memberdesc}
\begin{memberdesc}{ignore}
  Ignores everything (this is probably a bad idea, but sometimes you
  might just want to ignore everything).
\end{memberdesc}
\begin{memberdesc}{allwarning}
  Ignores everything but issues a warning.
\end{memberdesc}
\begin{memberdesc}{fontwarning}
  Issues a warning about font substitutions of the \LaTeX{}s NFSS.
\end{memberdesc}
\begin{memberdesc}{boxwarning}
  Issues a warning on under- and overfull horizontal and vertical boxes.
\end{memberdesc}

\begin{classdesc}{texmessagepattern}{pattern, warning=None}
  This is a derived class of \class{texmessage}. It can be used to
  construct simple \TeX{} message parsers, which validate a \TeX{}
  message matching a certain regular expression pattern \var{pattern}.
  When \var{warning} is set, a warning message is issued. Several of
  the \TeX{} message parsers described above are implemented using
  this class.
\end{classdesc}

\section{The \member{defaulttexrunner} instance}
\declaremodule{}{text}
\modulesynopsis{\TeX/\LaTeX interface}

\begin{datadesc}{defaulttexrunner}
  The \code{defaulttexrunner} is an instance of \class{texrunner}. It
  is created when the \module{text} module is loaded and it is used as
  the default texrunner instance by all \class{canvas} instances to
  implement its \method{text()} method.
\end{datadesc}

\begin{funcdesc}{preamble}{...}
  \code{defaulttexrunner.preamble}
\end{funcdesc}

\begin{funcdesc}{text}{...}
  \code{defaulttexrunner.text}
\end{funcdesc}

\begin{funcdesc}{set}{...}
  \code{defaulttexrunner.set}
\end{funcdesc}

\begin{funcdesc}{reset}{...}
  \code{defaulttexrunner.reset}
\end{funcdesc}

\section{Some internals on temporary files etc.}

It is not totally obvious how \TeX{} processes are supervised by
\PyX{} and why it's done that way. However there are good reasons for
it and the following description is intended for people wanting and/or
needing to understand how temporary files are used by \PyX. All others
don't need to care.

Each time \PyX{} needs to start a new \TeX{} process, it creates a
base file name for temporary files associated with this process. This
file name is used as \verb|\jobname| by \TeX. Since \TeX{} does not
handle directory names as part of \verb|\jobname|, the temporary files
will be created in the current directory. The \PyX{} developers
decided to not change the current directory at all, avoiding all kind
of issues with accessing files in the local directory, like for
loading graph data, \LaTeX{} style files etc.

\PyX{} creates a \TeX{} file containing \verb|\relax| only. It's only
use is to set \TeX{}s \verb|\jobname|. Immediately after processing
\verb|\relax| \TeX{} falls back to stdin to read more commands. \PyX{}
than uses \code{stdin} and \code{stdout} to avoid various buffering
issues which would occur when using files (or named pipes). By that
\PyX{} can fetch \TeX{} errors as soon as they occur while keeping the
\TeX{} process running (i.e. in a waiting state) for further input.
The size of the \TeX{} output is also availble immediately without
fetching the \code{dvi} file created by \TeX, since \PyX{} uses some
\TeX{} macros to output the extents of the boxes created for the
requested texts to \code{stdout} immediately. There is a TeX hack
\verb|--ipc| which \PyX{} knows to take advantage of to fetch
informations from the \code{dvi} file immediately as well, but it's
not available on all \TeX installations. Thus this feature is disabled
by default and fetching informations from the \code{dvi} is tried to
be limited to those cases, where no other option exists. By that
\TeX{} usually doesn't need to be started several times.

By default \PyX{} will clean up all temporary files after \TeX{} was
stopped. However the \code{usefiles} list allows for a renaming of the
files from (and to, if existing) the temporary \verb|\jobname| (+
suffix) handled by \PyX{}. Additionally, since \PyX{} does not write a
useful \TeX{} input file in a file and thus a
\code{usefiles=["example.tex"]} would not contain the code actually
passed to \TeX{}, the \code{texdebug} feature of the texrunner can be
used instead to get a the full input passed to \TeX{}.

In case you need to control the position where the temporary files are
created (say, you're working on a read-only directory), the suggested
solution is to switch the current directory before starting with text
processing in \PyX{} (i.e. an \code{os.chdir} at the beginning of your
script will do fine). You than just need to take care of specifying
full paths when accessing data from your original working directory,
but that's intended and necessary for that case.

\chapter{Module box: convex box handling}
\label{module:box}

This module has a quite internal character, but might still be useful
from the users point of view. It might also get further enhanced to
cover a broader range of standard arranging problems.

In the context of this module a box is a convex polygon having
optionally a center coordinate, which plays an important role for the
box alignment. The center might not at all be central, but it should
be within the box. The convexity is necessary in order to keep the
problems to be solved by this module quite a bit easier and
unambiguous.

Directions (for the alignment etc.) are usually provided as pairs
(dx, dy) within this module. It is required, that at least one of
these two numbers is unequal to zero. No further assumptions are taken.

\section{polygon}

A polygon is the most general case of a box. It is an instance of the
class \verb|polygon|. The constructor takes a list of points (which
are (x, y) tuples) in the keyword argument \verb|corners| and
optionally another (x, y) tuple as the keyword argument \verb|center|.
The corners have to be ordered counterclockwise. In the following list
some methods of this \verb|polygon| class are explained:

\begin{description}
\raggedright
\item[\texttt{path(centerradius=None, bezierradius=None,
beziersoftness=1)}:] returns a path of the box; the center might be
marked by a small circle of radius \verb|centerradius|; the corners
might be rounded using the parameters \verb|bezierradius| and
\verb|beziersoftness|
\item[\texttt{transform(*trafos)}:] performs a list of transformations
to the box
\item[\texttt{reltransform(*trafos)}:] performs a list of
transformations to the box relative to the box center

\begin{figure}
\centerline{\includegraphics{boxalign}}
\caption{circle and line alignment examples (equal direction and
distance)}
\label{fig:boxalign}
\end{figure}

\item[\texttt{circlealignvector(a, dx, dy)}:] returns a vector (a
tuple (x, y)) to align the box at a circle with radius \verb|a| in
the direction (\verb|dx|, \verb|dy|); see figure~\ref{fig:boxalign}
\item[\texttt{linealignvector(a, dx, dy)}:] as above, but align at a
line with distance \verb|a|
\item[\texttt{circlealign(a, dx, dy)}:] as circlealignvector, but
perform the alignment instead of returning the vector
\item[\texttt{linealign(a, dx, dy)}:] as linealignvector, but
perform the alignment instead of returning the vector
\item[\texttt{extent(dx, dy)}:] extent of the box in the direction
(\verb|dx|, \verb|dy|)
\item[\texttt{pointdistance(x, y)}:] distance of the point (\verb|x|,
\verb|y|) to the box; the point must be outside of the box
\item[\texttt{boxdistance(other)}:] distance of the box to the box
\verb|other|; when the boxes are overlapping, \verb|BoxCrossError| is
raised
\item[\texttt{bbox()}:] returns a bounding box instance appropriate to
the box
\end{description}

\section{functions working on a box list}

\begin{description}
\raggedright
\item[\texttt{circlealignequal(boxes, a, dx, dy)}:] Performs a circle
alignment of the boxes \verb|boxes| using the parameters \verb|a|,
\verb|dx|, and \verb|dy| as in the \verb|circlealign| method. For the
length of the alignment vector its largest value is taken for all
cases.
\item[\texttt{linealignequal(boxes, a, dx, dy)}:] as above, but
performing a line alignment
\item[\texttt{tile(boxes, a, dx, dy)}:] tiles the boxes \verb|boxes|
with a distance \verb|a| between the boxes (additional the maximal box
extent in the given direction (\verb|dx|, \verb|dy|) is taken into
account)
\end{description}

\section{rectangular boxes}

For easier creation of rectangular boxes, the module provides the
specialized class \verb|rect|. Its constructor first takes four
parameters, namely the x, y position and the box width and height.
Additionally, for the definition of the position of the center, two
keyword arguments are available. The parameter \verb|relcenter| takes
a tuple containing a relative x, y position of the center (they are
relative to the box extent, thus values between \verb|0| and
\verb|1| should be used). The parameter \verb|abscenter| takes a tuple
containing the x and y position of the center. This values are
measured with respect to the lower left corner of the box. By
default, the center of the rectangular box is set to this lower left
corner.


\chapter{Module connector}
\label{connector}

This module provides classes for connecting two \verb|box|-instances with
lines, arcs or curves.
All constructors of the following connector-classes take two
\verb|box|-instances as first arguments. They return a
\verb|normpath|-instance from the first to the second box, starting/ending at
the boxes' outline \verb|path|. The behaviour of the path is determined by the
boxes' center and some angle- and distance-keywords. The resulting path will
additionally be shortened by lengths given in the \verb|boxdists|-keyword (a
list of two lengths, default \verb|[0,0]|).

\section{Class line}

The constructor of the \verb|line| class accepts only boxes and the
\verb|boxdists|-keyword.

\section{Class arc}

The constructor also takes either the \verb|relangle|-keyword or a combination
of \verb|relbulge| and \verb|absbulge|. The ``bulge'' is the greatest distance
between the connecting arc and the straight connecting line.
(Default: \verb|relangle=45|, \verb|relbulge=None|,
\verb|absbulge=None|)\medskip

Note that the bulge- override the angle-keyword. When both \verb|relbulge| and
\verb|absbulge| are given they will be added.

\section{Class curve}

The construktor takes both angle- and bulge-keywords. Here, the bulges are
used as distances between bezier-curve control points:\medskip

\verb|absangle1| or \verb|relangle1|\\
\verb|absangle2| or \verb|relangle2|, where the absolute angle overrides the
relative if both are given. (Default: \verb|relangle1=45|,
\verb|relangle2=45|, \verb|absangle1=None|, \verb|absangle2=None|)\medskip

\verb|absbulge| and \verb|relbulge|, where they will be added if both are
given.\\ (Default: \verb|absbulge|=None \verb|relbulge|=0.39; these default
values produce similar output like the defaults of the arc-class.)\medskip


Note that relative angle-keywords are counted in the following way:
\verb|relangle1| is counted in negative direction, starting at the straight
connector line, and \verb|relangle2| is counted in positive direction.
Therefore, the outcome with two positive relative angles will always leave the
straight connector at its left and will not cross it.

\section{Class twolines}

This class returns two connected straight lines. There is a vast variety of
combinations for angle- and length-keywords. The user has to make sure to
provide a non-ambiguous set of keywords:\medskip

\verb|absangle1| or \verb|relangle1| for the first angle,\\
\verb|relangleM| for the middle angle and\\
\verb|absangle2| or \verb|relangle2| for the ending angle.
Again, the absolute angle overrides the relative if both are given. (Default:
all five angles are \verb|None|)\medskip

\verb|length1| and \verb|length2| for the lengths of the connecting lines.
(Default: \verb|None|)


\chapter{Module epsfile: EPS file inclusion}

With help of the \verb|epsfile.epsfile| class, you can easily embed
another EPS file in your canvas, thereby scaling, aligning the content
at discretion. The most simple example looks like
\begin{quote}
\begin{verbatim}
from pyx import *
c = canvas.canvas()
c.insert(epsfile.epsfile(0, 0, "file.eps"))
c.writeEPSfile("output")
\end{verbatim}
\end{quote}

All relevant parameters are passed to the \verb|epsfile.epsfile|
constructor. They are summarized in the following table:

\medskip
\begin{tabularx}{\linewidth}{l>{\raggedright\arraybackslash}X}
argument name&description\\
\hline
\texttt{x} & $x$-coordinate of position (measured in user
units by default).\\
\texttt{y} & $y$-coordinate of position (measured in user
units by default).\\
\texttt{filename} & Name of the EPS file (including a possible
extension).\\
\texttt{width=None} & Desired width of EPS graphics or \texttt{None}
for original width. Cannot be combined with scale specification.\\
\texttt{heigth=None} & Desired height of EPS graphics or \texttt{None}
for original height. Cannot be combined with scale specification.\\
\texttt{scale=None} & Scaling factor for EPS graphics or \texttt{None}
for no scaling. Cannot be combined with width or height specification.\\
\texttt{align="bl"} & Alignment of EPS graphics. The first character
specifies the vertical alignment: \texttt{b} for bottom, \texttt{c}
for center, and \texttt{t} for top. The second character fixes the
horizontal alignment: \texttt{l} for left, \texttt{c} for center
\texttt{r} for right.\\
\texttt{clip=1} & Clip to bounding box of EPS file?\\
\texttt{translatebbox=1} & Use lower left corner of bounding box of EPS
file? Set to $0$ with care.\\
\texttt{bbox=None} & If given, use \texttt{bbox} instance instead of
bounding box of EPS file.
\end{tabularx}



\label{epsfile}

%%% Local Variables:
%%% mode: latex
%%% TeX-master: "manual.tex"
%%% End:


\chapter{Module bbox}

\label{bbox}

The \texttt{bbox} module contains the definition of the \texttt{bbox}
class representing bounding boxes of graphical elements like paths,
canvases, etc.\ used in \PyX. Usually, you obtain \texttt{bbox}
instances as return values of the corresponding \texttt{bbox())}
method, but you may also construct a bounding box by yourself.

\section{bbox constructor}

The \texttt{bbox} constructor accepts the following keyword arguments

\medskip
\begin{tabularx}{\linewidth}{l>{\raggedright\arraybackslash}X}
keyword & description\\
\hline
\texttt{llx}&\texttt{None} (default) for $-\infty$ or $x$-position of
the lower left corner of the bbox (in user units)\\
\texttt{lly}&\texttt{None} (default) for $-\infty$ or $y$-position of
the lower left corner of the bbox (in user units)\\
\texttt{urx}&\texttt{None} (default) for $\infty$ or $x$-position of
the upper right corner of the bbox (in user units)\\
\texttt{ury}&\texttt{None} (default) for $\infty$ or $y$-position of
the upper right corner of the bbox (in user units)
\end{tabularx}

\section{bbox methods}

%Instances of the \texttt{bbox} class offer the following methods:
%\medskip

\begin{tabularx}
  {\linewidth}
  {>{\hsize=.85\hsize}X>{\raggedright\arraybackslash\hsize=1.15\hsize}X}
  \texttt{bbox} method & function \\
  \hline
  \texttt{intersects(other)} & returns \texttt{1} if the \texttt{bbox} instance
  and \texttt{other} intersect with each other.\\
  \texttt{transformed(self, trafo)}& returns \texttt{self} transformed
  by transformation \texttt{trafo}.\\
  \texttt{enlarged(all=0, bottom=None,
    \newline\phantom{enlarged(}left=None, top=None,
    \newline\phantom{enlarged(}right=None)} &
  return the bounding box enlarged by the given amount (in visual
  units). \texttt{all} is the default for all other directions, which
  is used whenever \texttt{None} is given for the corresponding
  direction.\\
  \texttt{path()} or \texttt{rect()} & return the \texttt{path} corresponding to the
  bounding box rectangle.\\
  \texttt{height()} & returns the height of the bounding box (in \PyX{}
  lengths).\\
  \texttt{width()} & returns the width of the bounding box (in \PyX{}
  lengths).\\
  \texttt{top()} & returns the $y$-position of the top of the bounding
  box (in \PyX{} lengths).\\
  \texttt{bottom()} & returns the $y$-position of the bottom of the
  bounding box (in \PyX{} lengths).\\
  \texttt{left()} & returns the $x$-position of the left side of the
  bounding box (in \PyX{} lengths).\\
  \texttt{right()} & returns the $x$-position of the right side of the
  bounding box (in \PyX{} lengths).\\
  \end{tabularx}
\medskip

Furthermore, two bounding boxes can be added (giving the bounding box
enclosing both) and multiplied (giving the intersection of both
bounding boxes).

%%% Local Variables:
%%% mode: latex
%%% TeX-master: "manual.tex"
%%% End:

\chapter{Module color}
\label{color}
\section{Color models}
PostScript provides different color models. They are available to
\PyX{} by different color classes, which just pass the colors down to
the PostScript level. This implies, that there are no conversion
routines between different color models available. However, some color
model conversion routines are included in Python's standard library in
the module \texttt{colorsym}. Furthermore also the comparison of
colors within a color model is not supported, but might be added in
future versions at least for checking color identity and for ordering
gray colors.

There is a class for each of the supported color models, namely
\verb|gray|, \verb|rgb|, \verb|cmyk|, and \verb|hsb|. The constructors
take variables appropriate for the color model. Additionally, a list of
named colors is given in appendix~\ref{colorname}.

\section{Example}
\begin{quote}
\begin{verbatim}
from pyx import *

c = canvas.canvas()

c.fill(path.rect(0, 0, 7, 3), [color.gray(0.8)])
c.fill(path.rect(1, 1, 1, 1), [color.rgb.red])
c.fill(path.rect(3, 1, 1, 1), [color.rgb.green])
c.fill(path.rect(5, 1, 1, 1), [color.rgb.blue])

c.writeEPSfile("color")
\end{verbatim}
\end{quote}

The file \verb|color.eps| is created and looks like:
\begin{quote}
\includegraphics{color}
\end{quote}

\section{Color palettes}

The color module provides a class \verb|palette| for transitions between
colors. A list of named palettes is available in appendix~\ref{palettename}.

\begin{classdesc}{palette}{min=0, max=1}
  This class provides the methods for the \verb|palette|. Different
  initializations can be found in \verb|linearpalette| and \verb|functionpalette|.

  \var{min} and \var{max} provide the valid range of the arguments for
  \verb|getcolor|.

  \begin{funcdesc}{getcolor}{parameter}
    Returns the color that corresponds to \var{parameter} (must be between
    \var{min} and \var{max}).
  \end{funcdesc}

  \begin{funcdesc}{select}{index, n\_indices}
    When a total number of \var{n\_indices} different colors is needed from the
    palette, this method returns the \var{index}-th color.
  \end{funcdesc}

\end{classdesc}


\begin{classdesc}{linearpalette}{startcolor, endcolor, min=0, max=1}
  This class provides a linear transition between two given colors. The linear
  interpolation is performed on the color components of the specific color
  model.

  \var{startcolor} and \var{endcolor} must be colors of the same color model.
\end{classdesc}

\begin{classdesc}{functionpalette}{functions, type, min=0, max=1}
  This class provides an arbitray transition between colors of the same
  color model.

  \var{type} is a string indicating the color model (one of \code{"cmyk"},
  \code{"rgb"}, \code{"hsb"}, \code{"grey"})

  \var{functions} is a dictionary that maps the color components onto given
  functions. E.g. for \code{type="rgb"} this dictionary must have the keys
  \code{"r"}, \code{"g"}, and \code{"b"}.

\end{classdesc}

\section{Transparency}

\begin{classdesc}{transparency}{value}
  Instances of this class will make drawing operations (stroking,
  filling) to become partially transparent. \var{value} defines the
  transparency factor in the range \code{0} (opaque) to \code{1}
  (transparent).

  Transparency is available in PDF output only since it is not
  supported by PostScript.
\end{classdesc}


\chapter{Module data}
\label{module:data}

\section{Reading a table from a file}

The module datafile contains the class \verb|datafile| which can be
used to read in a table from a file. You just have to construct an
instance and provide a filename as the parameter, e.g.
\verb|datafile("testdata")|. The parsing of the file, namely the
columns of the table, is done by matching regular expressions. They
can be modified, as they are additional named arguments of the
constructor. Furthermore there is the possibility to skip some of
the data points by some other keyword arguments as listed in the
following table:

\medskip
\begin{tabularx}{\linewidth}{l>{\raggedright\arraybackslash}X}
argument name&description\\
\hline
\texttt{commentpattern}&start a comment line; default: \texttt{re.compile(r"(\#+|!+|\%+)\textbackslash s*")}\\
\texttt{stringpattern}&a string column; default: \texttt{re.compile(r"\textbackslash"(.*?)\textbackslash"(\textbackslash s+|\$)}\\
\texttt{columnpattern}&any other column; default: \texttt{re.compile(r"(.*?)(\textbackslash s+|\$)}\\
\texttt{skiphead}&skip first data lines; default: \texttt{0}\\
\texttt{skiptail}&skip last data lines; default: \texttt{0}\\
\texttt{every}&only take every \texttt{every} data line into account; default: \texttt{1}
\end{tabularx}
\medskip

The processing of the input file is done by reading the file line by
line and first strip leading and tailing whitespaces of the line. Then
a check is performed, whether the line matches the comment pattern or
not. If it does match, this rest of the line is analysed like a table
line when no data was read before (otherwise it is just thrown away).
The result is interpreted as column titles. As the titles are
sequentially overwritten by another comment line previous to the data,
finally the last non-empty comment line determines the column titles.

Thus we have still to explain, how the reading of data lines works. We
create a list of entries for each column out of a given line. A line
resulting in an empty list (e.g. an empty line) is just ignored. As
shown in the table above, there is a special string column pattern.
When it matches it forces the interpretation of a column as a string.
Otherwise \verb|datafile| will try to convert the columns
automatically into floats except for the title line. When the
conversions fails, it just keeps the string.

The default string pattern allows for columns to contain whitespaces.
It matches a string whenever it starts with a quote (\verb|"|) and
then tries to find the end of that very string by another quote
immediately followed by a whitespace or the end of the line. Hence a
quote within a string is just ignored and no kind of escaping is
needed. The only disadvantage is, that you cannot describe a string
which contains a quote and a whitespace consecutively. However, you
can always replace this string pattern to fit your special needs.

Finally the number of columns is fixed to the maximal number contained
in the file and lines with less entries get filled with \verb|None|.
Also the titles list is cutted to this maximal number of columns.

\section{Accessing columns}

The method \verb|getcolumnno| takes a parameter as the column
description. If it matches exactly one entry in the titles list, the
number of this element is returned. Otherwise the parameter should be
an integer and it is checked, if this integer is a valid column index.
Like for other python indices a column number might be negative
counting the columns from the end. When an error occurres, the
exception \verb|ColumnError| is raised. Please note, that the datafile
inserts a first column having the index 0, which contains the line
number (starting at 1 and counting only data lines). Examples are
\verb|getcolumnno(1)| or \verb|getcolumnno("title")|.

The method \verb|getcolumn| takes the same argument as the method
\verb|getcolumnno| described above, but it returns a list with the
values of this very column.

\section{Mathematics on columns}

By the method \verb|addcolumn| a new column is appended. The method
takes a string as the first parameter which is interpreted as an
expression. When the expression contains an equal sign (\verb|=|),
everything left to the last equal sign will become the title of the
new column. If no equal sign is found, the title will be set to
\verb|None|. The part right to the last equal sign is interpreted as
an mathematical expression. A list of functions, predefined variables
and operators can be found in appendix~\ref{mathtree}. The list of
available functions and predefined variables can be extended by a
dictionary passed as the keyword argument \verb|context| to the
\verb|addcolumn| method.

The expression might contain variable names. The interpretation of
this names is done in the following way:
\begin{itemize}
\item The names can be a column title, but this is only allowed for
column titles which are valid variable names (e.g. they should start
with a letter or an underscore and contain only letters, digits and
the underscore).
\item A variable name can start with the dollar symbol (\verb|$|) and
the following integer number will directly refer to a column number.
\end{itemize}
The data referenced by variables in the expression need to be
floats, otherwise the result for that data line will be \verb|None|.

\section{Reading data from a sectioned config file}

The class \verb|sectionfile| provides a reader for files in the
ConfigFile format (see the description of the module \verb|ConfigFile|
from the pyx standard library).

\section{Own datafile readers}

The development of other datafile readers should be based on the
class \verb|data| by inheritance. When doing so, the methods
\verb|getcolumnno|, \verb|getcolumn|, and \verb|addcolumn| are
immediately available and the cooperation with other parts of \PyX{}
is assured. All what has to be done, is a call to the inherited
constructor supplying at least a sequence of data points as the
\verb|data| keyword argument. A data point itself is a sequence of
floats and/or strings. Additionally a sequence of column titles
(strings) might be given in the \verb|titles| argument.

\chapter{Graphs}
\label{graph}

\section{Introduction} % {{{
\PyX{} can be used for data and function plotting. At present
x-y-graphs and x-y-z-graphs are supported only. However, the component
architecture of the graph system described in section
\ref{graph:components} allows for additional graph geometries while
reusing most of the existing components.

Creating a graph splits into two basic steps. First you have to create
a graph instance. The most simple form would look like:
\begin{verbatim}
from pyx import *
g = graph.graphxy(width=8)
\end{verbatim}
The graph instance \code{g} created in this example can then be used
to actually plot something into the graph. Suppose you have some data
in a file \file{graph.dat} you want to plot. The content of the file
could look like:
\verbatiminput{graph.dat}
To plot these data into the graph \code{g} you must perform:
\begin{verbatim}
g.plot(graph.data.file("graph.dat", x=1, y=2))
\end{verbatim}
The method \method{plot()} takes the data to be plotted and optionally
a list of graph styles to be used to plot the data. When no styles are
provided, a default style defined by the data instance is used. For
data read from a file by an instance of \class{graph.data.file}, the
default are symbols. When instantiating \class{graph.data.file}, you
not only specify the file name, but also a mapping from columns to
axis names and other information the styles might make use of
(\emph{e.g.} data for error bars to be used by the errorbar style).

While the graph is already created by that, we still need to perform a
write of the result into a file. Since the graph instance is a canvas,
we can just call its \method{writeEPSfile()} method.
\begin{verbatim}
g.writeEPSfile("graph")
\end{verbatim}
The result \file{graph.eps} is shown in figure~\ref{fig:graph}.

\begin{figure}[ht]
\centerline{\includegraphics{graph}}
\caption[A minimalistic plot for the data from a file.]{A minimalistic plot for the data from file \file{graph.dat}.}
\label{fig:graph}
\end{figure}

Instead of plotting data from a file, other data source are available
as well. For example function data is created and placed into
\method{plot()} by the following line:
\begin{verbatim}
g.plot(graph.data.function("y(x)=x**2"))
\end{verbatim}
You can plot different data in a single graph by calling
\method{plot()} several times before \method{writeEPSfile()} or
\method{writePDFfile()}. Note that a calling \method{plot()} will fail
once a graph was forced to ``finish'' itself. This happens
automatically, when the graph is written to a file. Thus it is not an
option to call \method{plot()} after \method{writeEPSfile()} or
\method{writePDFfile()}. The topic of the finalization of a graph is
addressed in more detail in section~\ref{graph:graph}. As you can see
in figure~\ref{fig:graph2}, a function is plotted as a line by
default.

\begin{figure}[ht]
\centerline{\includegraphics{graph2}}
\caption{Plotting data from a file together with a function.}
\label{fig:graph2}
\end{figure}

While the axes ranges got adjusted automatically in the previous
example, they might be fixed by keyword options in axes constructors.
Plotting only a function will need such a setting at least in the
variable coordinate. The following code also shows how to set a
logathmic axis in y-direction:

\verbatiminput{graph3.py}

The result is shown in figure~\ref{fig:graph3}.

\begin{figure}[ht]
\centerline{\includegraphics{graph3}}
\caption{Plotting a function for a given axis range and use a
logarithmic y-axis.}
\label{fig:graph3}
\end{figure} % }}}

\section{Component architecture} % {{{
\label{graph:components}

Creating a graph involves a variety of tasks, which thus can be
separated into components without significant additional costs.
This structure manifests itself also in the \PyX{} source, where there
are different modules for the different tasks. They interact by some
well-defined interfaces. They certainly have to be completed and
stabilized in their details, but the basic structure came up in the
continuous development quite clearly. The basic parts of a graph are:

\begin{definitions}
\term{graph}
  Defines the geometry of the graph by means of graph coordinates with
  range [0:1]. Keeps lists of plotted data, axes \emph{etc.}
\term{data}
  Produces or prepares data to be plotted in graphs.
\term{style}
  Performs the plotting of the data into the graph. It gets data,
  converts them via the axes into graph coordinates and uses the graph
  to finally plot the data with respect to the graph geometry methods.
\term{key}
  Responsible for the graph keys.
\term{axis}
  Creates axes for the graph, which take care of the mapping from data
  values to graph coordinates. Because axes are also responsible for
  creating ticks and labels, showing up in the graph themselves and
  other things, this task is splitted into several independent
  subtasks. Axes are discussed separately in chapter~\ref{axis}.
\end{definitions} % }}}

\section{Module \module{graph.graph}: Graphs} % {{{
\label{graph:graph}

\declaremodule{}{graph.graph}
\modulesynopsis{Graph geometry}

The classes \class{graphxy} and \class{graphxyz} are part of the
module \module{graph.graph}. However, there are shortcuts to access
the classes via \code{graph.graphxy} and \code{graph.graphxyz},
respectively.

\begin{classdesc}{graphxy}{xpos=0, ypos=0, width=None, height=None,
ratio=goldenmean, key=None, backgroundattrs=None,
axesdist=0.8*unit.v\_cm, xaxisat=None, yaxisat=None, **axes}
  This class provides an x-y-graph. A graph instance is also a fully
  functional canvas.

  The position of the graph on its own canvas is specified by
  \var{xpos} and \var{ypos}. The size of the graph is specified by
  \var{width}, \var{height}, and \var{ratio}. These parameters define
  the size of the graph area not taking into account the additional
  space needed for the axes. Note that you have to specify at least
  \var{width} or \var{height}. \var{ratio} will be used as the ratio
  between \var{width} and \var{height} when only one of these is
  provided.

  \var{key} can be set to a \class{graph.key.key} instance to create
  an automatic graph key. \code{None} omits the graph key.

  \var{backgroundattrs} is a list of attributes for drawing the
  background of the graph. Allowed are decorators, strokestyles, and
  fillstyles. \code{None} disables background drawing.

  \var{axisdist} is the distance between axes drawn at the same side
  of a graph.

  \var{xaxisat} and \var{yaxisat} specify a value at the y and x axis,
  where the corresponding axis should be moved to. It's a shortcut for
  corresonding calls of \method{axisatv()} described below. Moving an
  axis by \var{xaxisat} or \var{yaxisat} disables the automatic
  creation of a linked axis at the opposite side of the graph.

  \var{**axes} receives axes instances. Allowed keywords (axes names)
  are \code{x}, \code{x2}, \code{x3}, \emph{etc.} and \code{y},
  \code{y2}, \code{y3}, \emph{etc.} When not providing an \code{x} or
  \code{y} axis, linear axes instances will be used automatically.
  When not providing a \code{x2} or \code{y2} axis, linked axes to the
  \code{x} and \code{y} axes are created automatically and \emph{vice
  versa}. As an exception, a linked axis is not created automatically
  when the axis is placed at a specific position by \var{xaxisat} or
  \var{yaxisat}. You can disable the automatic creation of axes by
  setting the linked axes to \code{None}. The even numbered axes are
  plotted at the top (\code{x} axes) and right (\code{y} axes) while
  the others are plotted at the bottom (\code{x} axes) and left
  (\code{y} axes) in ascending order each.
\end{classdesc}

Some instance attributes might be useful for outside read-access.
Those are:

\begin{memberdesc}{axes}
  A dictionary mapping axes names to the \class{anchoredaxis} instances.
\end{memberdesc}

To actually plot something into the graph, the following instance
method \method{plot()} is provided:

\begin{methoddesc}{plot}{data, styles=None}
  Adds \var{data} to the list of data to be plotted. Sets \var{styles}
  to be used for plotting the data. When \var{styles} is \code{None},
  the default styles for the data as provided by \var{data} is used.

  \var{data} should be an instance of any of the data described in
  section~\ref{graph:data}.

  When the same combination of styles (\emph{i.e.} the same
  references) are used several times within the same graph instance,
  the styles are kindly asked by the graph to iterate their
  appearance. Its up to the styles how this is performed.

  Instead of calling the plot method several times with different
  \var{data} but the same style, you can use a list (or something
  iterateable) for \var{data}.
\end{methoddesc}

While a graph instance only collects data initially, at a certain
point it must create the whole plot. Once this is done, further calls
of \method{plot()} will fail. Usually you do not need to take care
about the finalization of the graph, because it happens automatically
once you write the plot into a file. However, sometimes position
methods (described below) are nice to be accessible. For that, at
least the layout of the graph must have been finished. By calling the
\method{do}-methods yourself you can also alter the order in which the
graph components are plotted. Multiple calls to any of the
\method{do}-methods have no effect (only the first call counts). The
orginal order in which the \method{do}-methods are called is:

\begin{methoddesc}{dolayout}{}
  Fixes the layout of the graph. As part of this work, the ranges of
  the axes are fitted to the data when the axes ranges are allowed to
  adjust themselves to the data ranges. The other \method{do}-methods
  ensure, that this method is always called first.
\end{methoddesc}

\begin{methoddesc}{dobackground}{}
  Draws the background.
\end{methoddesc}

\begin{methoddesc}{doaxes}{}
  Inserts the axes.
\end{methoddesc}

\begin{methoddesc}{doplotitem}{plotitem}
  Plots the plotitem as returned by the graphs plot method.
\end{methoddesc}

\begin{methoddesc}{doplot}{}
  Plots all (remaining) plotitems.
\end{methoddesc}

\begin{methoddesc}{dokeyitem}{}
  Inserts a plotitem in the graph key as returned by the graphs plot method.
\end{methoddesc}

\begin{methoddesc}{dokey}{}
  Inserts the graph key.
\end{methoddesc}

\begin{methoddesc}{finish}{}
  Finishes the graph by calling all pending \method{do}-methods. This
  is done automatically, when the output is created.
\end{methoddesc}

The graph provides some methods to access its geometry:

\begin{methoddesc}{pos}{x, y, xaxis=None, yaxis=None}
  Returns the given point at \var{x} and \var{y} as a tuple
  \code{(xpos, ypos)} at the graph canvas. \var{x} and \var{y} are
  anchoredaxis instances for the two axes \var{xaxis} and \var{yaxis}.
  When \var{xaxis} or \var{yaxis} are \code{None}, the axes with names
  \code{x} and \code{y} are used. This method fails if called before
  \method{dolayout()}.
\end{methoddesc}

\begin{methoddesc}{vpos}{vx, vy}
  Returns the given point at \var{vx} and \var{vy} as a tuple
  \code{(xpos, ypos)} at the graph canvas. \var{vx} and \var{vy} are
  graph coordinates with range [0:1].
\end{methoddesc}

\begin{methoddesc}{vgeodesic}{vx1, vy1, vx2, vy2}
  Returns the geodesic between points \var{vx1}, \var{vy1} and
  \var{vx2}, \var{vy2} as a path. All parameters are in graph
  coordinates with range [0:1]. For \class{graphxy} this is a straight
  line.
\end{methoddesc}

\begin{methoddesc}{vgeodesic\_el}{vx1, vy1, vx2, vy2}
  Like \method{vgeodesic()} but this method returns the path element to
  connect the two points.
\end{methoddesc}

% dirty hack to add a whole list of methods to the index:
\index{xbasepath()@\texttt{xbasepath()} (graphxy method)}
\index{xvbasepath()@\texttt{xvbasepath()} (graphxy method)}
\index{xgridpath()@\texttt{xgridpath()} (graphxy method)}
\index{xvgridpath()@\texttt{xvgridpath()} (graphxy method)}
\index{xtickpoint()@\texttt{xtickpoint()} (graphxy method)}
\index{xvtickpoint()@\texttt{xvtickpoint()} (graphxy method)}
\index{xtickdirection()@\texttt{xtickdirection()} (graphxy method)}
\index{xvtickdirection()@\texttt{xvtickdirection()} (graphxy method)}
\index{ybasepath()@\texttt{ybasepath()} (graphxy method)}
\index{yvbasepath()@\texttt{yvbasepath()} (graphxy method)}
\index{ygridpath()@\texttt{ygridpath()} (graphxy method)}
\index{yvgridpath()@\texttt{yvgridpath()} (graphxy method)}
\index{ytickpoint()@\texttt{ytickpoint()} (graphxy method)}
\index{yvtickpoint()@\texttt{yvtickpoint()} (graphxy method)}
\index{ytickdirection()@\texttt{ytickdirection()} (graphxy method)}
\index{yvtickdirection()@\texttt{yvtickdirection()} (graphxy method)}

Further geometry information is available by the \member{axes}
instance variable, with is a dictionary mapping axis names to
\class{anchoredaxis} instances. Shortcuts to the anchoredaxis
positioner methods for the \code{x}- and \code{y}-axis become
available after \method{dolayout()} as \class{graphxy} methods
\code{Xbasepath}, \code{Xvbasepath}, \code{Xgridpath},
\code{Xvgridpath}, \code{Xtickpoint}, \code{Xvtickpoint},
\code{Xtickdirection}, and \code{Xvtickdirection} where the prefix
\code{X} stands for \code{x} and \code{y}.

\begin{methoddesc}{axistrafo}{axis, t}
  This method can be used to apply a transformation \var{t} to an
  \class{anchoredaxis} instance \var{axis} to modify the axis position
  and the like. This method fails when called on a not yet finished
  axis, i.e. it should be used after \method{dolayout()}.
\end{methoddesc}

\begin{methoddesc}{axisatv}{axis, v}
  This method calls \method{axistrafo()} with a transformation to move
  the axis \var{axis} to a graph position \var{v} (in graph
  coordinates).
\end{methoddesc}

The class \class{graphxyz} is very similar to the \class{graphxy}
class, except for its additional dimension. In the following
documentation only the differences to the \class{graphxy} class are
described.

\begin{classdesc}{graphxyz}{xpos=0, ypos=0, size=None,
                            xscale=1, yscale=1, zscale=1/goldenmean,
                            projector=central(10, -30, 30), key=None,
                            **axes}
  This class provides an x-y-z-graph.

  The position of the graph on its own canvas is specified by
  \var{xpos} and \var{ypos}. The size of the graph is specified by
  \var{size} and the length factors \var{xscale}, \var{yscale}, and
  \var{zscale}. The final size of the graph depends on the projector
  \var{projector}, which is called with \code{x}, \code{y}, and
  \code{z} values up to \var{xscale}, \var{yscale}, and  \var{zscale}
  respectively and scaling the result by \var{size}. For a parallel
  projector changing \var{size} is thus identical to changing
  \var{xscale}, \var{yscale}, and \var{zscale} by the same factor. For
  the central projector the projectors internal distance would also
  need to be changed by this factor. Thus \var{size} changes the size
  of the whole graph without changing the projection.

  \var{projector} defines the conversion of 3d coordinates to 2d
  coordinates. It can be an instance of \class{central} or
  \class{parallel} described below.

  \var{**axes} receives axes instances as for \class{graphxyz}. The
  graphxyz allows for 4 axes per graph dimension \code{x}, \code{x2},
  \code{x3}, \code{x4}, \code{y}, \code{y2}, \code{y3}, \code{y4},
  \code{z}, \code{z2}, \code{z3}, and \code{z4}. The x-y-plane is the
  horizontal plane at the bottom and the \code{x}, \code{x2},
  \code{y}, and \code{y2} axes are placed at the boundary of this
  plane with \code{x} and \code{y} always being in front. \code{x3},
  \code{x4}, \code{y3}, and \code{y4} are handled similar, but for the
  top plane of the graph. The \code{z} axis is placed at the origin of
  the \code{x} and \code{y} dimension, whereas \code{z2} is placed at
  the final point of the \code{x} dimension, \code{z3} at the final
  point of the \code{y} dimension and \code{z4} at the final point of
  the \code{x} and \code{y} dimension together.
\end{classdesc}

\begin{memberdesc}{central}
  The central attribute of the graphxyz is the \class{central} class.
  See the class description below.
\end{memberdesc}

\begin{memberdesc}{parallel}
  The parallel attribute of the graphxyz is the \class{parallel} class.
  See the class description below.
\end{memberdesc}

Regarding the 3d to 2d transformation the methods \method{pos},
\method{vpos}, \method{vgeodesic}, and \method{vgeodesic\_el} are
available as for class \class{graphxy} and just take an additional
argument for the dimension. Note that a similar transformation method
(3d to 2d) is available as part of the projector as well already, but
only the graph acknowledges its size, the scaling and the internal
tranformation of the graph coordinates to the scaled coordinates. As
the projector also implements a \method{zindex} and a \method{angle}
method, those are also available at the graph level in the graph
coordinate variant (i.e. having an additional v in its name and using
values from 0 to 1 per dimension).

\begin{methoddesc}{vzindex}{vx, vy, vz}
  The depths of the point defined by \var{vx}, \var{vy}, and \var{vz}
  scaled to a range [-1:1] where 1 in closed to the viewer. All
  arguments passed to the method are in graph coordinates with range
  [0:1].
\end{methoddesc}

\begin{methoddesc}{vangle}{vx1, vy1, vz1, vx2, vy2, vz2, vx3, vy3, vz3}
  The cosine of the angle of the view ray thru point \code{(vx1, vy1,
  vz1)} and the plane defined by the points \code{(vx1, vy1, vz1)},
  \code{(vx2, vy2, vz2)}, and \code{(vx3, vy3, vz3)}. All arguments
  passed to the method are in graph coordinates with range [0:1].
\end{methoddesc}

There are two projector classes \class{central} and \class{parallel}:

\begin{classdesc}{central}{distance, phi, theta, anglefactor=math.pi/180}
  Instances of this class implement a central projection for the given
  parameters.

  \var{distance} is the distance of the viewer from the origin. Note
  that the \class{graphxyz} class uses the range \code{-xscale} to
  \code{xscale}, \code{-yscale} to \code{yscale}, and \code{-zscale}
  to \code{zscale} for the coordinates \code{x}, \code{y}, and
  \code{z}. As those scales are of the order of one (by default), the
  distance should be of the order of 10 to give nice results. Smaller
  distances increase the central projection character while for huge
  distances the central projection becomes identical to the parallel
  projection.

  \code{phi} is the angle of the viewer in the x-y-plane and
  \code{theta} is the angle of the viewer to the x-y-plane. The
  standard notation for spheric coordinates are used. The angles are
  multiplied by \var{anglefactor} which is initialized to do a degree
  in radiant transformation such that you can specify \code{phi} and
  \code{theta} in degree while the internal computation is always done
  in radiants.
\end{classdesc}

\begin{classdesc}{parallel}{phi, theta, anglefactor=math.pi/180}
  Instances of this class implement a parallel projection for the
  given parameters. There is no distance for that transformation
  (compared to the central projection). All other parameters are
  identical to the \class{central} class.
\end{classdesc} % }}}

\section{Module \module{graph.data}: Data} % {{{
\label{graph:data}

\declaremodule{}{graph.data}
\modulesynopsis{Graph data}

The following classes provide data for the \method{plot()} method of a
graph. The classes are implemented in \module{graph.data}.

\begin{classdesc}{file}{filename, % {{{
                        commentpattern=defaultcommentpattern,
                        columnpattern=defaultcolumnpattern,
                        stringpattern=defaultstringpattern,
                        skiphead=0, skiptail=0, every=1, title=notitle,
                        context=\{\}, copy=1,
                        replacedollar=1, columncallback="\_\_column\_\_",
                        **columns}
  This class reads data from a file and makes them available to the
  graph system. \var{filename} is the name of the file to be read.
  The data should be organized in columns.

  The arguments \var{commentpattern}, \var{columnpattern}, and
  \var{stringpattern} are responsible for identifying the data in each
  line of the file. Lines matching \var{commentpattern} are ignored
  except for the column name search of the last non-empty comment line
  before the data. By default a line starting with one of the
  characters \character{\#}, \character{\%}, or \character{!} as well
  as an empty line is treated as a comment.

  A non-comment line is analysed by repeatedly matching
  \var{stringpattern} and, whenever the stringpattern does not match,
  by \var{columnpattern}. When the \var{stringpattern} matches, the
  result is taken as the value for the next column without further
  transformations. When \var{columnpattern} matches, it is tried to
  convert the result to a float. When this fails the result is taken
  as a string as well. By default, you can write strings with spaces
  surrounded by \character{\textquotedbl} immediately surrounded by
  spaces or begin/end of line in the data file. Otherwise
  \character{\textquotedbl} is not taken to be special.

  \var{skiphead} and \var{skiptail} are numbers of data lines to be
  ignored at the beginning and end of the file while \var{every}
  selects only every \var{every} line from the data.

  \var{title} is the title of the data to be used in the graph key. A
  default title is constructed out of \var{filename} and
  \var{**columns}. You may set \var{title} to \code{None} to disable
  the title.

  Finally, \var{columns} define columns out of the existing columns
  from the file by a column number or a mathematical expression (see
  below). When \var{copy} is set the names of the columns in the file
  (file column names) and the freshly created columns having the names
  of the dictionary key (data column names) are passed as data to the
  graph styles. The data columns may hide file columns when names are
  equal. For unset \var{copy} the file columns are not available to
  the graph styles.

  File column names occur when the data file contains a comment line
  immediately in front of the data (except for empty or empty comment
  lines). This line will be parsed skipping the matched comment
  identifier as if the line would be regular data, but it will not be
  converted to floats even if it would be possible to convert the
  items. The result is taken as file column names, \emph{i.e.} a
  string representation for the columns in the file.

  The values of \var{**columns} can refer to column numbers in the
  file starting at \code{1}. The column \code{0} is also available
  and contains the line number starting from \code{1} not counting
  comment lines, but lines skipped by \var{skiphead}, \var{skiptail},
  and \var{every}. Furthermore values of \var{**columns} can be
  strings: file column names or complex mathematical expressions. To
  refer to columns within mathematical expressions you can also use
  file column names when they are valid variable identifiers. Equal
  named items in context will then be hidden. Alternatively columns
  can be access by the syntax \code{\$\textless number\textgreater}
  when \var{replacedollar} is set. They will be translated into
  function calls to \var{columncallback}, which is a function to
  access column data by index or name.

  \var{context} allows for accessing external variables and functions
  when evaluating mathematical expressions for columns. Additionally
  to the identifiers in \var{context}, the file column names, the
  \var{columncallback} function and the functions shown in the table
  ``builtins in math expressions'' at the end of the section are
  available.

  Example:
  \begin{verbatim}
graph.data.file("test.dat", a=1, b="B", c="2*B+$3")
  \end{verbatim}
  with \file{test.dat} looking like:
  \begin{verbatim}
# A   B C
1.234 1 2
5.678 3 4
  \end{verbatim}
  The columns with name \code{"a"}, \code{"b"}, \code{"c"} will become
  \code{"[1.234, 5.678]"}, \code{"[1.0, 3.0]"}, and \code{"[4.0,
  10.0]"}, respectively. The columns \code{"A"}, \code{"B"},
  \code{"C"} will be available as well, since \var{copy} is enabled by
  default.

  When creating several data instances accessing the same file,
  the file is read only once. There is an inherent caching of the
  file contents.
\end{classdesc}

For the sake of completeness we list the default patterns:

\begin{memberdesc}{defaultcommentpattern}
  \code{re.compile(r\textquotedbl (\#+|!+|\%+)\e s*\textquotedbl)}
\end{memberdesc}

\begin{memberdesc}{defaultcolumnpattern}
  \code{re.compile(r\textquotedbl\e \textquotedbl(.*?)\e \textquotedbl(\e s+|\$)\textquotedbl)}
\end{memberdesc}

\begin{memberdesc}{defaultstringpattern}
  \code{re.compile(r\textquotedbl(.*?)(\e s+|\$)\textquotedbl)}
\end{memberdesc} % }}}

\begin{classdesc}{function}{expression, title=notitle, % {{{
                            min=None, max=None, points=100,
                            context=\{\}}
  This class creates graph data from a function. \var{expression} is
  the mathematical expression of the function. It must also contain
  the result variable name including the variable the function depends
  on by assignment. A typical example looks like \code{"y(x)=sin(x)"}.

  \var{title} is the title of the data to be used in the graph key. By
  default \var{expression} is used. You may set \var{title} to
  \code{None} to disable the title.

  \var{min} and \var{max} give the range of the variable. If not set,
  the range spans the whole axis range. The axis range might be set
  explicitly or implicitly by ranges of other data. \var{points} is
  the number of points for which the function is calculated. The
  points are choosen linearly in terms of graph coordinates.

  \var{context} allows for accessing external variables and functions.
  Additionally to the identifiers in \var{context}, the variable name
  and the functions shown in the table ``builtins in math
  expressions'' at the end of the section are available.
\end{classdesc} % }}}

\begin{classdesc}{paramfunction}{varname, min, max, expression, % {{{
                                 title=notitle, points=100,
                                 context=\{\}}
  This class creates graph data from a parametric function.
  \var{varname} is the parameter of the function. \var{min} and
  \var{max} give the range for that variable. \var{points} is the
  number of points for which the function is calculated. The points
  are choosen lineary in terms of the parameter.

  \var{expression} is the mathematical expression for the parametric
  function. It contains an assignment of a tuple of functions to a
  tuple of variables. A typical example looks like
  \code{"x, y = cos(k), sin(k)"}.

  \var{title} is the title of the data to be used in the graph key. By
  default \var{expression} is used. You may set \var{title} to
  \code{None} to disable the title.

  \var{context} allows for accessing external variables and functions.
  Additionally to the identifiers in \var{context}, \var{varname} and
  the functions shown in the table ``builtins in math expressions'' at
  the end of the section are available.
\end{classdesc} % }}}

\begin{classdesc}{values}{title="user provided values", % {{{
                          **columns}
  This class creates graph data from externally provided data.
  Each column is a list of values to be used for that column.

  \var{title} is the title of the data to be used in the graph key.
\end{classdesc} % }}}

\begin{classdesc}{points}{data, title="user provided points", % {{{
                          addlinenumbers=1, **columns}
  This class creates graph data from externally provided data.
  \var{data} is a list of lines, where each line is a list of data
  values for the columns.

  \var{title} is the title of the data to be used in the graph key.

  The keywords of \var{**columns} become the data column names. The
  values are the column numbers starting from one, when
  \var{addlinenumbers} is turned on (the zeroth column is added to
  contain a line number in that case), while the column numbers starts
  from zero, when \var{addlinenumbers} is switched off.
\end{classdesc} % }}}

\begin{classdesc}{data}{data, title=notitle, context={}, copy=1, % {{{
                        replacedollar=1, columncallback="\_\_column\_\_", **columns}
  This class provides graph data out of other graph data. \var{data}
  is the source of the data. All other parameters work like the equally
  called parameters in \class{graph.data.file}. Indeed, the latter is
  built on top of this class by reading the file and caching its
  contents in a \class{graph.data.list} instance.
\end{classdesc} % }}}

\begin{classdesc}{conffile}{filename, title=notitle, context={}, copy=1, % {{{
                            replacedollar=1, columncallback="\_\_column\_\_", **columns}
  This class reads data from a config file with the file name
  \var{filename}. The format of a config file is described within the
  documentation of the \module{ConfigParser} module of the Python
  Standard Library.

  Each section of the config file becomes a data line. The options in
  a section are the columns. The name of the options will be used as
  file column names. All other parameters work as in
  \var{graph.data.file} and \var{graph.data.data} since they all use
  the same code.
\end{classdesc} % }}}

\begin{classdesc}{cbdfile}{filename, minrank=None, maxrank=None, % {{{
                           title=notitle, context={}, copy=1,
                           replacedollar=1, columncallback="\_\_column\_\_", **columns}
  This is an experimental class to read map data from cbd-files. See
  \url{http://sepwww.stanford.edu/ftp/World_Map/} for some world-map
  data.
\end{classdesc} % }}}

The builtins in math expressions are listed in the following table:
\begin{tableii}{l|l}{textrm}{name}{value}
\code{neg}&\code{lambda x: -x}\\
\code{abs}&\code{lambda x: x < 0 and -x or x}\\
\code{sgn}&\code{lambda x: x < 0 and -1 or 1}\\
\code{sqrt}&\code{math.sqrt}\\
\code{exp}&\code{math.exp}\\
\code{log}&\code{math.log}\\
\code{sin}&\code{math.sin}\\
\code{cos}&\code{math.cos}\\
\code{tan}&\code{math.tan}\\
\code{asin}&\code{math.asin}\\
\code{acos}&\code{math.acos}\\
\code{atan}&\code{math.atan}\\
\code{sind}&\code{lambda x: math.sin(math.pi/180*x)}\\
\code{cosd}&\code{lambda x: math.cos(math.pi/180*x)}\\
\code{tand}&\code{lambda x: math.tan(math.pi/180*x)}\\
\code{asind}&\code{lambda x: 180/math.pi*math.asin(x)}\\
\code{acosd}&\code{lambda x: 180/math.pi*math.acos(x)}\\
\code{atand}&\code{lambda x: 180/math.pi*math.atan(x)}\\
\code{norm}&\code{lambda x, y: math.hypot(x, y)}\\
\code{splitatvalue}&see the \code{splitatvalue} description below\\
\code{pi}&\code{math.pi}\\
\code{e}&\code{math.e}
\end{tableii}
\code{math} refers to Pythons \module{math} module. The
\code{splitatvalue} function is defined as:

\begin{funcdesc}{splitatvalue}{value, *splitpoints}
  This method returns a tuple \code{(section, \var{value})}.
  The section is calculated by comparing \var{value} with the values
  of {splitpoints}. If \var{splitpoints} contains only a single item,
  \code{section} is \code{0} when value is lower or equal this item
  and \code{1} else. For multiple splitpoints, \code{section} is
  \code{0} when its lower or equal the first item, \code{None} when
  its bigger than the first item but lower or equal the second item,
  \code{1} when its even bigger the second item, but lower or equal
  the third item. It continues to alter between \code{None} and
  \code{2}, \code{3}, etc.
\end{funcdesc}

% }}}

\section{Module \module{graph.style}: Styles} % {{{
\label{graph:style}

\declaremodule{}{graph.style}
\modulesynopsis{Graph style}

Please note that we are talking about graph styles here. Those are
responsible for plotting symbols, lines, bars and whatever else into a
graph. Do not mix it up with path styles like the line width, the line
style (solid, dashed, dotted \emph{etc.}) and others.

The following classes provide styles to be used at the \method{plot()}
method of a graph. The plot method accepts a list of styles. By that
you can combine several styles at the very same time.

Some of the styles below are hidden styles. Those do not create any
output, but they perform internal data handling and thus help on
modularization of the styles. Usually, a visible style will depend on
data provided by one or more hidden styles but most of the time it is
not necessary to specify the hidden styles manually. The hidden styles
register themself to be the default for providing certain internal
data.

\begin{classdesc}{pos}{epsilon=1e-10} % {{{
  This class is a hidden style providing a position in the graph. It
  needs a data column for each graph dimension. For that the column
  names need to be equal to an axis name. Data points are considered
  to be out of graph when their position in graph coordinates exceeds
  the range [0:1] by more than \var{epsilon}.
\end{classdesc} % }}}

\begin{classdesc}{range}{usenames={}, epsilon=1e-10} % {{{
  This class is a hidden style providing an errorbar range. It needs
  data column names constructed out of a axis name \code{X} for each
  dimension errorbar data should be provided as follows:
  \begin{tableii}{l|l}{}{data name}{description}
    \lineii{\code{Xmin}}{minimal value}
    \lineii{\code{Xmax}}{maximal value}
    \lineii{\code{dX}}{minimal and maximal delta}
    \lineii{\code{dXmin}}{minimal delta}
    \lineii{\code{dXmax}}{maximal delta}
  \end{tableii}
  When delta data are provided the style will also read column data
  for the axis name \code{X} itself. \var{usenames} allows to insert a
  translation dictionary from axis names to the identifiers \code{X}.

  \var{epsilon} is a comparison precision when checking for invalid
  errorbar ranges.
\end{classdesc} % }}}

\begin{classdesc}{symbol}{symbol=changecross, size=0.2*unit.v\_cm, % {{{
                          symbolattrs=[]}
  This class is a style for plotting symbols in a graph.
  \var{symbol} refers to a (changeable) symbol function with the
  prototype \code{symbol(c, x\_pt, y\_pt, size\_pt, attrs)} and draws
  the symbol into the canvas \code{c} at the position \code{(x\_pt,
  y\_pt)} with size \code{size\_pt} and attributes \code{attrs}. Some
  predefined symbols are available in member variables listed below.
  The symbol is drawn at size \var{size} using \var{symbolattrs}.
  \var{symbolattrs} is merged with \code{defaultsymbolattrs} which is
  a list containing the decorator \class{deco.stroked}. An instance of
  \class{symbol} is the default style for all graph data classes
  described in section~\ref{graph:data} except for \class{function}
  and \class{paramfunction}.
\end{classdesc}

The class \class{symbol} provides some symbol functions as member
variables, namely:

\begin{memberdesc}{cross}
  A cross. Should be used for stroking only.
\end{memberdesc}

\begin{memberdesc}{plus}
  A plus. Should be used for stroking only.
\end{memberdesc}

\begin{memberdesc}{square}
  A square. Might be stroked or filled or both.
\end{memberdesc}

\begin{memberdesc}{triangle}
  A triangle. Might be stroked or filled or both.
\end{memberdesc}

\begin{memberdesc}{circle}
  A circle. Might be stroked or filled or both.
\end{memberdesc}

\begin{memberdesc}{diamond}
  A diamond. Might be stroked or filled or both.
\end{memberdesc}

\class{symbol} provides some changeable symbol functions as member
variables, namely:

\begin{memberdesc}{changecross}
  attr.changelist([cross, plus, square, triangle, circle, diamond])
\end{memberdesc}

\begin{memberdesc}{changeplus}
  attr.changelist([plus, square, triangle, circle, diamond, cross])
\end{memberdesc}

\begin{memberdesc}{changesquare}
  attr.changelist([square, triangle, circle, diamond, cross, plus])
\end{memberdesc}

\begin{memberdesc}{changetriangle}
  attr.changelist([triangle, circle, diamond, cross, plus, square])
\end{memberdesc}

\begin{memberdesc}{changecircle}
  attr.changelist([circle, diamond, cross, plus, square, triangle])
\end{memberdesc}

\begin{memberdesc}{changediamond}
  attr.changelist([diamond, cross, plus, square, triangle, circle])
\end{memberdesc}

\begin{memberdesc}{changesquaretwice}
  attr.changelist([square, square, triangle, triangle, circle, circle, diamond, diamond])
\end{memberdesc}

\begin{memberdesc}{changetriangletwice}
  attr.changelist([triangle, triangle, circle, circle, diamond, diamond, square, square])
\end{memberdesc}

\begin{memberdesc}{changecircletwice}
  attr.changelist([circle, circle, diamond, diamond, square, square, triangle, triangle])
\end{memberdesc}

\begin{memberdesc}{changediamondtwice}
  attr.changelist([diamond, diamond, square, square, triangle, triangle, circle, circle])
\end{memberdesc}

The class \class{symbol} provides two changeable decorators for
alternated filling and stroking. Those are especially useful in
combination with the \method{change}-\method{twice}-symbol methods
above. They are:

\begin{memberdesc}{changestrokedfilled}
  attr.changelist([deco.stroked, deco.filled])
\end{memberdesc}

\begin{memberdesc}{changefilledstroked}
  attr.changelist([deco.filled, deco.stroked])
\end{memberdesc} % }}}

\begin{classdesc}{line}{lineattrs=[]} % {{{
  This class is a style to stroke lines in a graph.
  \var{lineattrs} is merged with \code{defaultlineattrs} which is
  a list containing the member variable \code{changelinestyle} as
  described below. An instance of \class{line} is the default style
  of the graph data classes \class{function} and \class{paramfunction}
  described in section~\ref{graph:data}.
\end{classdesc}

The class \class{line} provides a changeable line style. Its
definition is:

\begin{memberdesc}{changelinestyle}
  attr.changelist([style.linestyle.solid, style.linestyle.dashed, style.linestyle.dotted, style.linestyle.dashdotted])
\end{memberdesc} % }}}

\begin{classdesc}{impulses}{lineattrs=[], fromvalue=0, % {{{
                             frompathattrs=[], valueaxisindex=1}
  This class is a style to plot impulses. \var{lineattrs} is merged
  with \code{defaultlineattrs} which is a list containing the member
  variable \code{changelinestyle} of the \class{line} class.
  \var{fromvalue} is the baseline value of the impulses. When set to
  \code{None}, the impulses will start at the baseline. When fromvalue
  is set, \var{frompathattrs} are the stroke attributes used to show
  the impulses baseline path.
\end{classdesc} % }}}

\begin{classdesc}{errorbar}{size=0.1*unit.v\_cm, errorbarattrs=[], % {{{
                            epsilon=1e-10}
  This class is a style to stroke errorbars in a graph. \var{size} is
  the size of the caps of the errorbars and \var{errorbarattrs} are
  the stroke attributes. Errorbars and error caps are considered to be
  out of the graph when their position in graph coordinates exceeds
  the range [0:1] by more that \var{epsilon}. Out of graph caps are
  omitted and the errorbars are cut to the valid graph range.
\end{classdesc} % }}}

\begin{classdesc}{text}{textname="text", dxname=None, dyname=None, % {{{
                        dxunit=0.3*unit.v\_cm, dyunit=0.3*unit.v\_cm,
                        textdx=0*unit.v\_cm, textdy=0.3*unit.v\_cm,
                        textattrs=[]}
  This class is a style to stroke text in a graph. The
  text to be written has to be provided in the data column named
  \code{textname}. \var{textdx} and \var{textdy} are the position of the
  text with respect to the position in the graph. Alternatively you can
  specify a \code{dxname} and a \code{dyname} and provide appropriate
  data in those columns to be taken in units of \var{dxunit} and
  \var{dyunit} to specify the position of the text for each point
  separately. \var{textattrs} are text attributes for the output of
  the text. Those attributes are merged with the default attributes
  \code{textmodule.halign.center} and \code{textmodule.vshift.mathaxis}.
\end{classdesc} % }}}

\begin{classdesc}{arrow}{linelength=0.25*unit.v\_cm, % {{{
                         arrowsize=0.15*unit.v\_cm,
                         lineattrs=[], arrowattrs=[], arrowpos=0.5,
                         epsilon=1e-10, decorator=deco.earrow}
  This class is a style to plot short lines with arrows into a
  two-dimensional graph to a given graph position. The arrow
  parameters are defined by two additional data columns named
  \code{size} and \code{angle} define the size and angle for each
  arrow. \code{size} is taken as a factor to \var{arrowsize} and
  \var{linelength}, the size of the arrow and the length of the line
  the arrow is plotted at. \code{angle} is the angle the arrow points
  to with respect to a horizontal line. The \code{angle} is taken in
  degrees and used in mathematically positive sense. \var{lineattrs}
  and \var{arrowattrs} are styles for the arrow line and arrow head,
  respectively. \var{arrowpos} defines the position of the arrow line
  with respect to the position at the graph. The default \code{0.5}
  means centered at the graph position, whereas \code{0} and \code{1}
  creates the arrows to start or end at the graph position,
  respectively. \var{epsilon} is used as a cutoff for short arrows in
  order to prevent numerical instabilities. \var{decorator} defines
  the decorator to be added to the line.
\end{classdesc} % }}}

\begin{classdesc}{rect}{gradient=color.gradient.Grey} % {{{
  This class is a style to plot colored rectangles into a
  two-dimensional graph. The size of the rectangles is taken from
  the data provided by the \class{range} style. The additional
  data column named \code{color} specifies the color of the rectangle
  defined by \var{gradient}. The valid color range is [0:1].

  \begin{note}
    Although this style can be used for plotting colored surfaces, it
    will lead to a huge memory footprint of \PyX{} together with a
    long running time and large outputs. Improved support for colored
    surfaces is planned for the future.
  \end{note}
\end{classdesc} % }}}

\begin{classdesc}{histogram}{lineattrs=[], steps=0, fromvalue=0, % {{{
                             frompathattrs=[], fillable=0, rectkey=0,
                             autohistogramaxisindex=0,
                             autohistogrampointpos=0.5, epsilon=1e-10}
  This class is a style to plot histograms. \var{lineattrs} is merged
  with \code{defaultlineattrs} which is \code{[deco.stroked]}. When
  \var{steps} is set, the histrogram is plotted as steps instead of
  the default being a boxed histogram. \var{fromvalue} is the baseline
  value of the histogram. When set to \code{None}, the histogram will
  start at the baseline. When fromvalue is set, \var{frompathattrs}
  are the stroke attributes used to show the histogram baseline path.

  The \var{fillable} flag changes the stoke line of the histogram to
  make it fillable properly. This is important on non-steped
  histograms or on histograms, which hit the graph boundary.
  \var{rectkey} can be set to generate a rectanglar area instead of a
  line in the graph key.

  In the most general case, a histogram is defined by a range
  specification (like for an errorbar) in one graph dimension (say,
  along the x-axis) and a value for the other graph dimension. This
  allows for the widths of the histogram boxes being variable. Often,
  however, all histogram bin ranges are equally sized, and instead of
  passing the range, the position of the bin along the x-axis fully
  specifies the histogram - assuming that there are at least two bins.
  This common case is supported via two parameters:
  \var{autohistogramaxisindex}, which defines the index of the
  independent histogram axis (in the case just described this would be
  \code{0} designating the x axis). \var{autohistogrampointpos},
  defines the relative position of the center of the histogram bin:
  \code{0.5} means that the bin is centered at the values passed to
  the style, \code{0} (\code{1}) means that the bin is aligned at the
  right-(left-)hand side.

  XXX describe, how to specify general histograms with varying bin widths

  Positions of the histograms are considered to be out of graph when
  they exceed the graph coordinate range [0:1] by more than
  \var{epsilon}.
\end{classdesc} % }}}

\begin{classdesc}{barpos}{fromvalue=None, frompathattrs=[], epsilon=1e-10} % {{{
  This class is a hidden style providing position information in a bar
  graph. Those graphs need to contain a specialized axis, namely a bar
  axis. The data column for this bar axis is named \code{Xname} where
  \code{X} is an axis name. In the other graph dimension the data
  column name must be equal to an axis name. To plot several bars in a
  single graph side by side, you need to have a nested bar axis and
  provide a tuple as data for nested bar axis.

  The bars start at \var{fromvalue} when provided. The \var{fromvalue}
  is marked by a gridline stroked using \var{frompathattrs}. Thus this
  hidden style might actually create some output. The value of a bar
  axis is considered to be out of graph when its position in graph
  coordinates exceeds the range [0:1] by more than \var{epsilon}.
\end{classdesc} % }}}

\begin{classdesc}{stackedbarpos}{stackname, addontop=0, epsilon=1e-10} % {{{
  This class is a hidden style providing position information in a bar
  graph by stacking a new bar on top of another bar. The value of the
  new bar is taken from the data column named \var{stackname}. When
  \var{addontop} is set, the values is taken relative to the previous
  top of the bar.
\end{classdesc} % }}}

\begin{classdesc}{bar}{barattrs=[], epsilon=1e-10, gradient=color.gradient.RedBlack} % {{{
  This class draws bars in a bar graph. The bars are filled using
  \var{barattrs}. \var{barattrs} is merged with \code{defaultbarattrs}
  which is a list containing \code{[color.gradient.Rainbow,
  deco.stroked([color.grey.black])]}.

  The bar style has limited support for 3d graphs: Occlusion does not
  work properly on stacked bars or multiple dataset. \var{epsilon} is
  used in 3d to prevent numerical instabilities on bars without hight.
  When \var{gradient} is not \code{None} it is used to calculate a
  lighting coloring taking into account the angle between the view ray
  and the bar and the distance between viewer and bar. The precise
  conversion is defined in the \method{lighting} method.
\end{classdesc} % }}}

\begin{classdesc}{changebar}{barattrs=[]} % {{{
  This style works like the \class{bar} style, but instead of the
  \var{barattrs} to be changed on subsequent data instances the
  \var{barattrs} are changed for each value within a single data
  instance. In the result the style can't be applied to several data
  instances and does not support 3d. The style raises an error instead.
\end{classdesc} % }}}

\begin{classdesc}{gridpos}{index1=0, index2=1, % {{{
                        gridlines1=1, gridlines2=1, gridattrs=[],
                        epsilon=1e-10}
  This class is a hidden style providing rectangular grid information
  out of graph positions for graph dimensions \var{index1} and
  \var{index2}. Data points are considered to be out of graph when
  their position in graph coordinates exceeds the range [0:1] by more
  than \var{epsilon}. Data points are merged to a single graph
  coordinate value when their difference in graph coordinates is below
  \var{epsilon}.
\end{classdesc} % }}}

\begin{classdesc}{grid}{gridlines1=1, gridlines2=1, gridattrs=[]} % {{{
  Strokes a rectangular grid in the first grid direction, when
  \var{gridlines1} is set and in the second grid direction, when
  \var{gridlines2} is set. \var{gridattrs} is merged with
  \code{defaultgridattrs} which is a list containing the member
  variable \code{changelinestyle} of the \class{line} class.
\end{classdesc} % }}}

\begin{classdesc}{surface}{colorname="color", % {{{
                           gradient=color.gradient.Grey,
                           mincolor=None, maxcolor=None,
                           gridlines1=0.05, gridlines2=0.05,
                           gridcolor=None,
                           backcolor=color.gray.black}
  Draws a surface of a rectangular grid. Each rectangle is divided
  into 4 triangles.

  The grid can be colored using values provided by the data column
  named \var{colorname}. The values are rescaled to the range [0:1]
  using mincolor and maxcolor (which are taken from the minimal and
  maximal values, but larger bounds could be set).

  If no \var{colorname} column exists, the surface style falls back
  to a lighting coloring taking into account the angle between the
  view ray and the triangle and the distance between viewer and
  triangle. The precise conversion is defined in the
  \method{lighting} method.

  If a \var{gridcolor} is set, the rectangular grid is marked by small
  stripes of the relative (compared to each rectangle) size of
  \var{gridlines1} and \var{gridlines2} for the first and second grid
  direction, respectively.

  \var{backcolor} is used to fill triangles shown from the back. If
  \var{backcolor} is set to \code{None}, back sides are not drawn
  differently from the front sides.

  The surface is encoded using a single mesh. While this is quite
  space efficient, it has the following implications:
  \begin{itemize}
    \item All colors must use the same color space.
    \item HSB colors are not allowed, whereas Gray, RGB, and CMYK are
    allowed. You can convert HSB colors into a different color space
    before passing them to the surface.
    \item The grid itself is also constructed out of triangles. The
    grid is transformed along with the triangles thus looking quite
    different from a stroked grid (as done by the grid style).
    \item Occlusion is handled by proper painting order.
    \item Color changes are continuous (in the selected color
    space) for each triangle.
  \end{itemize}
\end{classdesc} % }}}

% }}}

\section{Module \module{graph.key}: Keys} % {{{
\label{graph:key}

\declaremodule{}{graph.key}
\modulesynopsis{Graph keys}

The following class provides a key, whose instances can be passed to
the constructor keyword argument \code{key} of a graph. The class is
implemented in \module{graph.key}.

\begin{classdesc}{key}{dist=0.2*unit.v\_cm,
                       pos="tr", hpos=None, vpos=None,
                       hinside=1, vinside=1,
                       hdist=0.6*unit.v\_cm,
                       vdist=0.4*unit.v\_cm,
                       symbolwidth=0.5*unit.v\_cm,
                       symbolheight=0.25*unit.v\_cm,
                       symbolspace=0.2*unit.v\_cm,
                       textattrs=[],
                       columns=1, columndist=0.5*unit.v\_cm,
                       border=0.3*unit.v\_cm, keyattrs=None}
  This class writes the title of the data in a plot together with a
  small illustration of the style. The style is responsible for its
  illustration.

  \var{dist} is a visual length and a distance between the key
  entries. \var{pos} is the position of the key with respect to the
  graph. Allowed values are combinations of \code{"t"} (top),
  \code{"m"} (middle) and \code{"b"} (bottom) with \code{"l"} (left),
  \code{"c"} (center) and \code{"r"} (right). Alternatively, you may
  use \var{hpos} and \var{vpos} to specify the relative position
  using the range [0:1]. \var{hdist} and \var{vdist} are the distances
  from the specified corner of the graph. \var{hinside} and
  \var{vinside} are numbers to be set to 0 or 1 to define whether the
  key should be placed horizontally and vertically inside of the graph
  or not.

  \var{symbolwidth} and \var{symbolheight} are passed to the style to
  control the size of the style illustration. \var{symbolspace} is the
  space between the illustration and the text. \var{textattrs} are
  attributes for the text creation. They are merged with
  \code{[text.vshift.mathaxis]}.

  \var{columns} is a number of columns of the graph key and
  \var{columndist} is the distance between those columns.

  When \var{keyattrs} is set to contain some draw attributes, the
  graph key is enlarged by \var{border} and the key area is drawn
  using \var{keyattrs}.
\end{classdesc} % }}} % }}}

% vim:fdm=marker

The quadratic formula is $$-b \pm \sqrt{b^2 - 4ac} \over 2a$$
\bye

\makeatletter
 \newcommand{\be}{%
 \begingroup
 % \setlength{\arraycolsep}{2pt}
 \eqnarray%
 \@ifstar{\nonumber}{}%
  }
  \newcommand{\ee}{\endeqnarray\endgroup}
  \makeatother

 \begin{equation}
 x=\left\{ \begin{array}{cl}
 0 & \textrm{if }A=\ldots\\
 1 & \textrm{if }B=\ldots\\
 x & \textrm{this runs with as much text as you like, but without an raggeright text
.}\end{array}\right.
 \end{equation}
\appendix
\chapter{Mathematical expressions}
\label{mathtree}

At several points within \PyX{} mathematical expressions can be
provided in form of string parameters. They are evaluated by the
module \verb|mathtree|. This module is not described further in this
user manual, because it is considered to be a technical detail. We
just give a list of available operators, functions and predefined
variable names here.

\begin{description}
\item[Operators:]
\verb|+|; \verb|-|; \verb|*|; \verb|/|; \verb|**|
\item[Functions:]
\verb|neg| (negate); \verb|abs| (absolute value); \verb|sgn| (signum);
\verb|sqrt| (square root); \verb|exp|; \verb|log| (natural logarithm);
\verb|sin|, \verb|cos|, \verb|tan|, \verb|asin|, \verb|acos|,
\verb|atan| (trigonometric functions in radian units); \verb|sind|,
\verb|cosd|, \verb|tand|, \verb|asind|, \verb|acosd|, \verb|atand| (as
before but in degree units); \verb|norm| ($\sqrt{a^2+b^2}$ as an
example for functions with multiple arguments)
\item[predefined variables:]
\verb|pi| ($\pi$); \verb|e| ($e$)
\end{description}

\chapter{Named colors}
\label{colorname}
\centerline{\includegraphics{colorname}}

\chapter{Named palettes}
\label{palettename}
\centerline{\includegraphics{palettename}}

\chapter{Module \module{style}}
\label{pathstyles}
\centerline{\includegraphics{pathstyles}}

\end{document}


\section{Tutorial}
\label{sec:tutor}


This short tutorial presents the most common possible uses of the
BNfinder software. The first part of this tutorial is devoted to
presenting possible options of the software and the input files on
simplistic, synthetic examples. In the second part, we provide more
realistic examples taken from published studies of data for inferring
dynamic and static networks.

In this tutorial, we will assume that you are using the standalone
BNfinder application as downloaded from
\url{http://bioputer.mimuw.edu.pl/software/bnf}, however if you want,
you can also try out these examples with our webserver at
\url{http://bioputer.mimuw.edu.pl/BIAS/BNFinder}.

If you have any questions regarding this document or the described
software, please contact us:
 \url{bartek@mimuw.edu.pl} or \url{dojer@mimuw.edu.pl}
\subsection{Synthetic examples}
\label{sec:simple}
This section shows on several simple networks, how to prepare datasets
and set the parameters for network reconstruction with BNfinder. The
examples include a simple static network, dynamic network and a
network requiring setting prior probabilities.

\subsubsection{Simple static network}
\label{sec:simstat}

\begin{figure}[h]
  \centering
  \includegraphics[width=8cm]{img/network1}  
  \caption{Very simple network consisting of 2 regulators and 4 observable regulatees.}
  \label{fig:net1}
\end{figure}

The first example shows how to use BNfinder to learn a simple static
Bayesian network. Let us imagine that we are analysisng cells under
two conditions $P1$ and $P2$ and that we are interested whether any of
the four genes: $G1,G2,G3,G4$ are responding to these conditions. We
assume that the true network is depicted in Fig \ref{fig:net1}, i.e. 
\begin{itemize}
\item $G1$ is not dependent on $P1$ or $P2$,
\item $G2$  is more likely to be expressed under condition $P1$,
\item $G3$ is less  likely to be expressed under condition $P2$,
\item $G4$ is more likely to  be expressed under any of the conditions $P1$ or $P2$.
\end{itemize}

We have collected 100 datapoints from this network, each consisting of
both the state of conditions and the discrete state of expression of
the genes. You can download the input file here \url{data/input1.txt}.

If you open the file in a text editor, please note that the first line
contains the information on the assumed structure:
\begin{verbatim}
#regulators P1 P2
\end{verbatim}
This represents the fact, that we assume that genes ($G1..G4$) can
depend on conditions ($P1,P2$) and not the other way around.

You can try to run BNfinder on this file:
\begin{verbatim}
bnf -e input1.txt -n output1.sif -v
\end{verbatim}
and you will see, that the network topology is reconstructed
properly. Also the orientation of the regulatory interactions is
inferred properly as you can see in the output file
\url{data/output1.sif}.

You can also try to see whether the optimal network is representative
for a larger set of possible suboptimal networks:
\begin{verbatim}
bnf -e input1.txt -n output1w.sif -v -i 4 -t output1.txt
\end{verbatim}
This time, in the output file (\url{data/output1w.sif}), the edge
labels represent the relative weights of different edges. In the file
\url{data/output1.txt}, we can find the originally computed weights
(i.e. relative probabilities -- see manual for details)
for all considered suboptimal sets of parents for all genes.

Instead of fixing the number of returned parents sets 
(option \texttt{-i 4}) you can specify thresholds for their weights
and/or weight ratios to optimal weights.
For example, if you wish to get for each vertex $v$
all parents sets with weights $>\max(0.1,0.01\cdot w_{opt}(v))$, 
where $w_{opt}(v)$ denotes the weight of the optimal parents set of $v$, 
you can type:
\begin{verbatim}
bnf -e input1.txt -n output1a.sif -v -i -1 -m 0.1 -o 0.01 -t output1a.txt
\end{verbatim}

We can also try to analyze the data for this network without
discretization: \url{data/input2.txt}. In this case we need another
directive to indicate that some of the dataseries are continuous:
\begin{verbatim}
#continuous G1 G2 G3 G4
\end{verbatim}

Again if we run BNfinder on this data, 
\begin{verbatim}
bnf -e input2.txt -n output2.sif -v
\end{verbatim}
we can verify, that the output file contains correct information \url{data/output2.sif}.

\subsubsection{Simple dynamic network}
\label{sec:simdyn}

BNfinder can be used also to infer dynamic Bayesian networks from time
series data. In this case it is not necessary to specify the
regulators sets, because DBNs, unlike static networks do not need to
be acyclic.

In the first dataset: \url{data/input3.txt}, we have 1 serie of 20
consecutive measurements of gene expression from gene network depicted
in Fig. \ref{fig:net2}.

\begin{figure}[h]
  \centering
  \includegraphics[width=8cm]{img/network2}  
  \caption{Very simple dynamic network consisting of 5 observables}
  \label{fig:net2}
\end{figure}

If we run BNfinder on this data:
\begin{verbatim}
bnf -e input3.txt -n output3.sif -v 
\end{verbatim}
We can see that the program was unable to correctly reconstruct all
the edges. Again, if we look at the statistics of edge occurences in
suboptimal networks,
\begin{verbatim}
bnf -e input3.txt -n output3.sif -v -i 10 -t output3.txt
\end{verbatim}
we can see that the correct edges are the most commonly occuring ones,
but they score lower than empty parent sets.

In this case we can show how perturbational data can be integrated
into this framework. We have collected gene expression from 5
time-series containing one single gene knockout for each of the genes:
\url{data/input4.txt}. The perturbations are noted by including the
following lines in the preamble of the data file:
\begin{verbatim}
#perturbed EXP1 G1
#perturbed EXP2 G2
#perturbed EXP3 G3
#perturbed EXP4 G4
#perturbed EXP5 G0
\end{verbatim}

If we run BNfinder on the perturbed data, we can see that all the edges
are reconstructed with high confidence.
\begin{verbatim}
bnf -e input4.txt -n output4.sif -v -i 10 -t output4.txt
\end{verbatim}

\subsubsection{Setting priors}
\label{sec:simprio}
\begin{figure}[h]
  \centering
  \includegraphics[width=6cm]{img/network3}  
  \caption{Exemplary network containing dependencies of different strength}
  \label{fig:net3}
\end{figure}
In some cases it might be useful to include some prior information on
the network structure into the process of inference. We will
illustrate this on an example of a simple network similar to the one
described in section \ref{sec:simstat}. This time it is an even
simpler network, with 2 conditions and 2 genes as depicted in
Fig. \ref{fig:net3}. Even though topology of the network is very
simple, the problem lies in the fact that the dependence of $G2$ on
$P2$ is weaker than the dependence of $G1$ on $P1$. This is why if we
run our software on the unmodified dataset \url{data/input5.txt}, 
\begin{verbatim}
bnf -e input5.txt -n output5.sif -v  -i 10 -t output5.txt
\end{verbatim}
we can see that the program is unable to recover the $P2\rightarrow G2$
edge. However, if we expect that $G2$ is  responding weakly to its regulators, 
we can increase the prior probability of $G2$ being regulated by any of 
the factors $P1,P2$ via decreasing its weight:
\begin{verbatim}
#prioredge G2 0.33 P2 P1
\end{verbatim}
We can see the edge appearing in the result as expected (see \url{data/input6a.txt}):
\begin{verbatim}
bnf -e input6a.txt -n output6a.sif -v  -i 10 -t output6a.txt
\end{verbatim}

Similarly, if we expect, that in general gene response to condition $P2$ is weaker, we may modify the prior probability of the condition $P2$ to be a regulator:
\begin{verbatim}
#priorvert 0.33 P2 
\end{verbatim}

The result of running BNFinder with this input (\url{data/input6b.txt}) is very similar to the previous one:
\begin{verbatim}
bnf -e input6b.txt -n output6b.sif -v  -i 10 -t output6b.txt
\end{verbatim}

\subsection{Examples of published  datasets}
\label{sec:real}

In this section, we present two more realistic examples of published
datasets used for inference of Bayesian networks. The first one
consists of measurements of states of protein signalling network under
different perturbations \cite{pmid15845847}. It's been used to infer
causal relationships in the form of static Bayesian network.

The second dataset comes from documentation of the Banjo package
\cite{Smith2006} which can be downloaded from
(\url{http://www.cs.duke.edu/$\sim$amink/software/banjo}). It consists
of 2000 observations describing a relatively large dynamic network
consisting of 20 nodes. It may be considered a benchmark of the
efficiency of our algorithm.

The third dataset is converted from an example attached to the
globalMIT software for Bayesian network reconstruction. It is similar
to the second example as it is also generated from a dynamic Bayesian
network and consists of 2000 observations of 20 variables. However, in
this case the variables are much less interconnected and there are
many self-regulatory loops.

\textbf{PLEASE NOTE that these datasets are too large to be run through BNFinder webserver. If you would like to run them, please download the software. }

\subsubsection{Static Protein signalling network}
\label{sec:realStat}

In this section we present how BNfinder can be applied to a protein
signalling network analyzed by Sachs et al \cite{pmid15845847}. We
took the data from the article, and transformed it into the format
suitable for BNfinder. We also needed to specify several properties of
the data in the preamble of the file \url{data/sachs.inp}

\begin{figure}[h]
  \centering
  \includegraphics[width=8cm]{img/static}  
  \caption{Reconstruction of the protein signalling network. Dark blue
    arrows represent dependencies found in literature. Light blue
    arrows represent dependencies found by BNfinder but not expected
    by Sachs et al. \cite{pmid15845847}}
  \label{fig:stat}
\end{figure}

Firstly, we needed to specify that the data are continuous measurements:

\begin{verbatim}
#continuous praf pmek plcg PIP2 PIP3 p44/42 pakts473 PKA PKC P38 pjnk
\end{verbatim}

Then, we needed to specify the expected layer structure of the signalling pathway we are studying:
\begin{verbatim}
#regulators plcg
#regulators PIP3
#regulators PIP2
#regulators PKC
#regulators PKA
#regulators praf
#regulators pjnk pmek P38 
#regulators p44/42
#regulators pakts473
\end{verbatim}

Then we needed to specify which of the proteins are affected by different perturbations. 
\begin{verbatim}
#perturbed cd3cd28psitect_0 PIP2
#perturbed cd3cd28psitect_1 PIP2
#perturbed cd3cd28psitect_2 PIP2
...
#perturbed cd3cd28g0076_0 PKC
#perturbed cd3cd28g0076_1 PKC
#perturbed cd3cd28g0076_2 PKC
...
\end{verbatim}

When we finally run the BNfinder:
\begin{verbatim}
bnf -e sachs.inp -n sachs.sif -v 
\end{verbatim}
We obtain the network presented in Fig. \ref{fig:stat}. As we can see,
the topology is quite consistent with the literature data. Out of 17
expected edges, BNfinder recovers 11 correctly. 

\subsubsection{Dynamic Bayesian network}
\label{sec:realDyn}

This is a dataset of substantial size which is used  \cite{bnfinder} to
assess the performance of our inference algorithm. The input dataset
(\url{data/input7.txt}{}) consists of 2000 measurements of 20 variables
and it takes approximately 3 hours to compute it on a modern PC (2.4Ghz
Intel Core 2 duo). 

We can run BNfinder with the following command (note that we are
using the \texttt{-l} option to limit the number of parents to $5$):
\begin{verbatim}
bnf -e input7.txt -n output7.sif -v -l 5 -txt output7.txt
\end{verbatim}

In Fig. \ref{fig:dyn} we can see part of the network reconstructed by
BNfinder. All the edges reported by Banjo are also present in the
optimal network (dark blue). The optimal network contains a number of
additional edges, not reported by Banjo.

\begin{figure}[h]
  \centering
   \includegraphics[width=5cm]{img/dynamic}  
  \caption{The optimal network reconstructed by BNfinder from the dynamic benchmark
    dataset. The edges reported also by Banjo are shown in dark blue. }
  \label{fig:dyn}
\end{figure}

If you want to see how much faster the MDL algorithm is, you can also run BNfinder 
with the following command:
\begin{verbatim}
bnf -s MDL -e input7.txt -n output7mdl.sif -v -l 5 -txt output7mdl.txt
\end{verbatim}

\subsubsection{Dynamic network containing self-regulatory loops}
\label{sec:self-loops}

In this example, we can utilize both the \texttt{-g 1} option for
allowing self-regulations as well as the \texttt{-s MIT} option for
using the MIT score.

\begin{verbatim}
bnf -s MIT -e input8.txt -n output8.sif -v -l 3 -txt output8.txt -c output8.cpd -g 1
\end{verbatim}

One additional parameter that is unique to the MIT score is the
significance level $\alpha$ of the $\chi^2$ distribution
(\texttt{-a}). The default level for $\alpha$ is $.9999$, but we can
increase/decrease it if we want to see fewer/more edges respectively.

For example, setting the level alpha to a higher value should give us more edges in the result:

\begin{verbatim}
bnf -s MIT -e input8.txt -n output8.sif -v -l 3 -a .9  -g 1
\end{verbatim}

\subsubsection{Using multiple processors for faster computations}
\label{sec:multicore}

Since most current computers are equipped with multiple processors, we
can take advantage of that fact to speed up BNFinder
computation. Especially for large datasets, such as the ones described
in previous sections, we can take full advantage of the parallell
computation. For example, if we want BNfinder to run on 4 CPUs in
parallell, we can use the \texttt{-k 4} option as in the following example:

\begin{verbatim}
bnf -s MIT -e input8.txt -n output8.sif -v -l 3 -a .9  -g 1 -k 4
\end{verbatim}


\subsection{Example of classification with BNfinder}

\texttt{bnf-cv} and \texttt{bnc} tools can be used to solve classification tasks with classifier based on Bayesian networks. The former is used to perform a cross-validation test and the later to classify a dataset when you already have a classifier. In our example we will try to solve the following problem: we have points within the unit square; our positive set consists of those that are located in top-right and bottom-left corners, i.e. x + y > 1.8 or x + y < 0.2. The training set consists of 100 positive and 100 negative examples. They are visualised in the following Fig. \ref{fig:training}. The data can be downloaded from here (\url{data/training\_set.txt}). We marked x and y as continuous regulators. We will classify only one feature, but it is possible to perform cross-validation and classification procedure for more variables. All variables not marked as regulators are treated as variables to be explained by classifier.

\begin{figure}[h]
  \centering
   \includegraphics[width=10cm]{img/training}
  \caption{The training set used in the classification example. Positive examples are colored blue. }
  \label{fig:training}
\end{figure}

To perform a 10-fold cross-validation we can use the following command:
\begin{verbatim}
bnf-cv -e training_set.txt -c net.cpd -k 10 -r ROC.pdf
\end{verbatim}

As a result we obtain 10 files (net.cpd0, net.cpd1, ..., net.cpd9) containing networks in cpd format corresponding to respective folds of the cross-validation. Every execution of foregoing command will bring different results, because a split into 10 sets is done randomly. In the result file ROC.pdf there is a ROC plot showing the results of cross-validation (see \ref{fig:ROC}). Further results are printed to the standard output and contains (among others) information about regulators taken to each of 10 classifiers and AUC measure of each classifier's performance. 

\begin{figure}[h!]
  \centering
   \includegraphics[width=10cm]{img/ROC}
  \caption{The Receiver operating characteristics curve for 10-fold cross-validation. The thick curve shows the average performance of classifiers. }
  \label{fig:ROC}
\end{figure}

To perform classification task on a test dataset we can use any of the nets obtained from cross-validation task but usually it is better to train a classifier on the whole training dataset. It can be done by the following command:
\begin{verbatim}
bnf -e training_set.txt -c net.cpd
\end{verbatim}

We will test out classifier on this (\url{data/test\_set.txt}) dataset which consists of 1000 points from the unit square. Now, by using the \texttt{bnc} tool we can obtain the classification. In the Fig. \ref{fig:classificationresult} we can see the result of classifying our test dataset by classifier in the file net.cpd (we used 0.63 probability threshold to generate the plot):
\begin{verbatim}
bnc -o result.cls -p 1 -c net.cpd -d test_set.txt
\end{verbatim}

\begin{figure}[h!]
  \centering
   \includegraphics[width=10cm]{img/classificationresult}
  \caption{The result of classification. Blue and red points represent true positives and negatives. There was no false negatives. False positives are colored light red.}
  \label{fig:classificationresult}
\end{figure}

We can also find the most probable class for \texttt{corners} for every experiment by executing:
\begin{verbatim}
bnc -o result.cls -m 1 -c net.cpd -d test_set.txt
\end{verbatim}

Finally, by executing
\begin{verbatim}
bnf-cv -e training_set.txt -k 1 -r ROC.pdf
\end{verbatim}
command one can generate a colored plot of classifier trained on the whole training set. Different colors (explained on the right side of the picture) indicate probability thresholds above which we classify example as a positive one. An example of such a plot can be seen on the Fig. \ref{fig:rock1}.

\begin{figure}[h!]
  \centering
   \includegraphics[width=10cm]{img/rock1}
  \caption{The effect of ploting ROC curve from 1-fold cross-validation.}
  \label{fig:rock1}
\end{figure}

%%% Local Variables: 
%%% mode: latex
%%% TeX-master: "tut"
%%% End: 


\section{Supplementary methods}
 In the present section we give a brief exposition of the algorithm implemented in BNFinder and its computational cost for two generally used scoring criteria: Minimal Description Length and Bayesian-Dirichlet equivalence.
 For a fuller treatment, including detailed proofs, we refer the reader to
%   \cite{dojer06}.
     \cite{dojer06,dojer10}.

\subsection{Polynomial-time exact algorithm}

 A \emph{Bayesian network} (BN) $\N$ is a representation of a joint distribution of a set of discrete random variables $\X=\{X_1,\ldots,X_n\}$.
 The representation consists of two components:
 \begin{itemize}
  \item a directed acyclic graph $\G=(\X,\E)$ encoding conditional (in-)dependencies
  \item a family $\theta$ of conditional distributions $P(X_i|\Pa_i)$, where
   $$\Pa_i=\{Y\in\X|(Y,X_i)\in\E\}$$
 \end{itemize}
 The joint distribution of $\X$ is given by
 \begin{equation}\label{joinprob}
 P(\X)=\prod_{i=1}^n P(X_i|\Pa_i)
 \end{equation}

 The problem of learning a BN is understood as follows: 
 given a multiset of $\X$-instances $\D=\{\x_1,\ldots,\x_N\}$ find a network graph $\G$ that best matches $\D$.
 The notion of a good match is formalized by means of a \emph{scoring function} $S(\G:\D)$ having positive values and minimized for the best matching network. 
 Thus the point is to find a directed acyclic graph $\G$ with the set of vertices $\X$ minimizing $S(\G:\D)$.
 
 The BNFinder program is devoted to the case when there is no need to examine the acyclicity of the graph, for example:
 \begin{itemize}
  \item When dealing with \emph{dynamic} Bayesian networks. A dynamic BN  describes stochastic evolution of a set of random variables over discretized time. Therefore conditional distributions refer to random variables in neighboring time points. The acyclicity constraint is relaxed, because the ''unrolled'' graph (with a copy of each variable in each time point) is always acyclic (see \cite{friedman98} for more details). The following considerations apply to dynamic BNs as well. 
  \item In case of static Bayesian Networks, the user has to supply the
   algorithm with a partial ordering of the vertices, restricting the
   set of possible edges only to the ones consistent with the
   ordering. BNFinder lets the user to divide the set of variables into an
   ordered set of disjoint subsets of variables, where edges can only
   exist between variables from different subsets and they have to be
   consistent with the ordering. If such ordering is not known
   beforehand, one can try to run BNFinder with different orderings and
   choose a network with the best overall score.
 \end{itemize}
 
 %The detailed description of the most generally used MDL and BDe scores will be provided in further sections.
 In the sequel we consider some assumptions on the form of a scoring function.
 The first one states that $S(\G:\D)$ decomposes into a sum over the set of random variables of \emph{local scores}, depending on the values of a variable and its parents in the graph only.
 
\begin{ass}[additivity]\label{as1}
 $S(\G:\D) = \sum_{i=1}^n s(X_i,\Pa_i:\D|_{\{X_i\}\cup\Pa_i})$, where $\D|_{\Y}$ denotes the restriction of $\D$ to the values of the members of $\Y\subseteq\X$.
\end{ass}

 When there is no need to examine the acyclicity of the graph, this assumption allows to compute the parents set of each variable independently.
 Thus the point is to find $\Pa_i$ minimizing $s(X_i,\Pa_i:\D|_{\{X_i\}\cup\Pa_i})$ for each $i$.

 Let us fix a dataset $\D$ and a random variable $X$. 
 We denote by $\X'$ the set of potential parents of $X$ (possibly smaller than $\X$ due to given constraints on the structure of the network).
 To simplify the notation we continue to write $s(\Pa)$ for $s(X,\Pa:\D|_{\{X\}\cup\Pa})$.
 
 The following assumption expresses the fact that scoring functions decompose into 2 components: $g$ penalizing the complexity of a network and $d$ evaluating the possibility of explaining data by a network.

\begin{ass}[splitting]\label{as2}
 $s(\Pa)=g(\Pa)+d(\Pa)$ for some functions $g,d:\mathcal{P}(\X)\to\mathbb{R}^+$ satisfying $\Pa\subseteq\Pa'\Longrightarrow g(\Pa)\le g(\Pa')$.
\end{ass}

 This assumption is used in the following algorithm to avoid considering networks with inadequately large component $g$.

\begin{alg}\label{a1}~
\begin{center}
\begin{tabular}{|p{0.9\textwidth}|}\hline
\begin{enumerate}
\item $\Pa:=\emptyset$
\item for each $\PP\subseteq\X'$ chosen according to $g(\PP)$
\begin{enumerate}
\item if $s(\PP)<s(\Pa)$ then $\Pa:=\PP$
\item if $g(\PP)\ge s(\Pa)$ then return $\Pa$; stop
\end{enumerate}
\end{enumerate}\\
\hline
\end{tabular}
\end{center}
\end{alg}

 In the above algorithm \emph{choosing according to $g(\PP)$} means choosing increasingly with respect to the value of the component $g$ of the local score.
 
 \begin{theorem}
 Suppose that the scoring function satisfies Assumptions \ref{as1}-\ref{as2}. Then Algorithm \ref{a1} applied to each random variable finds an optimal network.
 \end{theorem}
 
 A disadvantage of the above algorithm is that finding a proper subset $\PP\subseteq\X'$ involves computing $g(\PP')$ for all $\subseteq$-successors $\PP'$ of previously chosen subsets. 
 It may be avoided when a further assumption is imposed.

\begin{ass}[uniformity]\label{as3}
 $|\Pa|=|\Pa'|\Longrightarrow g(\Pa)=g(\Pa')$.
\end{ass}

 The above assumption suggests the notation $\g(|\Pa|)=g(\Pa)$. 
 The following algorithm uses the uniformity of $g$ to reduce the number of computations of the component $g$.

\begin{alg}\label{a2}~
\begin{center}
\begin{tabular}{|p{0.9\textwidth}|}\hline
\begin{enumerate}
\item $\Pa:=\emptyset$
\item for $p=1$ to $n$%$|\X|$
\begin{enumerate}
\item if $\g(p)\ge s(\Pa)$ then return $\Pa$; stop
\item $\PP=arg min_{\{\Y\subseteq\X' : |\Y|=p\}}s(\Y)$
\item if $s(\PP)<s(\Pa)$ then $\Pa:=\PP$
\end{enumerate}
\end{enumerate}\\
\hline
\end{tabular}
\end{center}
\end{alg}

 \begin{theorem}
 Suppose that the scoring function satisfies Assumptions \ref{as1}-\ref{as3}. Then Algorithm \ref{a2} applied to each random variable finds an optimal network.
 \end{theorem}
 
\subsection{Minimal Description Length and Bayesian Information Criterion}

 The Minimal Description Length (MDL) scoring criterion originates from information theory \cite{lam94}. 
 A network $\N$ is viewed here as a model of compression of a dataset $\D$. 
 The optimal model minimizes the total length of the description, i.e. the sum of the description length of the model and of the compressed data.

 Let us fix a dataset $\D=\{\x_1,\ldots,\x_N\}$ and a random variable $X$. 
 Recall the decomposition $s(\Pa)=g(\Pa)+d(\Pa)$ of the local score for $X$. 
 In the MDL score $g(\Pa)$ stands for the length of the description of the local part of the network (i.e. the edges ingoing to $X$ and the conditional distribution $P(X|\Pa)$) and $d(\Pa)$ is the length of the compressed version of $X$-values in $\D$.
 
 Let $k_Y$ denote the cardinality of the set $\V_Y$ of possible values of the random variable $Y\in\X$.
 Thus we have 
 $$g(\Pa)=|\Pa|\log n+\frac{\log N}{2}(k_X-1)\prod_{Y\in\Pa}k_Y$$
 where $\frac{\log N}{2}$ is the number of bits we use for each numeric parameter of the conditional distribution. 
 This formula satisfies Assumption \ref{as2} but fails to satisfy Assumption \ref{as3}.
 Therefore Algorithm \ref{a1} can be used to learn an optimal network, but Algorithm \ref{a2} cannot. 
 
 However, for many applications we may assume that 
 all random variables have the same value set $\V$ of cardinality $k$.
 In this case we obtain the formula
 $$g(\Pa)=|\Pa|\log n+\frac{\log N}{2}(k-1)k^{|\Pa|}$$
 which satisfies Assumption \ref{as3}.
 For simplicity, we continue to work under this assumption.
 
 Compression with respect to the network model is understood as follows: when encoding the $X$-values, the values of $\Pa$-instances are assumed to be known. 
 Thus the optimal encoding length is given by 
 $$d(\Pa)=N\cdot H(X|\Pa)$$
 where $H(X|\Pa)=-\sum_{v\in\V}\sum_{\vp\in\V^{\Pa}}P(v,\vp)\log P(v|\vp)$ is the conditional entropy of $X$ given $\Pa$ (the distributions are estimated from $\D$).
 
 Since all the assumptions from the previous section are satisfied, Algorithm \ref{a2} may be applied to learn the optimal network. 
 Let us turn to the analysis of its complexity.
 
 \begin{theorem}
 The worst-case time complexity of Algorithm \ref{a2} for the MDL score is $\OO(n^{\log_k N}N\log_k N)$.
 \end{theorem}
 
 MDL is almost identical to Bayesian Information Criterion (BIC) 
 (see \cite{neapolitan03}), 
 which approximates Bayesian scores (see next section).
 The only difference is that the first element of the sum in the formula 
 for the $g$ component is omitted. The above theorem applies to BIC as well.
 
\subsection{Bayesian-Dirichlet equivalence}

 The Bayesian-Dirichlet equivalence (BDe) scoring criterion originates from Bayesian statistics \cite{cooper92}. 
 Given a dataset $\D$ the optimal network structure $\G$ maximizes the \emph{posterior} conditional probability $P(\G|\D)$. 
 We have 
 $$P(\G|\D)\propto P(\G)P(\D|\G)=P(\G)\int P(\D|\G,\theta)P(\theta|\G)d\theta$$
 where $P(\G)$ and $P(\theta|\G)$ are \emph{prior} probability distributions on graph structures and conditional distributions' parameters, respectively, and $P(\D|\G,\theta)$ is evaluated due to \eqref{joinprob}.
 
 Heckerman et al. \cite{heckerman95}, following Cooper and Herskovits \cite{cooper92}, identified a set of independence assumptions making possible decomposition of the integral in the above formula into a product over $\X$.
 Under this condition, together with a similar one regarding decomposition of $P(\G)$, the scoring criterion
 $$S(\G:\D)=-\log P(\G)-\log P(\D|\G)$$
 obtained by taking $-\log$ of the above term satisfies Assumption \ref{as1}. 
 Moreover, the decomposition $s(\Pa)=g(\Pa)+d(\Pa)$ of the local scores appears as well, with the components $g$ and $d$ derived from $-\log P(\G)$ and $-\log P(\D|\G)$, respectively.

% The distribution $P((\X,\E))\propto\alpha^{|\E|}$ with a penalty parameter $0<\alpha<1$ in general is used as a prior over the network structures. 
% This choice results in the function 
% $$g(|\Pa|)=|\Pa|\log\alpha^{-1}$$ 
% satisfying Assumptions \ref{as2} and \ref{as3}.

 The distribution $P((\X,\E))\propto\prod_{e\in\E}\alpha_{e}$ 
 with penalty parameters $0<\alpha_e<1$ 
 is commonly used as a prior over the network structures. 
 BNFinder sets $\alpha_{(Y,X)}=1/k_Y$ by default.
 This choice results in the function 
 $$g(\Pa)=\sum_{Y\in\Pa}\log k_Y$$ 
 satisfying Assumptions \ref{as2}.
 If we moreover assume that all random variables 
 have the same value set $\V$ of cardinality $k$,
 we obtain the function
 $$g(\Pa)=|\Pa|\log k$$ 
 satisfying also Assumption \ref{as3}.
 For simplicity, we continue to work under this assumption.

 However, it should be noticed that there are also used priors 
 which satisfy neither Assumption \ref{as2} nor \ref{as3}, 
 e.g.  $P(\G)\propto\alpha^{\Delta(\G,\G_0)}$, 
 where $\Delta(\G,\G_0)$ is the cardinality of the symmetric difference 
 between the sets of edges in $\G$ and in the prior network $\G_0$.
 
 The \emph{Dirichlet distribution} is generally used as a prior over the conditional distributions' parameters.
 It yields
 $$d(\Pa)=\log\left(\prod_{\vp\in\V^{|\Pa|}}
  \frac{\Gamma(\sum_{v\in\V}(H_{v,\vp}+N_{v,\vp}))}{\Gamma(\sum_{v\in\V}H_{v,\vp})}
  \prod_{v\in\V}\frac{\Gamma(H_{v,\vp})}{\Gamma(H_{v,\vp}+N_{v,\vp})}\right)$$
 where $\Gamma$ is the \emph{Gamma} function, $N_{v,\vp}$ denotes the number of samples in $\D$ with $X=v$ and $\Pa=\vp$, and $H_{v,\vp}$ is the corresponding \emph{hyperparameter} of the Dirichlet distribution.
 
 Setting all the hyperparameters to $1$ yields
 \begin{multline*}d(\Pa)=\log\left(\prod_{\vp\in\V^{|\Pa|}}
  \frac{(k-1+\sum_{v\in\V}N_{v,\vp})!}{(k-1)!}
  \prod_{v\in\V}\frac{1}{N_{v,\vp}!}\right)=\\
  =\sum_{\vp\in\V^{|\Pa|}}\left(
 \log({\textstyle k-1+\underset{v\in\V}{\sum}N_{v,\vp}})!-\log(k-1)!
  -\sum_{v\in\V}\log{N_{v,\vp}!}\right)
 \end{multline*}
 where $k=|\V|$.
 For simplicity, we continue to work under this assumption (following Cooper and Herskovits \cite{cooper92}).
 The general case may be handled in a similar way.
 
 The following result allows to refine the decomposition of the local score into the sum of the components $g$ and $d$.
 
 \begin{proposition}
 Define $d_{min}=\sum_{v\in\V}\left(\log(k-1+N_v)!-\log(k-1)!-\log N_v!\right)$, where $N_v$ denotes the number of samples in $\D$ with $X=v$. 
 Then $d(\Pa)\ge d_{min}$ for each $\Pa\in\X$.
 \end{proposition}
 
 By the above proposition, the decomposition of the local score given by $s(\Pa)=g'(\Pa)+d'(\Pa)$ with the components $g'(\Pa)=g(\Pa)+d_{min}$ and $d'(\Pa)=d(\Pa)-d_{min}$ satisfies all the assumptions required by Algorithm \ref{a2}. 
 Let us turn to the analysis of its complexity.
  
 \begin{theorem}
 The worst-case time complexity of Algorithm \ref{a2} for the BDe score with the decomposition of the local score given by $s(\Pa)=g'(\Pa)+d'(\Pa)$ is $\OO(n^{N\log_{\alpha^{-1}}k}N^2\log_{\alpha^{-1}}k)$.
 \end{theorem}


\subsection{Mutual information test}

The Mutual Information Test (MIT) scoring criterion originates from the concept of mutual information, belonging to the family of measures based on information theory \cite{deCampos}. Briefly speaking, this method combines mutual information measure and a statistical independence test based on the chi-square dustribution assosiated with it. The goodness of a fit of the particular network is computed as the total mutual information between each node and its parents. This score is then penalized by a term corresponding to the degree of statistical significance of the shared information. 

Let $\mathcal{D}$ be a dataset with $N$ observations, $\mathcal{G}$ be the dynamic bayesian network. 
Let $X = \{X_1,...,X_n\}$ be the set of $n$ variables, with each of it corresponding to $ \{r_1,...,r_n$\} discrete states.
Let's denote the set of parents of $X_i$ in $\mathcal{G}$ with corresponding $\{r_{i1},...,r_{is_i}\}$ discrete states
as $\Pa_i = \{X_{i1},...,X_{is_i}\}$. 
Then the MIT score is defined as follows \cite{globMIT}: 
$$\mathcal{S(G:D)} = \sum_{i=0; \Pa_i \neq \emptyset}^n \{ 2N \cdot I(X_i, \Pa_i) - \sum_{j=1}^{s_i} \chi_{\alpha l_i \sigma_i(j)} \}$$

In this formula $I(X_i, \Pa_i)$ denotes the mutual information between $X_i$ and its parents
as estimated from $\mathcal{D}$ and defined as $$ I(X;Y) = \sum_{y \in Y} \sum_{x \in X} p(x,y) \log \left(\frac{p(x,y)}{p(x)p(y)}\right)  $$
$\chi_{\alpha l_i \sigma_i(j)} $ is the chi-square distribution at significance level $1-\alpha$.
It is defined as the value such that $$p(\chi^{2}(l_{ij}) \le \chi_{\alpha l_ij})=\alpha $$
\\The term $ l_{i \sigma_i (j)}$ denotes the degrees of freedom and is defined as 
$$ l_{i \sigma_i (j)} = \begin{cases}
   (r_i-1)(r_{i \sigma_i (j)} -1) \prod_{k=1}^{j-1} r_{i \sigma_i (k)}, & \text{$j=2..,s_i$}.\\
   (r_i-1)(r_{i \sigma_i (j)} -1), & \text{$j=1$}.
 \end{cases}
$$
where $\sigma_i = \{\sigma_i (1),...,\sigma_i(s_i)\}$ is any permutation of the index set
$\{1...s_i\}$ of $\Pa_i$ such that, the number of states of the variables
decreases with the increasing position in the permutation. 

Recall the decomposition $ S_{MIT}(\Pa_i) = d_{MIT}(\Pa_i) + g_{MIT}(\Pa_i) $. 
In this case:  
$$ d_{MIT}(\Pa_i) = 2N\cdot I(X_i, \mathbf{X}) - 2N \cdot I(X_i, \Pa_i) $$
$$ g_{MIT}(\Pa_i) = \sum_{j=1}^{s_i} \chi_{\alpha l_i \sigma_i(j)} $$ 
Roughly, $d_{MIT}$ measures the accuracy of representing the joint distribution of 
$\mathcal{D}$ by $\mathcal{G}$ while $g_{MIT}$ measures the complexity of this representation. This decomposition
satisfies Assumption 2.
However, MIT score defined in this way does not satisfy Assumption 3.
Therefore, we introduce an assumption that all the variables have the same number of discrete states.

\begin{ass}[uniformity]\label{as4}
 All variables in $X$ have the same number of discrete states k.
\end{ass}

Under this assumption it can be easily shown that $g_{MIT}$ satisfies Assumption 3.
 \begin{theorem} \cite{globMITManual}
 The worst-case time complexity of Algorithm \ref{a2} for the MIT score under the assumption of the variables uniformity is polynomial in the number of variables. 
 \end{theorem}


\subsection{Continuous variables}

All the scoring functions implemented in BNFinder (MDL, BIC, BDe and MIT) 
were originally designed for discrete variables.
In order to avoid arbitrary discretization of continuous data 
we adapted them to deal with continuous variables directly. 
Moreover, our method works also with heterogenous data sets 
joining together discrete and continuous variables.

%% The distribution of each continuous variable is assumed to be 
%% a mixture of two normal distributions.
%% Mixture parameters are estimated from data clustered with the \emph{k-means} algorithm 
%% ($k=2$, cutting the set of variable values in the median yields the initial clusters).
%% Then the parameters are used in transforming continuous values into 
%% probability distributions on the mixture components.
%% These components are considered to be the two possible values 
%% (\emph{low} and \emph{high}) 
%% of a related hidden discrete variable.

The distribution of each continuous variable $X$ is assumed to be 
a mixture of two normal distributions.
Mixture components correspond to the two possible values 
(\emph{low} and \emph{high}) of a related hidden discrete variable $X'$
and $X$ is viewed as its observable reflection.
%Therefore regulatory relationships are learned 
%for discrete variables rather than continuous ones.
Consequently, the conditional distributions of $X$ is given by:
%\begin{multline*}
$$
P(X|\Pa)=
\sum_{v\in\Vc}\sum_{\vp\in\Vc^{|\Pa|}}P(X|X'=v)P(X'=v|\Pa'=\vp)P(\Pa'=\vp|\Pa)
$$
%=\\
%=\sum_{v\in\Vc}
%\left(\sum_{\vp\in\Vc^{|\Pa|}}P(X'=v|\Pa'=\vp)P(\Pa'=\vp|\Pa)\right)
%P(X|X'=v)
%\end{multline*}
%Since distributions $P(X|X'=v)$ are Gaussian, 
%$P(X|\Pa)$ is a Gaussian mixture with parameters dependent on values of $\Pa$.

%In a preprocessing step parameters of $P(X|X')$ are estimated separately 
%for each variable $X$.
Conditional distributions $P(X|X')$ are assumed to be independent 
for all variables $X$.
Thus we estimate their parameters separately for each $X$ in a preprocessing step.
Estimation is based on data clustering with the \emph{k-means} algorithm 
($k=2$, cutting the set of variable values in the median yields initial clusters).
Due to the independence assumption, these parameters enable us to calculate also
$P(\Pa'|\Pa)=\prod_{Y\in\Pa}P(Y'|Y)$.
Thus the space of possible conditional distributions on continuous variables
forms a family of Gaussian mixtures, parameterized by $P(X'|\Pa')$, 
%They are the only free parameters in $P(X|\Pa)$ and at the same time the parameters of
conditional distributions on corresponding discrete variables.

From a technical point of view,
BNFinder learns optimal network structures for these discrete variables, 
using scoring functions adapted to handle 
distributions on variable values instead of their determined values
(expected values of original scores are computed).
For continuous variables it gives optimal Bayesian networks from among 
all networks with conditional probability distributions
belonging to the above defined family of Gaussian mixtures.


The following results present the complexity of our algorithm 
with continuous MDL and BDe scoring functions.

 \begin{theorem}
 The worst-case time complexity of Algorithm \ref{a2} for the continuous MDL score is $\OO(n^{\log N}N^2)$.
 \end{theorem}
 
 \begin{theorem}
 The worst-case time complexity of Algorithm \ref{a2} for the continuous BDe score with the decomposition of the local score given by $s(\Pa)=g'(\Pa)+d'(\Pa)$ is $\OO((2n)^\frac{N}{\log\alpha^{-1}}N)$.
 \end{theorem}
 
 \subsection{Network density control}

Recall that scoring functions decompose into 2 components: 
$g$ penalizing the complexity of a network and 
$d$ evaluating the possibility of explaining data by a network.
The balance between these components influences 
the reliability of reconstructed relationships between variables --
high $g$-to-$d$ ratio results in high specificity, 
while low $g$-to-$d$ ratio yields high sensitivity.

BNFinder has 3 mechanisms controlling this balance:
\begin{enumerate}
\item Option \verb$-d$ directly multiplies $g$-to-$d$ ratio by 
a uniform factor for all pairs of variables.
\item Options \verb$-r$ and \verb$-u$ set $g$-to-$d$ ratios for all edges
according to specified proportion of false positive edges 
or of regulons having false positive regulators
(thus controlling type I error rate). 
It is particularly useful for heterogeneous sets of potential parents
(continuous and discrete, discrete with varying levels of discretization),
when different types of variables require specific treatment.
\item Input dataset preamble commands \verb$#prioredge$ and \verb$#priorvert$
modify $g$-to-$d$ ratios for specified network edges.
%This mechanism is intended to modify $g$-to-$d$ ratios selectively,
%in order
They are intended to incorporate into the learning process 
prior knowledge regarding possible variable dependencies.
This method may be combined with one of previous mechanisms.
\end{enumerate}

Option \verb$-d$ modifies $g$-to-$d$ ratio by virtual dataset multiplication.
%Remaining two mechanisms redefine component $g$ of the scoring function.
%In the case of BDe score prior distributions over network structures 
%are chosen accordingly.
Remaining two mechanisms adjust components $g$ of the scoring function.
It is done through redefining the formula for $g$ 
by raising parameters $k_Y$,
the number of discretization levels of a potential parent $Y$ 
to appropriate powers $w_{Y,X}$
(in the case of BDe, it is just a modification of 
a prior distribution over network structures).
Exponents $w_{Y,X}$ are either adjusted to required type I error rate 
or specified in the preamble of a dataset.
They must satisfy $w_{Y,X}>0$, default values $w_{Y,X}=1$ result 
in the original formula for $g$ component.
%Since $g$-component formulas in MDL and MIT scores are unparameterized,
%we redefined them,
%introducing parameters in place of 
%the numbers of parent variables' discretization levels.
%Since MDL and MIT scores have no corresponding parameters,
%we redefined their $g$-component formulas,
%introducing such parameters in place of 
%the numbers of parent variables' discretization levels.

The control of type I error rate is based on a statistical model for 1-element
set of potential parents and extrapolated to all sets.
In the 1-element case there are only 2 potential parent sets:
$\emptyset$ and $\{Y\}$, where $Y$ is the only potential parent
of considered regulated variable $X$.
First, BNFinder calculates the required type I error probability for edge $(Y,X)$.
When no prior distribution on the network structure 
is specified in the dataset preamble,
all edge error probabilities equal the requested type I error rate.
Otherwise they are weighted according to the inverses of prior parameters.

Under a null hypothesis $H_0$ that variables $X$ and $Y$ are independent,
type I error occurs when $s(\{Y\})<s(\emptyset)$.
We define $Z_{Y,X}=d(\{Y\})-d(\emptyset)$ and $z_{Y,X}=g(\emptyset)-g(\{Y\})$.
Thus $s(\{Y\})<s(\emptyset)$ if and only if $Z_{Y,X}<z_{Y,X}$.
Note that $Z_{Y,X}$ is a function of dataset values of random variables $X$ and $Y$,
so it is a random variable too.
On the other hand, $z_{Y,X}$ is independent of the data
and monotonically depends on $w_{Y,X}$.

BNFinder randomly permutes values of $Y$ in the dataset
and calculates $Z_{Y,X}$ for each permutation.
The number of permutations is chosen according to 
requested type I error probability and the dataset size. 
Moreover, it may be manually shrunk to avoid exhaustive computations.
The estimate of
cumulative distribution function for $Z_{Y,X}$ under $H_0$ assumption
is derived from calculated values and $d_{min}-d(\emptyset)$,
the lower bound on $Z_{Y,X}$.
%These values and $d_{min}-d(\emptyset)$, the lower bound on $Z_{Y,X}$,
%are then used to derive the estimate of 
%cumulative distribution function for $Z_{Y,X}$ under $H_0$ assumption.
Based on this distribution BNFinder adjusts $w_{Y,X}$ to yield
$P(Z_{Y,X}<z_{Y,X} | H_0)$ %=r_{Y,X}$, 
%where $r_{Y,X}$ is 
equal to the required type I error probability for edge $(Y,X)$.


\bibliographystyle{plain}
%\bibliography{regul_new}
\bibliography{regul}



\end{document}
