\chapter{Basic graphics}

\sectionauthor{J\"org Lehmann}{joergl@users.sourceforge.net} 

\label{graphics}

\section{Introduction}

The path module allows one to construct PostScript-like
\textit{paths}, which are one of the main building blocks for the
generation of drawings. A PostScript path is an arbitrary shape
consisting of straight lines, arc segments and cubic B\'ezier curves.
Such a path does not have to be connected but may also comprise
several disconnected segments, which will be called \textit{subpaths}
in the following.

XXX example for paths and subpaths

Usually, a path is constructed by passing a list of the path
primitives \class{moveto}, \class{lineto}, \class{curveto}, etc., to the
constructor of the \class{path} class. The following code snippet, for
instance, defines a path \var{p} that consists of a straight line
from the point $(0, 0)$ to the point $(1, 1)$
\begin{verbatim}
from pyx import *
p = path.path(path.moveto(0, 0), path.lineto(1, 1))
\end{verbatim}
Equivalently, one can also use the predefined \class{path} subclass
\class{line} and write
\begin{verbatim}
p = path.line(0, 0, 1, 1)
\end{verbatim}

While already some geometrical operations can be performed with this
path (see next section), another \PyX{} object is needed in order to
actually being able to draw the path, namely an instance of the
\class{canvas} class. By convention, we use the name \var{c} for this
instance:
\begin{verbatim}
c = canvas.canvas()
\end{verbatim}
In order to draw the path on the canvas, we use the \method{stroke()} method
of the \class{canvas} class, i.e.,
\begin{verbatim}
c.stroke(p)
c.writeEPSfile("line")
\end{verbatim}
To complete the example, we have added a \method{writeEPSfile()} call,
which writes the contents of the canvas to the file \file{line.eps}.
Note that an extension \file{.eps} is added automatically, if not
already present in the given filename.

As a second example, let us define a path which consists of more than 
one subpath:
\begin{verbatim}
cross = path.path(path.moveto(0, 0), path.rlineto(1, 1),
                  path.moveto(1, 0), path.rlineto(-1, 1))
\end{verbatim}
The first subpath is again a straight line from $(0, 0)$ to $(1, 1)$,
with the only difference that we now have used the \class{rlineto}
class, whose arguments count relative from the last point in the path.
The second \class{moveto} instance opens a new subpath starting at the
point $(1, 0)$ and ending at $(0, 1)$. Note that although both lines
intersect at the point $(1/2, 1/2)$, they count as disconnected
subpaths.  The general rule is that each occurence of a \class{moveto}
instance opens a new subpath. This means that if one wants to draw a
rectangle, one should not use
\begin{verbatim}
rect1 = path.path(path.moveto(0, 0), path.lineto(0, 1),
                  path.moveto(0, 1), path.lineto(1, 1),
                  path.moveto(1, 1), path.lineto(1, 0),
                  path.moveto(1, 0), path.lineto(0, 0))
\end{verbatim}
which would construct a rectangle out of four disconnected
subpaths (see Fig.~\ref{fig:rects}a). In a better solution (see
Fig.~\ref{fig:rects}b), the pen is not lifted between the first and
the last point:
%
\begin{figure}
\centerline{\includegraphics{rects}}
\caption{Rectangle consisting of (a) four separate lines, (b) one open
  path, and (c) one closed path. (d) Filling a
  path always closes it automatically.}
\label{fig:rects}
\end{figure}
%
\begin{verbatim}
rect2 = path.path(path.moveto(0, 0), path.lineto(0, 1), 
                  path.lineto(1, 1), path.lineto(1, 0))
\end{verbatim}
However, as one can see in the lower left corner of
Fig.~\ref{fig:rects}b, the rectangle is still incomplete.  It needs to
be closed, which can  be done explicitly by using for the last straight
line of the rectangle (from the point $(0, 1)$ back to the origin at $(0, 0)$)
the \class{closepath} directive:
\begin{verbatim}
rect3 = path.path(path.moveto(0, 0), path.lineto(0, 1), 
                  path.lineto(1, 1), path.lineto(1, 0),
                  path.closepath())
\end{verbatim}
The \class{closepath} directive adds a straight line from the current
point to the first point of the current subpath and furthermore
\textit{closes} the sub path, i.e., it joins the beginning and the end
of the line segment. This results in the intended rectangle shown in
Fig.~\ref{fig:rects}c. Note that filling the path implicitly closes
every open subpath, as is shown for a single subpath in
Fig.~\ref{fig:rects}d), which results from
\begin{verbatim}
c.stroke(rect2, [deco.filled([color.grey(0.95)])])
\end{verbatim}
More details on the available path elements can be found in
Sect.~\ref{path:pathitem}.

XXX more on styles and attributes and reference to corresponding section

Of course, rectangles are also predefined in \PyX{}, so above we could
have as well written
\begin{verbatim}
rect2 = path.rect(0, 0, 1, 1)
\end{verbatim}
Here, the first two arguments specify the origin of the rectangle
while the second two arguments define its width and height,
respectively. For more details on the predefined paths, we
refer the reader to Sect.~\ref{path:predefined}.

\section{Path operations}

Often, one wants to perform geometrical operations with a path before
placing it on a canvas by stroking or filling it.  For instance, one
might want to intersect one path with another one, split the paths at
the intersection points, and then join the segments together in a new
way. \PyX{} supports such tasks by means of a number of path methods,
which we will introduce in the following.

Suppose you want to draw the radii to the intersection points of a
circle with a straight line. This task can be done using the following
code which results in Fig.~\ref{fig:radii}
\verbatiminput{radii.py}
\begin{figure}
\centerline{\includegraphics{radii}}
\caption{Example: Intersection of circle with line yielding two radii.}
\label{fig:radii}
\end{figure}
Here, the basic elements, a circle around the point $(0, 0)$ with
radius $2$ and a straight line, are defined. Then, passing the \var{line}, to
the \method{intersect()} method of \var{circle}, we obtain a tuple of
parameter values of the intersection points. The first element of the
tuple is a list of parameter values for the path whose
\method{intersect()} method has been called, the second element is the
corresponding list for the path passed as argument to this method. In
the present example, we only need one list of parameter values, namely
\var{isects_circle}.  Using the \method{at()} path method to obtain
the point corresponding to the parameter value, we draw the radii for
the different intersection points. 

Another powerful feature of \PyX{} is its ability to split paths at a
given set of parameters. For instance, in order to fill in the
previous example the segment of the circle delimited by the straight
line (cf.\ Fig.~\ref{fig:radii2}), one first has to construct a path
corresponding to the outline of this segment. The following code
snippet yields this \var{segment}
\begin{verbatim}
arc1, arc2 = circle.split(isects_circle)
if arc1.arclen() < arc2.arclen():
    arc = arc1
else:
    arc = arc2

isects_line.sort()
line1, line2, line3 = line.split(isects_line)

segment = line2 << arc
\end{verbatim}
\begin{figure}
\centerline{\includegraphics{radii2}}
\caption{Example: Intersection of circle with line yielding radii and
  circle segment.}
\label{fig:radii2}
\end{figure}
Here, we first split the circle using the \method{split()} method passing
the list of parameters obtained above. Since the circle is closed,
this yields two arc segments. We then use the \method{arclen()}, which
returns the arc length of the path, to find the shorter of the two
arcs. Before splitting the line, we have to take into account that
the \method{split()} method only accepts a sorted list of parameters.
Finally, we join the straight line and the arc segment. For
this, we make use of the \verb|<<| operator, which not only adds
the paths (which could be done using \samp{line2 + arc}), but also
joins the last subpath of \var{line2} and the first one of
\var{arc}. Thus, \var{segment} consists of only a single subpath
and filling works as expected.

An important issue when operating on paths is the parametrisation
used. Internally, \PyX{} uses a parametrisation which uses an interval
of length $1$ for each path element of a path. For instance, for a
simple straight line, the possible parameter values range from $0$ to
$1$, corresponding to the first and last point, respectively, of the
line. Appending another straight line, would extend this range to a
maximal value of $2$. You can always query this maximal value using
the \method{range()} method of the \class{path} class.  

However, the situation becomes more complicated if more complex
objects like a circle are involved. Then, one could be tempted to
assume that again the parameter value range from $0$ to $1$, because
the predefined circle consists just of one \class{arc} together with a
\class{closepath} element. However, as a simple \samp{path.circle(0,
  0, 1).range()} will tell, this is not the case: the actual range is
much larger. The reason for this behaviour lies in the internal path
handling of \PyX: Before performing any non-trivial geometrical
operation with a path, it will automatically be converted into an
instance of the \class{normpath} class (see also
Sect.~\ref{path:normpath}). These so generated paths are already
separated in their subpaths and only contain straight lines and
B\'ezier curve segments. Thus, as is easily imaginable, they are much
simpler to deal with.

A unique way of accessing a point on the path is to use the arc length
of the path segment from the first point of the path to the given
point. Thus, all \PyX{} path methods that accept a parameter value
also allow the user to pass an arc length. For instance, 
\begin{verbatim}
from math import pi

pt1 = path.circle(0, 0, 1).at(arclen=pi)
pt2 = path.circle(0, 0, 1).at(arclen=3*pi/2)

c.stroke(path.path(path.moveto(*pt1), path.lineto(*pt2)))
\end{verbatim}
will draw a straight line from a point at angle $180$ degrees (in
radians $\pi$) to another point at angle $270$ degrees (in radians
$3\pi/2$) on the unit circle.

More information on the available path methods can be found 
in Sect.~\ref{path:path}.

\section{Attributes: Styles and Decorations}

XXX to be done