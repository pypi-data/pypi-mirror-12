\section{Module \module{canvas}}
\label{canvas}

\sectionauthor{J\"org Lehmann}{joergl@users.sourceforge.net} 

One of the central modules for the PostScript access in \PyX{} is
named \verb|canvas|. Besides providing the class \verb|canvas|, which
presents a collection of visual elements like paths, other canvases,
\TeX{} or \LaTeX{} elements, it contains the class
\texttt{canvas.clip} which allows clipping of the output.

A canvas may also be embedded in another one using its \texttt{insert}
method. This may be useful when you want to apply a transformation on
a whole set of operations..

\declaremodule{}{canvas}

\subsection{Class \class{canvas}}

This is the basic class of the canvas module, which serves to collect
various graphical and text elements you want to write eventually to an
(E)PS file.

\begin{classdesc}{canvas}{attrs=[], texrunner=None}
  Construct a new canvas, applying the given \var{attrs}, which can be
  instances of \class{trafo.trafo}, \class{canvas.clip},
  \class{style.strokestyle} or \class{stlye.fillstyle}.  The
  \var{texrunner} argument can be used to specify the texrunner
  instance used for the \method{text()} method of the canvas.  If not
  specified, it defaults to \var{text.defaulttexrunner}.
\end{classdesc}


Paths can be drawn on the canvas using one of the following methods:

\begin{methoddesc}{draw}{path, attrs}
  Draws \var{path} on the canvas applying the given \var{attrs}.
\end{methoddesc}

\begin{methoddesc}{fill}{path, attrs=[]}
  Fills the given \var{path} on the canvas applying the given
  \var{attrs}. 
\end{methoddesc}

\begin{methoddesc}{stroke}{path, attrs=[]}
  Strokes the given \var{path} on the canvas applying the given
  \var{attrs}.
\end{methoddesc}

Arbitrary allowed elements like other \class{canvas} instances can
be inserted in the canvas using

\begin{methoddesc}{insert}{item, attrs=[]}
  Inserts an instance of \class{base.canvasitem} into the canvas.  If
  \var{attrs} are present, \var{item} is inserted into a new
  \class{canvas}instance with \var{attrs} as arguments passed to its
  constructor is created. Then this \class{canvas} instance is
  inserted itself into the canvas. Returns \var{item}.
\end{methoddesc}

Text output on the canvas is possible using

\begin{methoddesc}{text}{x, y, text, attrs=[]}
  Inserts \var{text} at position (\var{x}, \var{y}) into the
  canvas applying \var{attrs}. This is a shortcut for
  \texttt{insert(texrunner.text(x, y, text, attrs))}).
\end{methoddesc}

The \class{canvas} class provides access to the total geometrical size
of its element:

\begin{methoddesc}{bbox}{}
  Returns the bounding box enclosing all elements of the canvas.
\end{methoddesc}

A canvas also allows one to set global options:

\begin{methoddesc}{set}{styles}
  Sets the given \var{styles} (instances of \class{style.fillstyle} or
  \class{style.strokestyle} or subclasses thereof).  for the rest of
  the canvas.
\end{methoddesc}

\begin{methoddesc}{settexrunner}{texrunner}
  Sets a new \var{texrunner} for the canvas.
\end{methoddesc}

The contents of the canvas can be written using:

\begin{methoddesc}{writeEPSfile}{filename, paperformat=None,
    rotated=0, fittosize=0, margin=1*unit.t_cm, bbox=None,
    bboxenlarge=1*unit.t_pt}
  Writes the canvas to \var{filename} (the extension \texttt{.eps} is
  appended automatically). Optionally, a \var{paperformat} can be
  specified, in which case the output will be centered with respect to
  the corresponding size using the given \var{margin}. See
  \var{canvas.\_paperformats} for a list of known paper formats .  Use
  \var{rotated}, if you want to center on a $90^\circ$ rotated version
  of the respective paper format. If \var{fittosize} is set, the
  output is additionally scaled to the maximal possible size.
  Normally, the bounding box of the canvas is calculated automatically
  from the bounding box of its elements.  Alternatively, you may
  specify the \var{bbox} manually. In any case, the bounding box
  becomes enlarged on all side by \var{bboxenlarge}. This may be used
  to compensate for the inability of \PyX{} to take the linewidths
  into account for the calculation of the bounding box.
\end{methoddesc}

\subsection{Patterns}

The \class{pattern} class allows the definition of PostScript Tiling
patterns (cf.\ Sect.~4.9 of the PostScript Language Reference Manual)
which may then be used to fill paths. The classes \class{pattern} and
\class{canvas} differ only in their constructor and in the absence of
a \method{writeEPSfile()} method in the former. The \class{pattern}
constructor accepts the following keyword arguments:

\medskip
\begin{tabularx}{\linewidth}{l>{\raggedright\arraybackslash}X}
keyword&description\\
\hline
\texttt{painttype}&\texttt{1} (default) for coloured patterns or
\texttt{2} for uncoloured patterns\\
\texttt{tilingtype}&\texttt{1} (default) for constant spacing tilings
(patterns are spaced constantly by a multiple of a device pixel),
\texttt{2} for undistored pattern cell, whereby the spacing may vary
by as much as one device pixel, or \texttt{3} for constant spacing and
faster tiling which behaves as tiling type \texttt{1} but with
additional distortion allowed to permit a more efficient
implementation.\\
\texttt{xstep}&desired horizontal spacing between pattern cells, use
\texttt{None} (default) for automatic calculation from pattern
bounding box.\\
\texttt{ystep}&desired vertical spacing between pattern cells, use
\texttt{None} (default) for automatic calculation from pattern
bounding box.\\
\texttt{bbox}&bounding box of pattern. Use \texttt{None} for an
automatical determination of the bounding box (including an
enlargement by $5$ pts on each side.)\\
\texttt{trafo}&additional transformation applied to pattern or
\texttt{None} (default). This may be used to rotate the pattern or to
shift its phase (by a translation).
\end{tabularx}
\medskip

After you have created a pattern instance, you define the pattern
shape by drawing in it like in an ordinary canvas. To use the pattern,
you simply pass the pattern instance to a \method{stroke()},
\method{fill()}, \method{draw()} or \method{set()} method of the
canvas, just like you would do with a colour, etc.



%%% Local Variables:
%%% mode: latex
%%% TeX-master: "manual.tex"
%%% End:
