% \iffalse
%<package>\NeedsTeXFormat{LaTeX2e}
%<*driver>
\def\fileversion{0.3}
\def\filedate{2004/12/10}
\ProvidesFile{glifaq.drv}
%</driver>
%<package>\ProvidesPackage{glifaq}
           [2004/12/10 Style for PyX FAQ (gli) v0.3]
%<*driver>
\documentclass{ltxdoc}
\usepackage{url}
\EnableCrossrefs
\CodelineIndex
\RecordChanges
\newcommand\PyX{P\kern-.3em\lower.5ex\hbox{Y}\kern-.18em X}
\newcommand{\acro}[1]{\textsc{#1}}
\begin{document}
   \DocInput{glifaq.dtx}
\end{document}
%</driver>
% \fi
%
% \CharacterTable
%  {Upper-case    \A\B\C\D\E\F\G\H\I\J\K\L\M\N\O\P\Q\R\S\T\U\V\W\X\Y\Z
%   Lower-case    \a\b\c\d\e\f\g\h\i\j\k\l\m\n\o\p\q\r\s\t\u\v\w\x\y\z
%   Digits        \0\1\2\3\4\5\6\7\8\9
%   Exclamation   \!     Double quote  \"     Hash (number) \#
%   Dollar        \$     Percent       \%     Ampersand     \&
%   Acute accent  \'     Left paren    \(     Right paren   \)
%   Asterisk      \*     Plus          \+     Comma         \,
%   Minus         \-     Point         \.     Solidus       \/
%   Colon         \:     Semicolon     \;     Less than     \<
%   Equals        \=     Greater than  \>     Question mark \?
%   Commercial at \@     Left bracket  \[     Backslash     \\
%   Right bracket \]     Circumflex    \^     Underscore    \_
%   Grave accent  \`     Left brace    \{     Vertical bar  \|
%   Right brace   \}     Tilde         \~}
%
% \CheckSum{239}
%
%
% \title{The \texttt{glifaq} package}
%
% \author{Gert-Ludwig Ingold\\\small\url{<gert.ingold@physik.uni-augsburg.de>}}
% \date{(v.\fileversion\ -- \filedate)}
% \maketitle
%
% \section{Introduction}
% The \texttt{glifaq} package has been developed to typeset the \acro{faq} of 
% the \PyX{} graphics package for \acro{Python}. It might contain a few
% commands which are useful for other \acro{faq}s as well.
%
% \section{Usage}
% \label{sec:usage}
% The package \texttt{glifaq} is intended for use with the \texttt{scrartcl}
% class which is part of \acro{KOMA-Script}. \texttt{scrartcl} should therefore
% be chosen as document class. 
%  
% The options of the package \texttt{glifaq} mostly control the kind of 
% questions which will be typeset. There are four kinds of questions:
% \begin{center}
% \begin{tabular}{ll}
% |a| & answered question\\
% |c| & question where the answer has been checked by an expert\\
% |o| & outdated question\\
% |t| & question not fully answered yet (= to do)
% \end{tabular}
% \end{center}
% The following options are recognized by the package:
% \begin{description}
% \item[all] prints all questions
% \item[answered] prints only questions which have no ``to do'' status, i.e.
%            they have a satisfying answer and might even have been checked
% \item[checked] prints only questions where the answers have been checked
% \item[outdated] prints only questions and answers which are outdated
% \item[todo] prints all questions where no satisfying answer has been 
%            formulated yet
% \item[unchecked] prints all questions which have not been checked yet
% \item[comments] if comments exist, they will be printed as well and a 
%            special mark will be put in the margin
% \item[notoc] the table of contents will not be printed
% \end{description}
% By default, all questions and the table of contents but no comments will be
% printed. If questions with ``to do'' status are printed, a special mark will
% be put in the margin.
%
% \DescribeMacro\question
% The most important task of the package \texttt{glifaq} is to define a command
% \begin{quote}
% |\question{|\meta{status}|}{|\meta{title}|}{|\meta{comment}|}{|^^A
%    \meta{answer}|}|
% \end{quote}
% to handle the typesetting of questions and their answers. As indicated, this 
% command takes four arguments:
% \begin{center}
% \begin{tabular}{ll}
% 1: & class of the question according to the table given above\\
%    & allowed values: |a|, |c|, |o|, |t|\\
% 2: & title of the question\\
% 3: & comments related to this question\\
% 4: & the answer to the question
% \end{tabular}
% \end{center}
%
% \DescribeEnv{progcode}
% For the typesetting of code snippets, an environment |progcode| has been
% defined which will use a fixed space font. Unfortunately, verbatim code
% cannot be used in arguments as is the case here in macro |\question|.
% To guarantee proper spaces and in particular indenting, a tilde has to
% used instead of a space.
%
% In addition, there exist a few simple definitions which may be useful:
%
% \DescribeMacro\uaref
% The macro |\uaref| acts like the usual |\ref| command but puts an $\uparrow$
% (|\uparrow|) in front of the reference.
%
% \DescribeMacro\toc
% The macro |\toc| replaces the usual |\tableofcontents| to allow for control
% via the |notoc| option.
%
% \DescribeMacro\PyX
% The macro |\PyX| defines the \PyX{} logo as employed by the developers of 
% \PyX{} in their manual.
%
% \DescribeMacro\tipagraph
% In order to explain the pronunciation of \PyX, the |tipa| package is
% needed which cannot be expected to be present in every installation. 
% Therefore, we provide via the |\tipagraph| command a way to alternatively
% include a graphics file. The command is used as follows
% \begin{quote}
% |\tipagraph{|\meta{tipa code}|}{|\meta{graphics filename}|}|
% \end{quote}
%
% \DescribeMacro\cs
% In some places, a backslash cannot be used verbatim, in particular in an
% argument of the |\question| macro. Using
% \begin{quote}
% \verb*+\cs +\meta{command name}
% \end{quote}
% will result in a backslash followed directly by the command name.
%
% \DescribeMacro\us
% The macro |\us| can be used to produce an underscore if there is no more
% direct way to do so.
%
% \DescribeMacro\hat
% Sometimes it may also be difficult to produce a hat character. In such cases, 
% |\hat| can be useful.
%
% \DescribeMacro\cb
% The last macro of this type is |\cb| which allows to enclose an argument in
% curly braces.
%
% \DescribeMacro\ctan
% For references to files on \acro{ctan}, the macro |\ctan| can be used where 
% the single argument should be the location of the file relative to the 
% \acro{ctan} root.
% \DescribeMacro\ftpCTAN
% One of the \acro{ctan} ftp servers is coded into the package via the 
% |\ftpCTAN| macro for direct reference from the \acro{pdf} version of the 
% \acro{faq}.
%
% \DescribeMacro\new\DescribeMacro\changed
% Finally, the two macros |\new| and |\changed| allow to mark questions as
% new or changed with respect to an earlier version of the \acro{faq}. These 
% macros should be used in the second argument of |\question| just behind the 
% text defining the title of a question.
%
% \section{The Description of the Package Code}
% \StopEventually
%
% We first define a few new switches:
% \begin{center} 
% \begin{tabular}{ll}
% |\if@a|& print answered questions if true\\
% |\if@c|& print corrected questions if true\\
% |\if@o|& print outdated questions if true\\
% |\if@t|& print unanswered or not fully answered questions if true\\
% |\ifc@mments|& print comments if true\\
% |\ift@c|& insert table of contents if true
% \end{tabular}
% \end{center}
% By default, we print all questions and the table of contents but no comments.
%    \begin{macrocode}
\newif\if@a \@atrue
\newif\if@c \@ctrue
\newif\if@o \@otrue
\newif\if@t \@ttrue
\newif\ifc@mments \c@mmentsfalse
\newif\ift@c \t@ctrue
%    \end{macrocode}
% Now we define the various options and set the switches according to the
% options' definition given in section~\ref{sec:usage}.
%    \begin{macrocode}
\DeclareOption{all}{\@atrue \@ctrue \@otrue \@ttrue}
\DeclareOption{answered}{\@atrue \@ctrue \@ofalse \@tfalse}
\DeclareOption{checked}{\@afalse \@ctrue \@ofalse \@tfalse}
\DeclareOption{outdated}{\@afalse \@cfalse \@otrue \@tfalse}
\DeclareOption{todo}{\@afalse \@cfalse \@ofalse \@ttrue}
\DeclareOption{unchecked}{\@atrue \@cfalse \@ofalse \@ttrue}
\DeclareOption{comments}{\c@mmentstrue}
\DeclareOption{notoc}{\t@cfalse}
\ProcessOptions\relax
%    \end{macrocode}
% Next, we load some needed packages if they have not been loaded before. 
%    \begin{macrocode}
\RequirePackage{ifthen}
\RequirePackage{remreset}
\RequirePackage{pifont}
%    \end{macrocode}
% The following macro will put a large cross in the margin to mark comments or
% questions with ``to do'' status.
%    \begin{macrocode}
\newcommand{\put@ding}{\mbox{}\marginpar{\Huge\ding{56}}}
%    \end{macrocode}
%
% Now comes the most important part of this package, the macro |\question|.
% This macro has to decide on the basis of the first argument and the
% options used with the package whether the current question is to be printed
% or not.
%
% It is assumed that in the section hierarchy, the question can take the role
% of a subsection or a subsubsection. Therefore, a mechanism has to be 
% provided which tells the question its place in the hierarchy. Independent
% of whether the question acts like a subsection or subsubsection, the layout
% of the question title should be the one of a subsubsection.
%
% We introduce three switches: |\@printtrue| indicates that the question has
% to be printed while |\@margtrue| requests a mark to be put into the margin
% to indicate a question with ``to do'' status and |\@outdatedtrue| will mark a
% question as outdated. 
%    \begin{macrocode}
\newif\if@print
\newif\if@marg 
\newif\if@outdated
%    \end{macrocode}
% Below we will need to find out which level a question corresponds to. If the
% value of the counter \texttt{question} is not equal to 0, the level is
% already known. A value of 1 indicates that the questions are on the level
% of a \texttt{subsection} while a value of 2 implies that the questions are
% on the \texttt{subsubsection} level. Whenever a new section is started, we
% will need to find out the level of the questions in this section. Therefore,
% we reset the counter \texttt{question} whenever a new section is started.
%    \begin{macrocode}
\newcounter{question}[section]
\setcounter{question}{0}
%    \end{macrocode}
% Since we want to set the title of a question always in the fontsize of a 
% \texttt{subsubsection}, we might need to temporarily change the fontsize 
% of the \texttt{subsection} title. In order to be able to restore the original
% value, we save the value of |\size@subsection| which is used in the 
% \acro{KOMA-Script} classes to define the fontsize of the \texttt{subsection} 
% title.
%    \begin{macrocode}
\let\save@sizesubsection\size@subsection
%    \end{macrocode}
%
% Now we are ready to define the |\question| macro. First, defaults are set
% not to print the question and not to print a mark in the margin. If the status
% of the question corresponds to one asked for by the option passed to the
% package, we change the print switch to |\@printtrue|. A ``to do'' question
% will in addition ask to print a mark in the margin by setting |\@margtrue|.
%    \begin{macrocode}
\newcommand{\question}[4]%
{\@printfalse\@margfalse\@outdatedfalse%
 \ifthenelse{\equal{a}{#1}\and\boolean{@a}}{\@printtrue}{}%
 \ifthenelse{\equal{c}{#1}\and\boolean{@c}}{\@printtrue}{}%
 \ifthenelse{\equal{o}{#1}\and\boolean{@o}}{\@printtrue\@outdatedtrue}{}%
 \ifthenelse{\equal{t}{#1}\and\boolean{@t}}{\@printtrue\@margtrue}{}%
%    \end{macrocode}
% If the \texttt{question} counter has the value 0, we need to determine the
% level of the question. The question should at least be on 
% \texttt{subsubsection} level if not higher. So we first assume it to be
% of this level and make corrections later on if necessary. The counter
% \texttt{question} is set to a level different from 0 (in this case to a
% value of 2 although this is nowhere exploited in the code). This counter
% will now be reset not only by a |\section| but also by a |\subsection|.
% Finally, we set the command |\questi@n| which will typeset the question title
% as \texttt{subsubsection}.
%
% In a second step, we check whether the question is on |\subsection| level
% instead of |\subsubsection| level as was assumed before. In this case,
% the counter \texttt{question} is set to 1 (again it is only important that
% the value is different from 0). This counter should no longer be reset
% by a |\subsection| so we remove the corresponding entry from the reset list.
% Finally, the command |\questi@n| typesetting the title of the question is
% set to |\subsection|.
%    \begin{macrocode}
 \ifthenelse{\value{question} = 0}{%
   \setcounter{question}{2}%
   \@addtoreset{question}{subsection}%
   \let\questi@n\subsubsection%
   \ifthenelse{\value{subsection} = 0}%
      {\setcounter{question}{1}%
       \@removefromreset{question}{subsection}%
       \let\questi@n\subsection}%
      {}%
  }{}
%    \end{macrocode}
% Now we are ready to typeset the question if this is what the options passed
% to the package ask for. Before typesetting the question title, we temporarily
% set the fontsize of the |\subsection| title to the fontsize of the 
% |\subsubsection| just in case we are on the |\subsection| level. |\questi@n|
% does the typesetting of the title and a mark is put into the margin if
% requested by |\@margtrue|. 
%
% After the title, a comment, if present and asked for, is typeset in sans 
% serif and small fontsize. Finally, the answer, i.e. the contents of argument 
% 4, is typeset.
%    \begin{macrocode}
  \if@print%
    \let\size@subsection\size@subsubsection
     \questi@n{#2 \if@outdated\outd@ted\fi}\if@marg\put@ding\else\fi%
    \let\size@subsection\save@sizesubsection
   \ifc@mments
    \ifthenelse{\equal{}{#3}}{}
    {\put@ding{\sffamily\small#3}\par}%
   \else\fi
    #4
  \else\fi}
%    \end{macrocode}  
%
% Since it is not possible to use verbatim code in macro arguments, we cannot
% typeset code snippets in an answer by using a \texttt{verbatim} environment.
% We therefore define the environment \texttt{progcode}. In order to ensure
% proper indentation, we make the tilde an active character and define
% it as space preceded by a |\strut| so that the space is not ignored.
% In addition, several layout parameters are set like a global indentation
% of the code and the use of a small fixed space font.
%    \begin{macrocode}
\newenvironment{progcode}
  {\list{}{\leftmargin\parindent\rightmargin\z@}%
         \ttfamily\small%
         \catcode`\~=13%
         \def~{\strut\ }%
         \item\relax}
        {\endlist}
%    \end{macrocode}
% To facilitate references preceded by a $\uparrow$ we define |\uaref| which
% works like the standard |\ref|.
%    \begin{macrocode}
\DeclareRobustCommand\uaref[1]{$\uparrow$\ref{#1}}
%    \end{macrocode}
% The option \texttt{notoc} only works if instead of |\tableofcontents| the
% command |\toc| defined here is used.
%    \begin{macrocode}
\DeclareRobustCommand\toc[0]{\ift@c\tableofcontents\else\fi}
%    \end{macrocode}
% The definition of the \PyX{} logo is copied from the code used by the \PyX{}
% developers in their manual. Here, we also provide the simple string ``PyX''
% for use in the \acro{pdf} bookmarks.
%    \begin{macrocode}
\DeclareRobustCommand\PyX[0]{\texorpdfstring%
           {P\kern-.3em\lower.5ex\hbox{Y}\kern-.18em X}{PyX}%
          }
%    \end{macrocode}  
% We now check for the presence of the |tipa| package. If present, it
% will be loaded and the command |\tipagraph| will hand over its first
% argument to the command |\textipa|. Otherwise, the |graphicx| package
% is loaded and |\tipagraph| will lead to the inclusion of the graphics file
% given in the second argument. The vertical shift of the box has been chosen 
% by trial and error. The horizontal spacing is slightly different in the two 
% alternatives but this should not give to rise to major problems. 
%    \begin{macrocode}
\IfFileExists{tipa.sty}%
  {\usepackage{tipa}%
   \newcommand{\tipagraph}[2]{\textipa{##1}}}%
  {\usepackage{graphicx}%
   \newcommand{\tipagraph}[2]{\raisebox{-0.8ex}{\includegraphics{##2}}}}
%    \end{macrocode}
% When a backslash character is needed as part of a verbatim command name,
% but verbatim code cannot be used, the macro |\cs| can be employed.
% Again, we take care of the requirements of the \acro{pdf} bookmarks.
%    \begin{macrocode}
\DeclareRobustCommand\cs[1]{%
    \texorpdfstring{\texttt{\char`\\}}{\textbackslash}#1%
    }
%    \end{macrocode}
% In order to avoid problems with verbatim code, we define |\us| and |\hat|
% to produce an underscore and a hat, respectively.
%    \begin{macrocode}
\DeclareRobustCommand\us[0]{\texttt{\char`\_}}
\DeclareRobustCommand\hat[0]{\texttt{\char`\^}}
%    \end{macrocode}
% The macro |\cb| encloses its argument in curly braces and should be used
% when verbatim code does not work.
%    \begin{macrocode}
\DeclareRobustCommand\cb[1]{\texttt{\char`\{#1\char`\}}}
%    \end{macrocode}
% For files on \acro{ctan}, we define the macro |\ctan| which is used as follows
% \begin{quote}
% |\ctan{|\meta{file location relative to CTAN root}|}|
% \end{quote}
% In order for the link to connect to a \acro{ctan} server, |\ftpCTAN| contains
% the URL of one of the \acro{ctan} servers which is chosen quite arbitrarily 
% and could be replaced by another \acro{ctan} server. Note that we use
% |\path| instead of |\url| to avoid that a link is created to the second
% argument instead of the first one.
%    \begin{macrocode}
\def\ftpCTAN{ftp://ctan.tug.org/tex-archive/}
\DeclareRobustCommand\ctan[1]{%
    \href{\ftpCTAN#1}{\path{CTAN:#1}}%
}
%    \end{macrocode}
% In order to mark questions as new or changed with respect to a previous
% release of the \acro{faq}, we define the macros |\new| and |\changed| which 
% are intended to be used at the end of the second argument of |\question|.
% No output is generated for \acro{pdf} bookmarks.
%    \begin{macrocode}
\DeclareRobustCommand\new[0]{\texorpdfstring%
           {\quad\raisebox{0.3ex}{\fbox{\tiny\normalfont NEW}}}{}
	  }
\DeclareRobustCommand\changed[0]{\texorpdfstring%
           {\quad\raisebox{0.3ex}{\fbox{\tiny\normalfont CHANGED}}}{}
	  }
%    \end{macrocode}
% Outdated questions should be marked also in the \acro{pdf} bookmarks. 
%    \begin{macrocode}
\DeclareRobustCommand\outd@ted[0]{\texorpdfstring%
           {\quad\colorbox{black}{\color{white}\small OUTDATED}}
		      {~(outdated)}
	  }
%    \end{macrocode}
% \Finale
\endinput
