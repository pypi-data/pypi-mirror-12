\chapter{Module text: \TeX/\LaTeX{} interface}
\label{module:text}

\section{Basic functionality}

The \verb|text| module seamlessly integrates the famous typesetting
technique of \TeX/\LaTeX{} into \PyX. The basic procedure is:
\begin{itemize}
\item start \TeX/\LaTeX{} as soon as text creation is requested
\item create boxes containing the requested text on the fly
\item immediately analyse the \TeX/\LaTeX{} output for errors etc.
\item boxes are written into the dvi output
\item box extents are immediately available (they are contained in the
\TeX/\LaTeX{} output)
\item as soon as PostScript needs to be written, stop \TeX/\LaTeX{},
analyse the dvi output and generate the requested PostScript
\item use Type1 fonts for the PostScript generation 
\end{itemize}

Note that in order that Type1 fonts can be used by \PyX, an
appropriate \verb|psfonts.map| containing entries for the used fonts
has to be present in your texmf tree.

\section{The texrunner}
Instances of the class \verb|texrunner| represent a \TeX/\LaTeX{}
instance. The keyword arguments of the constructor are listed in the
following table:

\medskip
\begin{tabularx}{\linewidth}{l>{\raggedright\arraybackslash}X}
keyword&description\\
\hline
\texttt{mode}&\texttt{"tex"} (default) or \texttt{"latex"}\\
\texttt{lfs}&Specifies a latex font size file to be used with \TeX{} (not in \LaTeX). Those files (with the suffix \texttt{.lfs}) can be created by \texttt{createlfs.tex}. Possible values are listed when a requested name could not be found.\\
\texttt{docclass}&\LaTeX{} document class; default is \texttt{"article"}\\
\texttt{docopt}&specifies options for the document class; default is \texttt{None}\\
\texttt{usefiles}$^1$&access to \TeX/\LaTeX{} jobname files; default: \texttt{None}; example: \texttt{("spam.aux", "eggs.log")}\\
\texttt{waitfortex}&wait this number of seconds for a \TeX/\LaTeX{} response; default \texttt{5}\\
\texttt{texdebug}&filename to store \TeX/\LaTeX{} commands; default \texttt{None}\\
\texttt{dvidebug}&dvi debug messages like \texttt{dvitype} (boolean); default \texttt{0}\\
\texttt{errordebug}&verbose level of \TeX/\LaTeX{} error messages; valid values are \texttt{0}, \texttt{1} (default), \texttt{2}\\
\texttt{dvicopy}&get rid of virtual fonts which \PyX{} cannot handle (boolean); default \texttt{0}\\
\texttt{pyxgraphics}&enables the usage of the graphics package without further configuration (boolean); default 1\\
\texttt{texmessagestart}$^{1,2}$&parsers for the \TeX/\LaTeX{} start message; default: \texttt{texmessage.start}\\
\texttt{texmessagedocclass}$^{1,2}$&parsers for \LaTeX{}s \texttt{\textbackslash{}documentclass} statement; default: \texttt{texmessage.load}\\
\texttt{texmessagebegindoc}$^{1,2}$&parsers for \LaTeX{}s \texttt{\textbackslash{}begin\{document\}} statement; default: \texttt{(texmessage.load, texmessage.noaux)}\\
\texttt{texmessageend}$^{1,2}$&parsers for \TeX{}s \texttt{\textbackslash{}end}/ \LaTeX{}s \texttt{\textbackslash{}end\{document\}} statement; default: \texttt{texmessage.texend}\\
\texttt{texmessagedefaultpreamble}$^{1,2}$&default parsers for preamble statements; default: \texttt{texmessage.load}\\
\texttt{texmessagedefaultrun}$^{1,2}$&default parsers for text statements; default: \texttt{(texmessage.loadfd, texmessage.graphicsload)}\\
\end{tabularx}
\medskip

$^1$
The parameter might contain None, a single entry or a sequence of entries.

$^2$
\TeX/\LaTeX{} message parsers are described in more detail below.

\medskip
The \verb|texrunner| instance provides several methods to be called by
the user. First there is a method called \verb|set|. It takes the same
kewword arguments as the constructor and its purpose is to provide an
access to the \verb|texrunner| settings for a given instance. This is
important for the \verb|defaulttextunner|. The \verb|set| method
fails, when a modification can't be applied anymore (e.g.
\TeX/\LaTeX{} was already started).

The \verb|preamble| method can be called before the \verb|text| method
only (see below). It takes a \TeX/\LaTeX{} expression and optionally
one or several \TeX/\LaTeX{} message parsers. The preamble expressions
should be used to perform global settings, but should not create any
\TeX/\LaTeX{} dvi output. In \LaTeX, the preamble expressions are
inserted before the \verb|\begin{document}| statement. Note, that you
can use \verb|\AtBeginDocument{...}| to postpone the direct
evaluation.

Finally there is a \verb|text| method. The first two parameters are
the \verb|x| and \verb|y| position of the output to be generated. The
third parameter is a \TeX/\LaTeX{} expression and further parameters
are attributes for this command. Those attributes might be
\TeX/\LaTeX{} settings as described below, \TeX/\LaTeX{} message
parsers as described below as well, \PyX{} transformations, and \PyX{}
fill styles (like colors). The \verb|text| method returns a box (see
chapter~\ref{module:box}), which can be inserted into a canvas
instance by its \verb|insert| method to get the text.

Note that for the generation of the PostScript code the \TeX/\LaTeX{}
instance must be terminated. However, a \TeX/\LaTeX{} instance is
started again when the \verb|text| method is called again. A call of
the \verb|preamble| method will still fail, but you can explicitly
call the \verb|reset| method to allow for new \verb|preamble| settings
as well. The \verb|reset| method takes a boolean parameter
\verb|reinit| which can be set to run the old preamble settings.

\section{\TeX/\LaTeX{} settings}

\begin{description}
\raggedright
\item[Horizontal alignment:] \verb|halign.left| (default),
\verb|halign.center|, \verb|halign.right|, \verb|halign(x)| (\verb|x|
is a value between \verb|0| and \verb|1| standing for left and right,
respectively)
\item[Vertical box:] Usually, \TeX/\LaTeX{} expressions are handled in
horizontal mode (so-called LR-mode in \TeX/\LaTeX; everything goes
into a single line). You may use \verb|parbox(x)|, where \verb|x| is the
width of the text, to switch to a multiline mode (so-called vertical
mode in \TeX/\LaTeX).
\begin{figure}
\centerline{\includegraphics{textvalign}}
\caption{valign example}
\label{fig:textvalign}
\end{figure}
\item[Vertical alignment:] \verb|valign.top|, \verb|valign.middle|,
\verb|valign.bottom|; when no \verb|parbox| is defined, additionally
\verb|valign.baseline| (default); when \verb|parbox| is defined,
additionally \verb|valign.topbaseline| (default),
\verb|valign.middlebaseline|, and \verb|valign.bottombaseline|; see
figure~\ref{fig:textvalign} for an example
\item[Vertical shift:] \verb|vshift.char(lowerratio, heightstr="0")|
(lowers the output by \verb|lowerratio| of the height of
\verb|heightstr|), \verb|vshift.bottomzero=vshift.char(0)| (doesn't
have an effect), \verb|vshift.middlezero=vshift.char(0.5)| (shifts
down by half of the height of a \verb|0|),
\verb|vshift.topzero=vshift.char(1)| (shifts down by the height of the a
\verb|0|), \verb|vshift.mathaxis| (shifts down by the height of the
mathematical axis)
\item[Mathmode:] \verb|mathmode| switches the mathmode of \TeX/\LaTeX
\item[Font size:] \verb|size.tiny|, \verb|size.scriptsize|,
\verb|size.footnotesize|, \verb|size.small|, \verb|size.normalsize|
(default), \verb|size.large|, \verb|size.Large|, \verb|size.LARGE|,
\verb|size.huge|, \verb|size.Huge|
\end{description}

\section{Using the graphics-bundle with \LaTeX}

The packages in \LaTeX-graphics bundle (color.sty, graphics.sty,
graphicx.sty, \ldots) make extensive use of \verb|\special| commands. Here
are some notes on this topic. Please install the appropriate driver file
\verb|pyx.def|, which defines all the specials, in your \LaTeX-tree and add
the content of both files \verb|color.cfg| and \verb|graphics.cfg| to your
personal configuration files.\footnote{If you do not know what I am talking about
right now -- just ignore this paragraph, but make sure not to set the
\texttt{pyxgraphics} keyword to 0.} After you have installed the \verb|.cfg|
files please use the \verb|text| module always with the \verb|pyxgraphics|
keyword set to 0, this switches off a hack that might be convenient for less
experienced \LaTeX-users.\medskip

You can then import the packages of the graphics-bundle and related packages
(e.g.~rotating, \ldots) with the option~\verb|pyx|, e.g.{}\\
\rule{0.1\linewidth}{0sp}\verb|\usepackage[pyx]{color,graphicx}|\\
Please note that the option~\verb|pyx| is only available with
\verb|pyxgraphics=0| and a properly installed driver file. Otherwise do not
use this option, omit it completely or say~\verb|[dvips]|.\medskip

When defining colours in \LaTeX\ as one of the colour models \{\verb|gray|,
\verb|cmyk|, \verb|rgb|, \verb|RGB|, \verb|hsb|\} then pyx will use the
corresponding values (one to four real numbers) for output. When you use one of the
\verb|named| colors in \LaTeX\ then pyx will use the corresponding predefined
colour (see module \verb|color| and the colour table at the end of the
manual).

When importing eps-graphics in \LaTeX\ then pyx will rotate, scale and clip
your file like you expect it. Note that pyx cannot import other graphics files
than eps at the moment.\medskip

For reference purpose, the following specials can be handled by the
\verb|text| module at the moment:\medskip

\begingroup
\leftskip3em
\parindent-3em
\parskip0.5ex
\texttt{PyX:color\_begin (model) (spec)}\\
  starts a colour. (model)~is one of
  \{\verb|gray|, \verb|cmyk|, \verb|rgb|, \verb|hsb|, \verb|texnamed|\}.
  (spec)~depends on the model: a name or some numbers.\par
\texttt{PyX:color\_end} ends a colour.\par
\texttt{PyX:epsinclude file= llx= lly= urx= ury= width= height= clip=0/1}\\
  includes an eps-file. The values of llx to ury are in the files' coordinate
  system and specify the part of the graphics that should become the specified
  width and height in the outcome. The graphics may be clipped. The last three
  parameters are optional.\par
\texttt{PyX:scale\_begin (x) (y)}\\
  begins scaling from the current point.\par
\texttt{PyX:scale\_end} ends scaling.\par
\texttt{PyX:rotate\_begin (angle)} begins rotation around the current
  point.\par
\texttt{PyX:rotate\_end} ends rotation.\par
\endgroup


\section{\TeX/\LaTeX{} message parsers}

Message parsers are used to scan the output of \TeX/\LaTeX. The output
is analysed by a sequence of message parsers. Each of them analyses
the output and remove those parts of the output, it feels responsible
for. If there is nothing left in the end, the message got validated,
otherwise an exception is raised reporting the problem.

\medskip
\begin{tabular}{ll}
parser name&purpose\\
\hline
\texttt{texmessage.load}&loading of files (accept \texttt{(file ...)})\\
\texttt{texmessage.loadfd}&loading of files (accept \texttt{(file.fd)})\\
\texttt{texmessage.graphicsload}&loading of graphic files (accept \texttt{<file.eps>})\\
\texttt{texmessage.ignore}&accept everything as a valid output\\
\end{tabular}
\medskip

More specialised message parsers should become available as required.
Please feal free to contribute (e.g. with ideas/problems; code is
desired as well, of course). There are further message parsers for
\PyX{}s internal use, but we skip them here as they are not
interesting from the users point of view.

\section{The defaulttexrunner instance}
The \verb|defaulttexrunner| is an instance of the class
\verb|texrunner|, which is automatically created by the \verb|text|
module. Additionally, the methods \verb|text|, \verb|preamble|, and
\verb|set| are available as module functions accessing the
\verb|defaulttexrunner|. This single \verb|texrunner| instance is
sufficient in most cases.

%%% Local Variables:
%%% mode: latex
%%% TeX-master: "manual.tex"
%%% End: